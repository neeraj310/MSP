\documentclass[a4paper,12pt]{scrreprt}
\usepackage{scrlayer}
\DeclareNewLayer[
    foreground,
    contents={%
      \parbox[b][\layerheight][c]{\layerwidth}
        {\centering (This page intentionally left blank)}%
    }
  ]{blankpage.fg}
\DeclarePageStyleByLayers{blank}{blankpage.fg}
\usepackage{scrhack}
\usepackage{lscape}
\usepackage{tablefootnote}
\usepackage[utf8]{inputenc}
\usepackage[table,xcdraw]{xcolor}
\usepackage[T1]{fontenc}
\usepackage{silence}
\WarningFilter{scrreprt}{Usage of package `fancyhdr'}
\usepackage{graphicx}
\usepackage{tikz}
\usepackage{pgfplots}
\usepackage{times}
\usepackage{listings}
\usepackage{amssymb}
\usepackage{amsfonts}
\usepackage{amsmath}
\usepackage{multirow}
\usepackage[english]{babel}
\usepackage[utf8]{inputenc}
\usepackage{caption}
\usepackage[noend]{algpseudocode}
\usepackage[ruled,vlined,linesnumbered]{algorithm2e}
\usepackage{tabularx}
\usepackage{booktabs}
\usepackage[tikz]{mdframed}
\usepackage[toc,page]{appendix}
\usepackage{color}   %May be necessary if you want to color links
\usepackage{hyperref}
\usepackage{pgfplots}
\usepackage{standalone}
\usepackage[labelfont=bf,format=plain,justification=raggedright,singlelinecheck=false]{caption}
\usepackage{mathtools}
\usetikzlibrary{matrix}
\usepackage{float}
\usepackage{subcaption}
\usepackage{placeins}
\usetikzlibrary{shapes}

\usepgfplotslibrary{groupplots}

\hypersetup{
    colorlinks,
    citecolor=black,
    filecolor=black,
    linkcolor=black,
    urlcolor=black
}
\definecolor{pastelgray}{rgb}{0.81, 0.81, 0.77}

\newcounter{example}[chapter]
\newenvironment{mscexample}[1][]
	{\refstepcounter{example}\par\medskip
		\begin{mdframed}[backgroundcolor=pastelgray, topline=false, leftline=false, rightline=false]
		\noindent \textbf{Example~\thechapter.\theexample #1} \rmfamily
		} {
		\end{mdframed}
	} {\medskip}
	
\newcounter{proof}[chapter]
\newenvironment{mscproof}[1][]
	{\refstepcounter{proof}\par\medskip
		\noindent \textbf{Proof~\thechapter.\theproof #1} \rmfamily
		}{ $\hfill\blacksquare$
	} {\medskip}

\newcounter{conclusion}[chapter]
\newenvironment{mscconclusion}[1][]
	{\refstepcounter{conclusion}\par\medskip
		\begin{mdframed}[backgroundcolor=lightgray, topline=false, leftline=false, rightline=false]
		\noindent \textbf{Conclusion~\thechapter.\theconclusion #1} \rmfamily
		}{
		\end{mdframed}
	} {\medskip}
	
\newenvironment{mscsummary}{\fbox{Summary}}{\\}

\newcommand{\comm}[1]{}
\DeclarePairedDelimiter\ceil{\lceil}{\rceil}
\DeclarePairedDelimiter\floor{\lfloor}{\rfloor}

\titlehead{Institut für Informatik, Universität Zürich}
\subject{\vspace*{2cm}MSc Project Report}
\title{Implementing Learned Indexes on 1 and 2 Dimensional Data}
\author{
  Neeraj Kumar, Nivedita Nivedita, Xiaozhe Yao\\[-5pt]
  \scriptsize Matrikelnummer: 19-765-189, 19-756-303, 19-759-570\\[-5pt]
  \scriptsize Email: \texttt{\{neeraj.kumar,nivedita.nivedita, xiaozhe.yao\}@uzh.ch}
}
\date{\vspace*{2cm}\today}
\publishers{
  \small supervised by \\ 
  Prof.\ Dr.\ Michael H. Böhlen and \\ Mr. Qing\ Chen \\[5cm]
  \begin{tikzpicture}[overlay]
    \node at (-3,-3) {\includegraphics[height=1.5cm]{IFIlogo}};
    \node at (7,-3) {\includegraphics[height=1.5cm]{dbtgBW}};
  \end{tikzpicture}
}
 
\newtheorem{definition}{Definition}
\newtheorem{example}{Example}
\newtheorem{theorem}{Theorem}
\newtheorem{lemma}{Lemma}
\newcommand{\comment}[1]{}
\begin{document}

\begingroup
\let\newpage\relax%
\maketitle
\newpage\null\thispagestyle{blank}\newpage
\setcounter{page}{0}
\endgroup

\begin{abstract}
Databases use indexes to find records efficiently. Among these indexes, B-tree and KD-tree are two successful indexes used for 1-dimensional and 2-dimensional data. In this project, we implement both the learned index \cite{kraska2018case} for 1-D data and the learned index, named LISA \cite{li2020lisa} for 2-D data from scratch using Python. Afterwards, we have conducted sanity check to ensure that our implementations are correct. In addition to the implementation and evaluation, we have theoretically analyse some properties that the learned indexes hold. Beyond that, we also explore and discuss the properties that the learned indexes should hold.

As an extension to the existing learned indexes, we explore the possibilities of using convolution operation and convolutional neural network as learned indexes and report our results. 
\end{abstract}

\setcounter{tocdepth}{2}
\tableofcontents 

\chapter{Introduction}

Over the years, indexes have been widely used in databases to improve the speed of data retrieval. In the past decades, the database indexes generally fall into hand-engineered data structures and algorithms, such as B-Tree, KD-Tree, Hash Table, etc. These indexes have played an important role in databases and have been widely used in modern data management systems (DBMS). Despite their success, they do not consider the distribution of the database entries, which might be helpful in designing faster indexes.

For example, if the dataset contains integers from $1$ to $1$ million, the key can be used directly as an offset. With the key used as an offset, the values with the key can be retrieved in $\mathcal{O}(1)$ time complexity. Compared with B-Tree, which always takes $\mathcal{O}(\log n)$ time complexity for the same query. At the same time, by using the key as an offset directly, we do not need any extra overhead regarding memory space, where the B-Tree needs extra $\mathcal{O}(n)$ space complexity to save the tree.

From the above example, we found there are two promising advantages of learned indexes over hand-engineered indexes:
\begin{enumerate}
  \item Learned indexes may be faster when performing queries, especially when the number of entries in the database are extremely huge.
  \item Learned indexes may take less memory space, as we only need to save the model with constant size.
  \end{enumerate}
  
 We will explore and analyse these two advantages qualitatively in the \textit{chapter 5}. 

Nowadays, to leverage these two advantages, researchers proposed learned indexes \cite{kraska2018case}, where machine learning techniques are applied to automatically learn the distribution of the database entries and build the data-driven indexes. This approach has been shown to be powerful and competitive compared with hand-engineered indexes, such as B-Tree.

In this report, we explore the development of database indexes, from hand-engineered indexes to the learned index. After that, we explore the possibilities of using complex convolutional neural networks as database indexes. This report is organised into the following chapters:

\begin{enumerate}
	\item \textbf{Introduction}. In this chapter, we illustrate the organisation of this report. Besides, we go through the modern computer systems and introduce the general information about database indexes.
	\item \textbf{Implementation}. In this chapter, we thoroughly describe the implementation of one and two dimensional indexes, including B-Tree, baseline learned index, recursive model, KD-Tree and LISA.
	\item \textbf{Evaluation}. In this chapter, we perform evaluation among the indexes we implemented with different evaluation dataset. 
	\item \textbf{Insights and Findings}. We demonstrate our findings during the implementation in this chapter. Besides, we also discuss the advantages and disadvantages of different indexes.
	\item \textbf{Conclusions}. 
\end{enumerate}

\section{Notations}

In this report, we will use the following notations:

\begin{table}[h]
\begin{tabularx}{\textwidth}{@{}XX@{}}
\toprule
  \underline{Sets and Spaces} \\
  $\mathbb{R}$ & The set of real numbers \\
  $\mathbb{R}^d$ & The set of $d$ dimensional real space \\
  \underline{Random Variables} \\
  $\textbf{X}$ & A vector or matrix \\
  $x$ & A single value in $\textbf{X}$ \\
  $(x,y)$ & A tuple contains two values \\	
  \underline{Hyper-Parameters} \\
  N   & A pre-set hyper parameter \\
  \underline{Functions} \\
  $\mathcal{LR}$ & Linear Regression Function\\
  $\mathcal{P}$ & Polynomial Function\\
  $\mathcal{M}$ & Mapping Function\\
  $\mathcal{O}$ & Big-O notation for complexity\\
  \underline{Others} \\
  $\clubsuit$ & End of Example \\
  $\blacksquare$ & End of Proof \\
\bottomrule
\end{tabularx}
\end{table}


\section{Terminologies}

In the following chapters, we will use the following terminologies

\textbf{Index model} is a function that maps the index of a row of data into the location (e.g. page index) of the data. For example, in one-dimensional case, the index models include B-Tree, Linear Regression models, etc.

\textbf{Key} is a special attribute in the database that could identify a record. In our work, the key could be a scalar in one-dimensional case, or a $(x,y)$ pair in two-dimensional case.

\textbf{Order of a tree} is the maximum number of children that a node can have.

\textbf{Internal node} is any node of a tree that has child nodes and is not a root node.

\textbf{Leaf node} is any node that does not have child nodes.

\textbf{Level} of a node is defined as the number of edges between this node and the root node.


\section{Motivation}

In traditional database indexes, the complexity for locating an item is usually bounded by some function related to the total number of elements. For example, with a B-Tree, an item can be found within $\mathcal{O}(\log n)$ time complexity. In the meantime, saving a B-Tree as index takes $n$ space complexity. With the rapid growing of the volume of data, $n$ becomes much larger than ever before. Hence, the big data era is calling for a database index that have constant complexity in both time and space.

To achieve such a goal, the distribution of the data is important. For example, assume that the data is fixed-length records over a set of continuous integers from 1 to 100 million, the conventional B-Tree index can be replaced by the keys themselves, making the query time complexity an $\mathcal{O}(1)$ rather than $\mathcal{O}(\log n)$. Similarly, the space complexity would be reduced from $\mathcal{O}(n)$ to $\mathcal{O}(1)$. This example shows that with the knowledge of the distribution of the data, it is possible to locate the item in database in constant time.

Formally, we define the index of each record as $x$ and the corresponding location as $y$ and we represent the whole data as $(X, Y)$ pairs with the total number of pairs defined as $N$. We could then normalise the $Y$ into $\tilde{Y}\in[0,1]$ so that the $\tilde{y}$ represents the portion of the $y$ among the whole $Y$. With these definitions, we can then define a function $F:X\to \tilde{Y}$ that maps the index into the portion of the $y$. We have $y=F(x)* N$. As the output of this function can be considered as the probability of $X\leq x$, we can regard this function $F(x)$ as the cumulative distribution function (CDF) of $X$, i.e. $F(x)=\mathbb{P}(X\leq x)$. Now that $N$ is determined by the length of data records, we only need to learn such CDF and we called the learned CDF function as \textbf{learned index model}.

From the perspective of the distribution of data records, our previous example can be rephrased as following. Our data records are $(X, Y)$ pairs with a linear relation, i.e. $y=x, \forall y\in Y$. We are looking for a function $F$ such that $y=x=F(x)* N$, and hence we end up with $F(x)=\frac{1}{N}*x$. If we use this linear function $F(x)$ as the index model, then we could locate the data within $\mathcal{O}(1)$ time complexity and we only need to store the total number of records as the only parameter. Compared with B-Tree and other indexes, the advantages are enormous.

Even though there might be potential advantages, the learned index model has several assumptions, as listed below.
\begin{enumerate}
	\item All data records are stored in memory. 
	\item All data records are sorted by $X$.
	\item All data records are stored statically in database, hence we do not take insertion and deletion into consideration.
\end{enumerate}




\chapter{Implementation}

\begin{mscsummary}
	In this chapter, we first describe how we implement the B-Tree, Baseline model and recursive model for one-dimensional data. After that, we illustrate the implementation of LISA and LISA Baseline, which are two index models for two-dimensional data. At the end, we describe how we use these indexes to perform point query, range query and $K$NN query.
\end{mscsummary}

\section{One Dimensional Data}

\subsection{B-Tree}

B-Tree and its variants have been widely used as indexes in
databases. For example, the PostgreSQL uses B-Tree as its index. B-Trees
can be considered as a natural generalisation of binary search tree. In
binary search tree, there is only one key and two possible children in
the internal node. However, an internal node of B-Tree can contain
several keys and children. The keys in a node serve as dividing points
and separate the range of keys. With this structure, we make a
multi-way decision based on comparisons with the keys stored at the node
$x$. The image below illustrates a simple B-Tree.

In this section, we introduce the construction and query processes
of B-Trees and then analyse their properties.

\subsubsection{Motivation}

In computers, the memories are organised in an hierarchical way. For example, a classical computer system consists three layers of memory: the CPU cache, main memory and the hard disk. In such a system, the CPU cache is the fastest but the most expensive while and hard disk is the cheapest but also the slowest. When querying for an item, the CPU will first try to fetch it from the CPU cache. If not there the CPU will then try to fetch it from the main memory, and then the hard disk.

%TODO: Some statements are needed to show the data is stored in blocks in the memory.

At the same time, the traditional hard disk drive (HDD) is made by a moving mechanical structure.

%TODO: Illustrate the mechanical structure.
% Should we be writing this since we are using in-memory ?

In summary, there are two properties in classical computer systems that we need to take into account:

\begin{enumerate}
	\item The memory is not flat, meaning that memory references are not equally expensive. 
	\item 
\end{enumerate}

\subsubsection{Definition and Terms}
\begin{figure}[htp]
    \centering
    \includegraphics[width=0.6\textwidth]{graphs/B-Tree_definitions.png}
    \caption{B-Tree}
    \label{fig:B-Tree}
\end{figure}

Before we formally define B-Trees, we assume the following terms:

\begin{itemize}
\item
  \textbf{Keys}: The key in a database is a special attribute that could identify a row in the database. In our work, each key corresponds to a 1-dimensional
  \textbf{value} and forms a key-value pair.
\item
  \textbf{Order}: The Order of a B-Tree is the maximum number of children that a node can have. Number of keys in a node is always one less than the order of the tree at the maximum.
\item
  \textbf{Internal Node}: An internal node is any node of the tree that has child nodes and is not a root node.
\item
  \textbf{Leaf Node}: A leaf node is any node that does not have child nodes.
\end{itemize}

Each node in a B-Tree has the following attributes:

\begin{itemize}
\item
  $x.n$ is the number of keys currently stored in the node $x$.
\item
  Inside each node, the keys are sorted in non decreasing order, so that
  we have $x.keys_1\leq x.keys_2\leq\cdots\leq x.keys_{x.n}$.
\item
  $x.leaf$, a Boolean value determines if current node is a leaf node.
\end{itemize}

With these properties, A B-Tree $T$ whose root is $T.root$ have the
following properties:

\begin{itemize}
\item
  Each internal node $x$ contains $x.n+1$ children. We assume the
  children are $x.c_1,\cdots,x.c_{x.n+1}$.
\item
  The nodes in the tree have lower and upper bounds on the number of
  keys that can contain. These bounds can be expressed in terms of a
  fixed integer $t$.
\end{itemize}

\subsubsection{Insertion of B-Tree}

When inserting keys into a binary search tree, we search for the leaf
position at which to insert the new key. However, with B-Tree, we cannot
simply find the position, create a new node and insert the value because
the tree will be imbalanced again. Hence, in this section, we illustrate
an operation that splits a full node around its median key

\begin{algorithm}[H]
    \SetAlgoLined
    \SetKwInOut{Input}{Input}
     \Input{\texttt{$m$:order\_of\_tree ,$(k,v)$:(key, value), $N$:Node;}}
    \SetKwInOut{Output}{Output}
     \Output{\texttt{B-Tree}}
     \eIf{$N$ is a leaf and not yet full}  
     {
        \texttt{insert $(k,v)$ into $N$}
     }
     {
        \texttt{create new Node N'}\\
        \texttt{Find the median of the node}\\
        \texttt{Add the value at the median location to the new node N'}\\
     }
    \If{$N$ is root with no children and not full}
    {
    
        \texttt{insert $(k,v)$ into $N$}
        
    }
     \caption{Algorithm for B-Tree insertion}
     \label{B-Tree Insertion}
\end{algorithm}


\begin{figure}[htp]
    \centering
    \includegraphics[width=0.6\textwidth]{graphs/B-Tree_example01.png}
    \caption{B-Tree key insertion}
    \label{fig:B-Tree key insertion1}
\end{figure}

\begin{figure}[htp]
    \centering
    \includegraphics[width=0.3\textwidth]{graphs/B-Tree_example02.png}
    \caption{B-Tree key insertion}
    \label{fig:B-Tree key insertion2}
\end{figure}

\subsubsection{Insertion in a B-Tree}

There are two conditions in insertion:
\begin{itemize}
\item
\textbf{When the node is empty or not created at all}:
In the algorithm, initially when the first key is inserted and there is no root to the tree it will check the condition if there are any nodes and if not create a new one. It will insert the new key in the node and keeps inserting the new keys until it is one less than the order of the B-Tree being created. 
\item
\textbf{When the node is full}:
As soon as the maximum number of allowed children has reached for the root node a new empty node is created. Suppose we have inserted [1,11,21] to a node of a B-Tree with degree 4. Now if we want to insert a new value 31 into the tree, since it has reached the maximum number of children it will find the median of the existing node [1,11,21] which is 11 and increase the level of the B-tree to 2 and make [1] and [21] child nodes of [11]. So now if a new value is to be inserted it can be inserted. Splitting of the node happens each time it reaches it's maximum allowed keys. Now 31 will be compared with 11 and since 31 > 11 it will be inserted in the right child and it will have an updated value of [21,31]. More keys can then be inserted until it reaches it's maximum and splits again. Once the node is split it's parent is also updated to the median value i.e., [11] in this case for nodes [1] and [21,31,41] as can be seen in \ref{fig:B-Tree key insertion2}.
\end{itemize}

\subsubsection{Search in a B-Tree}

\begin{algorithm}[H]
    \SetAlgoLined
    \SetKwInOut{Input}{Input}
     \Input{\texttt{k ; key, root ; Root of the B-Tree}}
    \SetKwInOut{Output}{Output}
     \Output{\texttt{Value associated with key}}
     \For{$i\gets0$ \KwTo $len(Root Node)$}{
         \If{keys in root greater than k}
         {
            \texttt{SEARCH\_CHILD(key)} //Search linearly the child associated with the key location one before\\
            \eIf{Child is a leaf}
            {
                \texttt{Linearly search until the key is reached}
            }
            {
                \texttt{SEARCH\_CHILD(key)}
            }
            
         }
    }
     \caption{Algorithm for B-Tree Search}
     \label{B-Tree Search}
\end{algorithm}

Search in a B-Tree is basically the comparison of the value of the key that needs to be searched with the keys in the node. It first linearly checks the value in the root key and looks for a key value which is greater than the searched key. As soon as it finds a key greater than the searched key it will search in the child of the key before it. For example if we were to search of key [41] in the example of the B-Tree used above, we would first check the value [11] and since it smaller than [31] and there are no keys greater than this in the root node it will search in the right child of the root node. It will then linearly search the node and locate it's associated value and return it. 

\subsection{Baseline Learned Index}

\subsubsection{Overview}

The B-Tree can be regarded as a function $\mathcal{F}$ that maps the key $x$ into its corresponding page index $y$. It is known to us that the pages are allocated in a way that the every $S$ entries are allocated in a page where $S$ is a pre-defined parameter. For example, if we set $S$ to be 10 items per page, then the first page will contain the first 10 keys and their corresponding values. Similarly, the second 10 keys and their corresponding values will be allocated to the second page.

If we know the CDF of $X$ as $F(X\leq x)$ and the total number of entries $N$, then the position of $x$ can be estimated as $p=F(x)*N$ and the page index where it should be allocated to is given by

$$y=\floor{\frac{p}{S}}=\floor{\frac{F(x)*N}{S}}$$  

For example, if the keys are uniformly distributed from $0$ to $1000$, i.e. the CDF of $X$ is defined as $F(X\leq x)=\frac{x}{1000}$ and we set $S=10, N=1001$. Then for any key $x$, we immediately know it will be allocated into $y=\floor{\frac{1000}{10}*\frac{x}{1000}}=\floor{\frac{x}{10}}$. Assume that we have a key $698$, then we can calculate $y=\floor{\frac{698}{10}}=69$. By doing so, the page index is calculated in constant time and space.

In this example, we see that the distribution of $X$ is essential and our goal of learned index in one-dimensional data is to learn such distribution. To do so, we apply two different techniques as the baseline, the polynomial regression and fully connected neural network.

To train such a learned index, we first manually generate the $X$ with respect to a certain distribution. We then save the generated $X$ into a dense array with the length $N$. Then we use the proportional index, i.e. the index of each $x$ divided by $N$ as the expected output $y$.

\subsubsection{Polynomial Regression}
 
The polynomial regression model with degree $m$ can be formalised as 

$$ \hat{y_i}= \beta_0+\beta_1x_i+\beta_2x_i^2+\cdots+\beta_mx_i^m$$ and it can be expressed in a matrix form as below

$$
\begin{bmatrix}
y_1 \\ y_2\\ \vdots \\ y_n 
\end{bmatrix}=\begin{bmatrix}
1 & x_1 & x_1^2 &\cdots & x_1^m \\ 
1 & x_2 & x_2^2 &\cdots & x_2^m \\ 
\vdots \\ 
1 & x_n & x_n^2 &\cdots & x_n^m \\ 
\end{bmatrix}\begin{bmatrix}
\beta_0 \\ \beta_1 \\ \vdots \\ \beta_m 
\end{bmatrix}
$$ which can be written as $Y=\boldsymbol{X}\boldsymbol{\beta}$. 
 
 Our goal is to find $\beta$ such that the sum of squared error, i.e. $\text{S}(\boldsymbol{\beta})=\sum_{i=1}^n(\hat{y}-y)^2$ is minimal. This optimisation problem can be resolved by ordinary least square estimation as shown below.
 
 First we have the error as
 
 \begin{equation}
 \begin{split}
 \text{S}(\boldsymbol{\beta})=||\boldsymbol{y}-\boldsymbol{X} \boldsymbol{\beta}||& =(\boldsymbol{y}-\boldsymbol{X}\boldsymbol{\beta})^T(\boldsymbol{y}-\boldsymbol{X}\boldsymbol{\beta})\\
 	& =\boldsymbol{y}^T\boldsymbol{y}-\boldsymbol{\beta}^T\boldsymbol{X}^T\boldsymbol{y}-\boldsymbol{y}^T\boldsymbol{X}\boldsymbol{\beta}+\boldsymbol{\beta}^T\boldsymbol{X}^T\boldsymbol{X}\boldsymbol{\beta}
\end{split}
 \end{equation}
 
 Here we know that $(\boldsymbol{\beta}^T\boldsymbol{X}^T\boldsymbol{y})^T=\boldsymbol{y}^T\boldsymbol{X}\boldsymbol{\beta}$ is a $1\times 1$ matrix, i.e. a scalar. Hence it is equal to its own transpose. As a result we could simplify the error as
 
 \begin{equation}
 	\begin{split}
 		\text{S}(\boldsymbol{\beta})=\boldsymbol{y}^T\boldsymbol{y}-2\boldsymbol{\beta}^T\boldsymbol{X}^T\boldsymbol{y}+\boldsymbol{\beta}^T\boldsymbol{X}^T\boldsymbol{X}\boldsymbol{\beta}
 	\end{split}
 \end{equation}
 
 In order to find the minimum of $S(\boldsymbol{\beta})$, we differentiate it with respect to $\boldsymbol{\beta}$ as 
 
 \begin{equation}
 	\nabla_{\boldsymbol{\beta}}S=-2\boldsymbol{X}^T\boldsymbol{y}+2(\boldsymbol{X}^T\boldsymbol{X})\boldsymbol{\beta}
 \end{equation}
 
 By let it to be zero, we end up with 
 
 \begin{equation}
 \begin{split}
 	 &	-\boldsymbol{X}^T\boldsymbol{y}+(\boldsymbol{X}^T\boldsymbol{X})\boldsymbol{\beta}=0 \\
 	& \implies \boldsymbol{\beta}= (\boldsymbol{X}^T\boldsymbol{X})^{-1}\boldsymbol{X}^T\boldsymbol{y}
 \end{split}
 \end{equation}
 
\subsubsection{Fully Connected Neural Network}

\subsection{Recursive Model Index}

%TODO: there should be some graph to demonstrate the last-mile problem.

In our baseline models, it is not very difficult to reduce the mean square error from millions to thousands. However, it is much harder to reduce it from thousands to tens. This is the so called last-mile problem.

In order to solve this problem, recursive model index was proposed \cite{kraska2018case}. The idea is to split the whole set of data into smaller pieces and assign each piece an index model. By doing so, each model is only responsible for a small range of keys. Ideally, in each smaller range, the keys are distributed in a way that is easier to be learned by our index models, such as polynomial model, fully connected model or even traditional B-Tree model.

As shown in Fig. \ref{rmi_structure}. A recursive model can be regarded as a tree structure, which contains a root model that receives the full dataset for training. Then the root model will split the dataset into several parts. Each sub-model will then receive one part of the full dataset. Then we train the sub-models one by one with the partial training dataset. 

\begin{mscexample}
	For example, in the Fig. \ref{rmi_structure}, the full dataset will be split into three parts and each sub-model receives one part. To train this recursive model, we first train the root model with the whole dataset. Then the root model will split the dataset into 3 parts according to the predicted value of each data point in the dataset. Then each sub-model will receive one part and we train the sub-model accordingly.
\end{mscexample}

\begin{figure*}[h]
\centering
\includegraphics[scale=0.4]{graphs/implementation/one-dim/rmi_demo.pdf}
\caption{An example recursive model index with one root model and three leaf model.}
\label{rmi_structure}
\end{figure*}

\subsubsection{Properties}

Similar to a tree, we define the following terms in a recursive model:

\begin{enumerate}
	\item \textbf{Node Model}. Every node is responsible for making decisions with given input data. In one dimensional case, it can be regarded as a function $f:\mathbb{R}\to\mathbb{R}, x\to y$ where $x$ is the input index and $y$ is the corresponding page block. In principle, each node can be implemented as any machine learning model, from linear regression to neural network, or a traditional tree-based model, such as B-Tree.
	\item \textbf{Internal Node Model}. Internal nodes are all nodes except for leaf nodes and the root node. Every internal node receives a certain part of training data from the full dataset, and train a model on it. 
\end{enumerate}

In the following sections, we will use the notations defined below:
\begin{enumerate}
	\item $N_M^{(i)}$ is the number of models in the $i$th stage.
	%TODO: more notations
	%TODO: modify algorithms accordingly
\end{enumerate}


\subsubsection{Training}

In order to construct a recursive model, we need to have several parameters listed below:
\begin{enumerate}
	\item The training dataset, notated as $(X, Y)$ with entries notated as $(x,y)$.
	\item The number of stages, notated as $N_S$. It is an integer variable.
	\item The number of models at each stage, notated as $N_M$. It is a list of integer variable. $N_M^{(i+1)}$ represents the number of models in the $i$th stage.
\end{enumerate}

The training process of recursive model is an up-bottom process. There will be only one root model that receives the whole training data. After the root model is trained, we iterate over all the training data and predict the page by the root model. After the iteration, we get a new set of pairs $(X, Y_0)$. Then we map $\forall y_0\in Y_0$ into the selected model id in next stage by $\texttt{next}=y_0 * N_M^{(i+1)}/\texttt{max(Y)}$.

% REVISE THIS
\begin{algorithm}[H]
    \SetAlgoLined
    \SetKwInOut{Input}{input}
    \Input{\texttt{$N_S$: A scalar representing the number of stages; \\ $N_M$: An array representing the number of models at each stage; \\ x; y}}
     \texttt{trainset=[[(x,y)]]} \\
     \texttt{stage$\gets 0$} \\
     \While{\texttt{stage} \textless \texttt{$N_S$}}{
      \While{\texttt{model} \textless \texttt{$N_M$[stage]}} {
        \texttt{model.train(trainset[stage][model])} \\
        \texttt{models[stage].append(model)}
      }
      \uIf{\texttt{stage<$N_S$-1}} {
        \For{\texttt{$i\gets0$ \KwTo $len(x)$}}{
        	\texttt{next\_model = 0}\\
        	\For{\texttt{$j\gets 0$ \KwTo stage-1}} {
        		\texttt{output = models[stage][next\_model]} \\
        		\texttt{next\_model = output * $N_M$[stage+1]/max\_y}\\
        	}
            	\texttt{model = models[next\_model]} \\
            	\texttt{output = model.predict(x[i])} \\
            	\texttt{next = output * $N_M$[stage+1]/max\_y} \\
            	\texttt{trainset[stage+1][next].add((x[i],y[i]))}
        }
      }
     \texttt{stage=stage+1}
     }
     \caption{Training of Recursive Model Index}
\end{algorithm}

\subsubsection{Polynomial Internal Models}

In the recursive model index, we use internal models to learn the CDF of a part of the full training data. In order to learn the CDF, we need to know or assume the distribution of a specific part of the data. In this report, we support the following distributions.

\begin{table}[h]
  \begin{tabularx}{\textwidth}{@{}XX@{}}
  \toprule
    Linear Regression & $wx+b$ \\
    Quadratic Regression & $ax^2+bx+c$ \\
    B-Tree & N/A \\
    Fully Connected Neural Network & N/A \\
  \bottomrule
  \end{tabularx}
  \end{table}

Here we describe how we fit a polynomial model.

The polynomial regression model with degree $m$ can be formalised as 

$$ \hat{y_i}= \beta_0+\beta_1x_i+\beta_2x_i^2+\cdots+\beta_mx_i^m$$ and it can be expressed in a matrix form as below

$$
\begin{bmatrix}
y_1 \\ y_2\\ \vdots \\ y_n 
\end{bmatrix}=\begin{bmatrix}
1 & x_1 & x_1^2 &\cdots & x_1^m \\ 
1 & x_2 & x_2^2 &\cdots & x_2^m \\ 
\vdots \\ 
1 & x_n & x_n^2 &\cdots & x_n^m \\ 
\end{bmatrix}\begin{bmatrix}
\beta_0 \\ \beta_1 \\ \vdots \\ \beta_m 
\end{bmatrix}
$$ which can be written as $Y=\boldsymbol{X}\boldsymbol{\beta}$. 
 
 \begin{mscproof}
 	 Our goal is to find $\beta$ such that the sum of squared error, i.e. $$\text{S}(\boldsymbol{\beta})=\sum_{i=1}^n(\hat{y}-y)^2$$ is minimal. This optimisation problem can be resolved by ordinary least square estimation as shown below.
 
 First we have the error as
 
 \begin{equation}
 \begin{split}
 \text{S}(\boldsymbol{\beta})=||\boldsymbol{y}-\boldsymbol{X} \boldsymbol{\beta}||& =(\boldsymbol{y}-\boldsymbol{X}\boldsymbol{\beta})^T(\boldsymbol{y}-\boldsymbol{X}\boldsymbol{\beta})\\
 	& =\boldsymbol{y}^T\boldsymbol{y}-\boldsymbol{\beta}^T\boldsymbol{X}^T\boldsymbol{y}-\boldsymbol{y}^T\boldsymbol{X}\boldsymbol{\beta}+\boldsymbol{\beta}^T\boldsymbol{X}^T\boldsymbol{X}\boldsymbol{\beta}
\end{split}
 \end{equation}
 
 Here we know that $(\boldsymbol{\beta}^T\boldsymbol{X}^T\boldsymbol{y})^T=\boldsymbol{y}^T\boldsymbol{X}\boldsymbol{\beta}$ is a $1\times 1$ matrix, i.e. a scalar. Hence it is equal to its own transpose. As a result we could simplify the error as
 
 \begin{equation}
 	\begin{split}
 		\text{S}(\boldsymbol{\beta})=\boldsymbol{y}^T\boldsymbol{y}-2\boldsymbol{\beta}^T\boldsymbol{X}^T\boldsymbol{y}+\boldsymbol{\beta}^T\boldsymbol{X}^T\boldsymbol{X}\boldsymbol{\beta}
 	\end{split}
 \end{equation}
 
 In order to find the minimum of $S(\boldsymbol{\beta})$, we differentiate it with respect to $\boldsymbol{\beta}$ as 
 
 \begin{equation}
 	\nabla_{\boldsymbol{\beta}}S=-2\boldsymbol{X}^T\boldsymbol{y}+2(\boldsymbol{X}^T\boldsymbol{X})\boldsymbol{\beta}
 \end{equation}
 
 By let it to be zero, we end up with 
 
 \begin{equation}
 \begin{split}
 	 &	-\boldsymbol{X}^T\boldsymbol{y}+(\boldsymbol{X}^T\boldsymbol{X})\boldsymbol{\beta}=0 \\
 	& \implies \boldsymbol{\beta}= (\boldsymbol{X}^T\boldsymbol{X})^{-1}\boldsymbol{X}^T\boldsymbol{y}
 \end{split}
 \end{equation}
 
 \end{mscproof}

\section{Two Dimensional Data}

\subsection{$K$D-Tree}

Similar to search with binary search tree, we also need to traverse the tree in order to perform point query. However, we need to switch the dimensions when we compare the values between the query key and the values in the nodes.

\begin{algorithm}[H]
    \SetAlgoLined
    \SetKwInOut{Input}{Input}
    \SetKwInOut{Output}{Output}
    \Input{\texttt{t}: The node being searched; \texttt{x}: The query key; \texttt{cd}: Current dimension}
    \Output{\texttt{n}: the node that contains the query key}
    \texttt{DIM=2}\\
    \uIf{\texttt{t==NULL}} {
    	\Return \texttt{NULL}
    }
    \uIf{\texttt{x[0]==t.data[0]} and \texttt{x[1]==t.data[1]}} {
    	\Return \texttt{t}
    }
    \uElseIf{\texttt{x[cd]<t.data}} {
    	\Return \texttt{pointSearch(t.left, x, (cd+1) \% DIM)}
    }
    \uElseIf{\texttt{x[cd]>t.data}} {
    	\Return \texttt{pointSearch(t.right, x, (cd+1) \% DIM)}
    }
    \caption{Point Query with $K$D-Tree}
    \label{algo: point_query_kdtree}
\end{algorithm}

The point query works in the following steps:

\begin{enumerate}
	\item From Line $2$ to $3$, we first check if current node is \texttt{NULL}. If so, that means that we have already traversed all the possible nodes and found nothing. In this case, the query returns \texttt{NULL}.
	\item From Line $4$ to $5$, we check if the current node contains the same key as the query key. If so, the current node is the node that we are looking for. Hence, we return the current node in this case.
	\item Otherwise, from Line $6$ to $9$, we check if the current dimension of the query key is smaller, larger or equal to the current dimension of the data in the node.
	\begin{enumerate}
	\item If it is smaller, then we search on the left subtree of current node, with the same query key and switched dimension.
	\item If it is larger, then we search on the right subtree of current node, with the same query key and switched dimension.
	\end{enumerate}
\end{enumerate}

\begin{mscexample}    
	In the previous figure \ref{fig:kd_tree_example}, we showed an example $K$D-tree. If we want to search for $(50, 30)$ in this tree, we would follow the following steps:
	\begin{enumerate}
		\item We first check the root node and compares the $x$-coordinate. As $50>30$, we go to the right subtree of the root node.
		\item Then in the subtree, we compare the $y$-coordinate. As $50<70$, we go to the left subtree of this node.
		\item Then in the left subtree, the termination condition is reached, hence we return this node as result.
	\end{enumerate}
\end{mscexample}


\subsection{Learned 2D Index Overview}

It is difficult to apply traditional $K$NN query pruning strategies applicable for $K$D-Trees, to LISA model as it doesn't maintain a tree like structure with all nodes and
entries based on MBRs (minimum bounding rectangle) and parent-children relationships. %Shard boundaries are learned per mapped interval and no data structure is maintained to refer to shards in adjacent mapped intervals. 
The key idea in LISA paper $K$NN query implementation is to convert it into a range query by estimating an appropriate query range. LISA paper suggests a learning model to learn an appropriate distance bound from underlying training data for every query point and specific value of K. However, we have used empirically estimates to learn this distance bound for different values of $K$. This distance bound is used to convert the $K$NN query to range query.The query range is augmented if less than K neighbors are found in a range query. 

Consider a query point $q_{knn}=(x_{0},x_{1})$, let $x^{'} \in V$ be the $K$th nearest key to $x$ in database at a distance value $\delta = \| x^{'}-q_{knn}\|_{2} $. Lets define $ \mathcal{Q}(q_{knn},\delta) \triangleq [x_{0}-\delta, x_{0}+\delta) \times[x_{1}-\delta, x_{1}+\delta)$ and $\mathcal{B}(q_{knn}, \delta)  \triangleq \{p \in V \mid \| q_{knn}-p\|_{2} \leq \delta \} $. We can create a query rectangle $qr =  \mathcal{Q}(q_{knn}, \delta + \epsilon)$ where $\epsilon \rightarrow 0$. As shown in Fig. \ref{fig:KNN_Query_Lisa}, K nearest keys to $q_{knn}$ are all in $\mathcal{B}(q_{knn}, \delta)$ and thus in $\mathcal{Q}$. $K$NN query can be solved using the range query if we can estimate an appropriate distance bound $\delta$ for every query point.

\begin{figure*}[t]
    \centering
    \includegraphics[width=0.7\textwidth]{graphs/KNN_Query_Lisa.png}
    \caption{KNN Query Implementation in Lisa(K=3)\\
    1)$q_{knn}$ represents the query point, $ \mathcal{Q}(x,\delta) \triangleq [x_{0}-\delta, x_{0}+\delta) \times[x_{1}-\delta, x_{1}+\delta)$, represents query rectangle and $ \mathcal{B}(x, \delta)$ represents the key space at distance $\delta$ containing K nearest keys.\\
    2)KNN query can be solved by range query if we can estimate an appropriate distance bound $\delta$ for every query point\\
    }
    \label{fig:KNN_Query_Lisa}
\end{figure*}
In our experiments, we find the $\delta$ empirically. We try with different values of $\delta$ and choose the one for which we get the best results. 

\section{Queries}

\subsection{Point Query}

A point query is a database operation that finds the records that exactly match our query conditions. In this project, we perform point query on $1$-dimensional data and $2$ dimensional data. We assign the database records into pages, predict the page index with the index models and then perform sequential search on the predicted page. In order to evaluate the errors that different index models are making, we focus on predicting the page indices and ignore the sequential search operation on a specific page. 

\begin{mscexample}
For example, assume we have an $1$-dimensional array $[1,2,3,4]$ and two pages such that $[1,2]\in P_0$ and $[3,4]\in P_1$. A point query for $x=2$ is expected to return 0 as the page index.
\end{mscexample}

\subsubsection{Point Query with B-Tree}

B-Tree and its variants have been widely used as indexes in
databases. For example, the PostgreSQL uses B-Tree as its index. B-Trees
can be considered as a natural generalisation of binary search tree. In
binary search tree, there is only one key and two possible children in
the internal node. However, an internal node of B-Tree can contain
several keys and children. The keys in a node serve as dividing points
and separate the range of keys. With this structure, we make a
multi-way decision based on comparisons with the keys stored at the node
$x$. The image below illustrates a simple B-Tree.

In this section, we introduce the construction and query processes
of B-Trees and then analyse their properties.

\subsubsection{Motivation}

In computers, the memories are organised in an hierarchical way. For example, a classical computer system consists three layers of memory: the CPU cache, main memory and the hard disk. In such a system, the CPU cache is the fastest but the most expensive while and hard disk is the cheapest but also the slowest. When querying for an item, the CPU will first try to fetch it from the CPU cache. If not there the CPU will then try to fetch it from the main memory, and then the hard disk.

%TODO: Some statements are needed to show the data is stored in blocks in the memory.

At the same time, the traditional hard disk drive (HDD) is made by a moving mechanical structure.

%TODO: Illustrate the mechanical structure.
% Should we be writing this since we are using in-memory ?

In summary, there are two properties in classical computer systems that we need to take into account:

\begin{enumerate}
	\item The memory is not flat, meaning that memory references are not equally expensive. 
	\item 
\end{enumerate}

\subsubsection{Definition and Terms}
\begin{figure}[htp]
    \centering
    \includegraphics[width=0.6\textwidth]{graphs/B-Tree_definitions.png}
    \caption{B-Tree}
    \label{fig:B-Tree}
\end{figure}

Before we formally define B-Trees, we assume the following terms:

\begin{itemize}
\item
  \textbf{Keys}: The key in a database is a special attribute that could identify a row in the database. In our work, each key corresponds to a 1-dimensional
  \textbf{value} and forms a key-value pair.
\item
  \textbf{Order}: The Order of a B-Tree is the maximum number of children that a node can have. Number of keys in a node is always one less than the order of the tree at the maximum.
\item
  \textbf{Internal Node}: An internal node is any node of the tree that has child nodes and is not a root node.
\item
  \textbf{Leaf Node}: A leaf node is any node that does not have child nodes.
\end{itemize}

Each node in a B-Tree has the following attributes:

\begin{itemize}
\item
  $x.n$ is the number of keys currently stored in the node $x$.
\item
  Inside each node, the keys are sorted in non decreasing order, so that
  we have $x.keys_1\leq x.keys_2\leq\cdots\leq x.keys_{x.n}$.
\item
  $x.leaf$, a Boolean value determines if current node is a leaf node.
\end{itemize}

With these properties, A B-Tree $T$ whose root is $T.root$ have the
following properties:

\begin{itemize}
\item
  Each internal node $x$ contains $x.n+1$ children. We assume the
  children are $x.c_1,\cdots,x.c_{x.n+1}$.
\item
  The nodes in the tree have lower and upper bounds on the number of
  keys that can contain. These bounds can be expressed in terms of a
  fixed integer $t$.
\end{itemize}

\subsubsection{Insertion of B-Tree}

When inserting keys into a binary search tree, we search for the leaf
position at which to insert the new key. However, with B-Tree, we cannot
simply find the position, create a new node and insert the value because
the tree will be imbalanced again. Hence, in this section, we illustrate
an operation that splits a full node around its median key

\begin{algorithm}[H]
    \SetAlgoLined
    \SetKwInOut{Input}{Input}
     \Input{\texttt{$m$:order\_of\_tree ,$(k,v)$:(key, value), $N$:Node;}}
    \SetKwInOut{Output}{Output}
     \Output{\texttt{B-Tree}}
     \eIf{$N$ is a leaf and not yet full}  
     {
        \texttt{insert $(k,v)$ into $N$}
     }
     {
        \texttt{create new Node N'}\\
        \texttt{Find the median of the node}\\
        \texttt{Add the value at the median location to the new node N'}\\
     }
    \If{$N$ is root with no children and not full}
    {
    
        \texttt{insert $(k,v)$ into $N$}
        
    }
     \caption{Algorithm for B-Tree insertion}
     \label{B-Tree Insertion}
\end{algorithm}


\begin{figure}[htp]
    \centering
    \includegraphics[width=0.6\textwidth]{graphs/B-Tree_example01.png}
    \caption{B-Tree key insertion}
    \label{fig:B-Tree key insertion1}
\end{figure}

\begin{figure}[htp]
    \centering
    \includegraphics[width=0.3\textwidth]{graphs/B-Tree_example02.png}
    \caption{B-Tree key insertion}
    \label{fig:B-Tree key insertion2}
\end{figure}

\subsubsection{Insertion in a B-Tree}

There are two conditions in insertion:
\begin{itemize}
\item
\textbf{When the node is empty or not created at all}:
In the algorithm, initially when the first key is inserted and there is no root to the tree it will check the condition if there are any nodes and if not create a new one. It will insert the new key in the node and keeps inserting the new keys until it is one less than the order of the B-Tree being created. 
\item
\textbf{When the node is full}:
As soon as the maximum number of allowed children has reached for the root node a new empty node is created. Suppose we have inserted [1,11,21] to a node of a B-Tree with degree 4. Now if we want to insert a new value 31 into the tree, since it has reached the maximum number of children it will find the median of the existing node [1,11,21] which is 11 and increase the level of the B-tree to 2 and make [1] and [21] child nodes of [11]. So now if a new value is to be inserted it can be inserted. Splitting of the node happens each time it reaches it's maximum allowed keys. Now 31 will be compared with 11 and since 31 > 11 it will be inserted in the right child and it will have an updated value of [21,31]. More keys can then be inserted until it reaches it's maximum and splits again. Once the node is split it's parent is also updated to the median value i.e., [11] in this case for nodes [1] and [21,31,41] as can be seen in \ref{fig:B-Tree key insertion2}.
\end{itemize}

\subsubsection{Search in a B-Tree}

\begin{algorithm}[H]
    \SetAlgoLined
    \SetKwInOut{Input}{Input}
     \Input{\texttt{k ; key, root ; Root of the B-Tree}}
    \SetKwInOut{Output}{Output}
     \Output{\texttt{Value associated with key}}
     \For{$i\gets0$ \KwTo $len(Root Node)$}{
         \If{keys in root greater than k}
         {
            \texttt{SEARCH\_CHILD(key)} //Search linearly the child associated with the key location one before\\
            \eIf{Child is a leaf}
            {
                \texttt{Linearly search until the key is reached}
            }
            {
                \texttt{SEARCH\_CHILD(key)}
            }
            
         }
    }
     \caption{Algorithm for B-Tree Search}
     \label{B-Tree Search}
\end{algorithm}

Search in a B-Tree is basically the comparison of the value of the key that needs to be searched with the keys in the node. It first linearly checks the value in the root key and looks for a key value which is greater than the searched key. As soon as it finds a key greater than the searched key it will search in the child of the key before it. For example if we were to search of key [41] in the example of the B-Tree used above, we would first check the value [11] and since it smaller than [31] and there are no keys greater than this in the root node it will search in the right child of the root node. It will then linearly search the node and locate it's associated value and return it. 

\subsubsection{Point Query with $K$D-Tree Model Index}

Similar to search with binary search tree, we also need to traverse the tree in order to perform point query. However, we need to switch the dimensions when we compare the values between the query key and the values in the nodes.

\begin{algorithm}[H]
    \SetAlgoLined
    \SetKwInOut{Input}{Input}
    \SetKwInOut{Output}{Output}
    \Input{\texttt{t}: The node being searched; \texttt{x}: The query key; \texttt{cd}: Current dimension}
    \Output{\texttt{n}: the node that contains the query key}
    \texttt{DIM=2}\\
    \uIf{\texttt{t==NULL}} {
    	\Return \texttt{NULL}
    }
    \uIf{\texttt{x[0]==t.data[0]} and \texttt{x[1]==t.data[1]}} {
    	\Return \texttt{t}
    }
    \uElseIf{\texttt{x[cd]<t.data}} {
    	\Return \texttt{pointSearch(t.left, x, (cd+1) \% DIM)}
    }
    \uElseIf{\texttt{x[cd]>t.data}} {
    	\Return \texttt{pointSearch(t.right, x, (cd+1) \% DIM)}
    }
    \caption{Point Query with $K$D-Tree}
    \label{algo: point_query_kdtree}
\end{algorithm}

The point query works in the following steps:

\begin{enumerate}
	\item From Line $2$ to $3$, we first check if current node is \texttt{NULL}. If so, that means that we have already traversed all the possible nodes and found nothing. In this case, the query returns \texttt{NULL}.
	\item From Line $4$ to $5$, we check if the current node contains the same key as the query key. If so, the current node is the node that we are looking for. Hence, we return the current node in this case.
	\item Otherwise, from Line $6$ to $9$, we check if the current dimension of the query key is smaller, larger or equal to the current dimension of the data in the node.
	\begin{enumerate}
	\item If it is smaller, then we search on the left subtree of current node, with the same query key and switched dimension.
	\item If it is larger, then we search on the right subtree of current node, with the same query key and switched dimension.
	\end{enumerate}
\end{enumerate}

\begin{mscexample}    
	In the previous figure \ref{fig:kd_tree_example}, we showed an example $K$D-tree. If we want to search for $(50, 30)$ in this tree, we would follow the following steps:
	\begin{enumerate}
		\item We first check the root node and compares the $x$-coordinate. As $50>30$, we go to the right subtree of the root node.
		\item Then in the subtree, we compare the $y$-coordinate. As $50<70$, we go to the left subtree of this node.
		\item Then in the left subtree, the termination condition is reached, hence we return this node as result.
	\end{enumerate}
\end{mscexample}


\subsubsection{Point Query with Baseline Index Model}

\subsubsection{Overview}

The B-Tree can be regarded as a function $\mathcal{F}$ that maps the key $x$ into its corresponding page index $y$. It is known to us that the pages are allocated in a way that the every $S$ entries are allocated in a page where $S$ is a pre-defined parameter. For example, if we set $S$ to be 10 items per page, then the first page will contain the first 10 keys and their corresponding values. Similarly, the second 10 keys and their corresponding values will be allocated to the second page.

If we know the CDF of $X$ as $F(X\leq x)$ and the total number of entries $N$, then the position of $x$ can be estimated as $p=F(x)*N$ and the page index where it should be allocated to is given by

$$y=\floor{\frac{p}{S}}=\floor{\frac{F(x)*N}{S}}$$  

For example, if the keys are uniformly distributed from $0$ to $1000$, i.e. the CDF of $X$ is defined as $F(X\leq x)=\frac{x}{1000}$ and we set $S=10, N=1001$. Then for any key $x$, we immediately know it will be allocated into $y=\floor{\frac{1000}{10}*\frac{x}{1000}}=\floor{\frac{x}{10}}$. Assume that we have a key $698$, then we can calculate $y=\floor{\frac{698}{10}}=69$. By doing so, the page index is calculated in constant time and space.

In this example, we see that the distribution of $X$ is essential and our goal of learned index in one-dimensional data is to learn such distribution. To do so, we apply two different techniques as the baseline, the polynomial regression and fully connected neural network.

To train such a learned index, we first manually generate the $X$ with respect to a certain distribution. We then save the generated $X$ into a dense array with the length $N$. Then we use the proportional index, i.e. the index of each $x$ divided by $N$ as the expected output $y$.

\subsubsection{Polynomial Regression}
 
The polynomial regression model with degree $m$ can be formalised as 

$$ \hat{y_i}= \beta_0+\beta_1x_i+\beta_2x_i^2+\cdots+\beta_mx_i^m$$ and it can be expressed in a matrix form as below

$$
\begin{bmatrix}
y_1 \\ y_2\\ \vdots \\ y_n 
\end{bmatrix}=\begin{bmatrix}
1 & x_1 & x_1^2 &\cdots & x_1^m \\ 
1 & x_2 & x_2^2 &\cdots & x_2^m \\ 
\vdots \\ 
1 & x_n & x_n^2 &\cdots & x_n^m \\ 
\end{bmatrix}\begin{bmatrix}
\beta_0 \\ \beta_1 \\ \vdots \\ \beta_m 
\end{bmatrix}
$$ which can be written as $Y=\boldsymbol{X}\boldsymbol{\beta}$. 
 
 Our goal is to find $\beta$ such that the sum of squared error, i.e. $\text{S}(\boldsymbol{\beta})=\sum_{i=1}^n(\hat{y}-y)^2$ is minimal. This optimisation problem can be resolved by ordinary least square estimation as shown below.
 
 First we have the error as
 
 \begin{equation}
 \begin{split}
 \text{S}(\boldsymbol{\beta})=||\boldsymbol{y}-\boldsymbol{X} \boldsymbol{\beta}||& =(\boldsymbol{y}-\boldsymbol{X}\boldsymbol{\beta})^T(\boldsymbol{y}-\boldsymbol{X}\boldsymbol{\beta})\\
 	& =\boldsymbol{y}^T\boldsymbol{y}-\boldsymbol{\beta}^T\boldsymbol{X}^T\boldsymbol{y}-\boldsymbol{y}^T\boldsymbol{X}\boldsymbol{\beta}+\boldsymbol{\beta}^T\boldsymbol{X}^T\boldsymbol{X}\boldsymbol{\beta}
\end{split}
 \end{equation}
 
 Here we know that $(\boldsymbol{\beta}^T\boldsymbol{X}^T\boldsymbol{y})^T=\boldsymbol{y}^T\boldsymbol{X}\boldsymbol{\beta}$ is a $1\times 1$ matrix, i.e. a scalar. Hence it is equal to its own transpose. As a result we could simplify the error as
 
 \begin{equation}
 	\begin{split}
 		\text{S}(\boldsymbol{\beta})=\boldsymbol{y}^T\boldsymbol{y}-2\boldsymbol{\beta}^T\boldsymbol{X}^T\boldsymbol{y}+\boldsymbol{\beta}^T\boldsymbol{X}^T\boldsymbol{X}\boldsymbol{\beta}
 	\end{split}
 \end{equation}
 
 In order to find the minimum of $S(\boldsymbol{\beta})$, we differentiate it with respect to $\boldsymbol{\beta}$ as 
 
 \begin{equation}
 	\nabla_{\boldsymbol{\beta}}S=-2\boldsymbol{X}^T\boldsymbol{y}+2(\boldsymbol{X}^T\boldsymbol{X})\boldsymbol{\beta}
 \end{equation}
 
 By let it to be zero, we end up with 
 
 \begin{equation}
 \begin{split}
 	 &	-\boldsymbol{X}^T\boldsymbol{y}+(\boldsymbol{X}^T\boldsymbol{X})\boldsymbol{\beta}=0 \\
 	& \implies \boldsymbol{\beta}= (\boldsymbol{X}^T\boldsymbol{X})^{-1}\boldsymbol{X}^T\boldsymbol{y}
 \end{split}
 \end{equation}
 
\subsubsection{Fully Connected Neural Network}

\subsubsection{Point Query with Recursive Model Index}

%TODO: there should be some graph to demonstrate the last-mile problem.

In our baseline models, it is not very difficult to reduce the mean square error from millions to thousands. However, it is much harder to reduce it from thousands to tens. This is the so called last-mile problem.

In order to solve this problem, recursive model index was proposed \cite{kraska2018case}. The idea is to split the whole set of data into smaller pieces and assign each piece an index model. By doing so, each model is only responsible for a small range of keys. Ideally, in each smaller range, the keys are distributed in a way that is easier to be learned by our index models, such as polynomial model, fully connected model or even traditional B-Tree model.

As shown in Fig. \ref{rmi_structure}. A recursive model can be regarded as a tree structure, which contains a root model that receives the full dataset for training. Then the root model will split the dataset into several parts. Each sub-model will then receive one part of the full dataset. Then we train the sub-models one by one with the partial training dataset. 

\begin{mscexample}
	For example, in the Fig. \ref{rmi_structure}, the full dataset will be split into three parts and each sub-model receives one part. To train this recursive model, we first train the root model with the whole dataset. Then the root model will split the dataset into 3 parts according to the predicted value of each data point in the dataset. Then each sub-model will receive one part and we train the sub-model accordingly.
\end{mscexample}

\begin{figure*}[h]
\centering
\includegraphics[scale=0.4]{graphs/implementation/one-dim/rmi_demo.pdf}
\caption{An example recursive model index with one root model and three leaf model.}
\label{rmi_structure}
\end{figure*}

\subsubsection{Properties}

Similar to a tree, we define the following terms in a recursive model:

\begin{enumerate}
	\item \textbf{Node Model}. Every node is responsible for making decisions with given input data. In one dimensional case, it can be regarded as a function $f:\mathbb{R}\to\mathbb{R}, x\to y$ where $x$ is the input index and $y$ is the corresponding page block. In principle, each node can be implemented as any machine learning model, from linear regression to neural network, or a traditional tree-based model, such as B-Tree.
	\item \textbf{Internal Node Model}. Internal nodes are all nodes except for leaf nodes and the root node. Every internal node receives a certain part of training data from the full dataset, and train a model on it. 
\end{enumerate}

In the following sections, we will use the notations defined below:
\begin{enumerate}
	\item $N_M^{(i)}$ is the number of models in the $i$th stage.
	%TODO: more notations
	%TODO: modify algorithms accordingly
\end{enumerate}


\subsubsection{Training}

In order to construct a recursive model, we need to have several parameters listed below:
\begin{enumerate}
	\item The training dataset, notated as $(X, Y)$ with entries notated as $(x,y)$.
	\item The number of stages, notated as $N_S$. It is an integer variable.
	\item The number of models at each stage, notated as $N_M$. It is a list of integer variable. $N_M^{(i+1)}$ represents the number of models in the $i$th stage.
\end{enumerate}

The training process of recursive model is an up-bottom process. There will be only one root model that receives the whole training data. After the root model is trained, we iterate over all the training data and predict the page by the root model. After the iteration, we get a new set of pairs $(X, Y_0)$. Then we map $\forall y_0\in Y_0$ into the selected model id in next stage by $\texttt{next}=y_0 * N_M^{(i+1)}/\texttt{max(Y)}$.

% REVISE THIS
\begin{algorithm}[H]
    \SetAlgoLined
    \SetKwInOut{Input}{input}
    \Input{\texttt{$N_S$: A scalar representing the number of stages; \\ $N_M$: An array representing the number of models at each stage; \\ x; y}}
     \texttt{trainset=[[(x,y)]]} \\
     \texttt{stage$\gets 0$} \\
     \While{\texttt{stage} \textless \texttt{$N_S$}}{
      \While{\texttt{model} \textless \texttt{$N_M$[stage]}} {
        \texttt{model.train(trainset[stage][model])} \\
        \texttt{models[stage].append(model)}
      }
      \uIf{\texttt{stage<$N_S$-1}} {
        \For{\texttt{$i\gets0$ \KwTo $len(x)$}}{
        	\texttt{next\_model = 0}\\
        	\For{\texttt{$j\gets 0$ \KwTo stage-1}} {
        		\texttt{output = models[stage][next\_model]} \\
        		\texttt{next\_model = output * $N_M$[stage+1]/max\_y}\\
        	}
            	\texttt{model = models[next\_model]} \\
            	\texttt{output = model.predict(x[i])} \\
            	\texttt{next = output * $N_M$[stage+1]/max\_y} \\
            	\texttt{trainset[stage+1][next].add((x[i],y[i]))}
        }
      }
     \texttt{stage=stage+1}
     }
     \caption{Training of Recursive Model Index}
\end{algorithm}

\subsubsection{Polynomial Internal Models}

In the recursive model index, we use internal models to learn the CDF of a part of the full training data. In order to learn the CDF, we need to know or assume the distribution of a specific part of the data. In this report, we support the following distributions.

\begin{table}[h]
  \begin{tabularx}{\textwidth}{@{}XX@{}}
  \toprule
    Linear Regression & $wx+b$ \\
    Quadratic Regression & $ax^2+bx+c$ \\
    B-Tree & N/A \\
    Fully Connected Neural Network & N/A \\
  \bottomrule
  \end{tabularx}
  \end{table}

Here we describe how we fit a polynomial model.

The polynomial regression model with degree $m$ can be formalised as 

$$ \hat{y_i}= \beta_0+\beta_1x_i+\beta_2x_i^2+\cdots+\beta_mx_i^m$$ and it can be expressed in a matrix form as below

$$
\begin{bmatrix}
y_1 \\ y_2\\ \vdots \\ y_n 
\end{bmatrix}=\begin{bmatrix}
1 & x_1 & x_1^2 &\cdots & x_1^m \\ 
1 & x_2 & x_2^2 &\cdots & x_2^m \\ 
\vdots \\ 
1 & x_n & x_n^2 &\cdots & x_n^m \\ 
\end{bmatrix}\begin{bmatrix}
\beta_0 \\ \beta_1 \\ \vdots \\ \beta_m 
\end{bmatrix}
$$ which can be written as $Y=\boldsymbol{X}\boldsymbol{\beta}$. 
 
 \begin{mscproof}
 	 Our goal is to find $\beta$ such that the sum of squared error, i.e. $$\text{S}(\boldsymbol{\beta})=\sum_{i=1}^n(\hat{y}-y)^2$$ is minimal. This optimisation problem can be resolved by ordinary least square estimation as shown below.
 
 First we have the error as
 
 \begin{equation}
 \begin{split}
 \text{S}(\boldsymbol{\beta})=||\boldsymbol{y}-\boldsymbol{X} \boldsymbol{\beta}||& =(\boldsymbol{y}-\boldsymbol{X}\boldsymbol{\beta})^T(\boldsymbol{y}-\boldsymbol{X}\boldsymbol{\beta})\\
 	& =\boldsymbol{y}^T\boldsymbol{y}-\boldsymbol{\beta}^T\boldsymbol{X}^T\boldsymbol{y}-\boldsymbol{y}^T\boldsymbol{X}\boldsymbol{\beta}+\boldsymbol{\beta}^T\boldsymbol{X}^T\boldsymbol{X}\boldsymbol{\beta}
\end{split}
 \end{equation}
 
 Here we know that $(\boldsymbol{\beta}^T\boldsymbol{X}^T\boldsymbol{y})^T=\boldsymbol{y}^T\boldsymbol{X}\boldsymbol{\beta}$ is a $1\times 1$ matrix, i.e. a scalar. Hence it is equal to its own transpose. As a result we could simplify the error as
 
 \begin{equation}
 	\begin{split}
 		\text{S}(\boldsymbol{\beta})=\boldsymbol{y}^T\boldsymbol{y}-2\boldsymbol{\beta}^T\boldsymbol{X}^T\boldsymbol{y}+\boldsymbol{\beta}^T\boldsymbol{X}^T\boldsymbol{X}\boldsymbol{\beta}
 	\end{split}
 \end{equation}
 
 In order to find the minimum of $S(\boldsymbol{\beta})$, we differentiate it with respect to $\boldsymbol{\beta}$ as 
 
 \begin{equation}
 	\nabla_{\boldsymbol{\beta}}S=-2\boldsymbol{X}^T\boldsymbol{y}+2(\boldsymbol{X}^T\boldsymbol{X})\boldsymbol{\beta}
 \end{equation}
 
 By let it to be zero, we end up with 
 
 \begin{equation}
 \begin{split}
 	 &	-\boldsymbol{X}^T\boldsymbol{y}+(\boldsymbol{X}^T\boldsymbol{X})\boldsymbol{\beta}=0 \\
 	& \implies \boldsymbol{\beta}= (\boldsymbol{X}^T\boldsymbol{X})^{-1}\boldsymbol{X}^T\boldsymbol{y}
 \end{split}
 \end{equation}
 
 \end{mscproof}

\subsubsection{Point Query with Lisa}

It is difficult to apply traditional $K$NN query pruning strategies applicable for $K$D-Trees, to LISA model as it doesn't maintain a tree like structure with all nodes and
entries based on MBRs (minimum bounding rectangle) and parent-children relationships. %Shard boundaries are learned per mapped interval and no data structure is maintained to refer to shards in adjacent mapped intervals. 
The key idea in LISA paper $K$NN query implementation is to convert it into a range query by estimating an appropriate query range. LISA paper suggests a learning model to learn an appropriate distance bound from underlying training data for every query point and specific value of K. However, we have used empirically estimates to learn this distance bound for different values of $K$. This distance bound is used to convert the $K$NN query to range query.The query range is augmented if less than K neighbors are found in a range query. 

Consider a query point $q_{knn}=(x_{0},x_{1})$, let $x^{'} \in V$ be the $K$th nearest key to $x$ in database at a distance value $\delta = \| x^{'}-q_{knn}\|_{2} $. Lets define $ \mathcal{Q}(q_{knn},\delta) \triangleq [x_{0}-\delta, x_{0}+\delta) \times[x_{1}-\delta, x_{1}+\delta)$ and $\mathcal{B}(q_{knn}, \delta)  \triangleq \{p \in V \mid \| q_{knn}-p\|_{2} \leq \delta \} $. We can create a query rectangle $qr =  \mathcal{Q}(q_{knn}, \delta + \epsilon)$ where $\epsilon \rightarrow 0$. As shown in Fig. \ref{fig:KNN_Query_Lisa}, K nearest keys to $q_{knn}$ are all in $\mathcal{B}(q_{knn}, \delta)$ and thus in $\mathcal{Q}$. $K$NN query can be solved using the range query if we can estimate an appropriate distance bound $\delta$ for every query point.

\begin{figure*}[t]
    \centering
    \includegraphics[width=0.7\textwidth]{graphs/KNN_Query_Lisa.png}
    \caption{KNN Query Implementation in Lisa(K=3)\\
    1)$q_{knn}$ represents the query point, $ \mathcal{Q}(x,\delta) \triangleq [x_{0}-\delta, x_{0}+\delta) \times[x_{1}-\delta, x_{1}+\delta)$, represents query rectangle and $ \mathcal{B}(x, \delta)$ represents the key space at distance $\delta$ containing K nearest keys.\\
    2)KNN query can be solved by range query if we can estimate an appropriate distance bound $\delta$ for every query point\\
    }
    \label{fig:KNN_Query_Lisa}
\end{figure*}
In our experiments, we find the $\delta$ empirically. We try with different values of $\delta$ and choose the one for which we get the best results. 


\subsection{Range Query}

A range query is a database operation that retrieves all the records that lies in a range. In this project, we perform range query on $2$-dimensional data only. In addition, we only consider a range query where the range is defined as a rectangle. Under these assumptions, a range query can be formalised as a query $\mathcal{Q}(\boldsymbol{l}, \boldsymbol{u})$ where $l,u\in\mathbb{R}^2$ and $split\_axis$ as $\mathcal{S}$.

\begin{mscexample}
	For example, assume we have the points
	$$
	[(1,2), (3,4), (3.5, 4), (5,6)]
	$$
	and the range query $\mathcal{Q}((2,3), (5,5))$, as shown below:
	
	\begin{minipage}[t]{\linewidth}
	\centering
   	\includegraphics[width=10cm]{graphs/implementation/queries/range_query.pdf}
   	\label{fig:range_query_demo}
   	\captionof{figure}{A Range Query Example where $\mathcal{Q}(\boldsymbol{l}, \boldsymbol{u})=\mathcal{Q}((2,3),(5,5))$}
	\end{minipage}
	
In this example, the range query should return the points that lies inside the red rectangle, i.e. $((3,4), (3.5, 4))$.

\end{mscexample}

\subsubsection{Range Query with $K$D-Tree}

\begin{figure}[htp]
    \centering
    \includegraphics[width=1.0\textwidth]{graphs/KD_Tree_Range_Query_Algorithm.png}
    \caption{$K$D-Tree for Range Query Algorithm (Orange color are the visited points and blue color are the points not visited) 
    1) Tree representation of points. \\
    2) Subdivision of space into cells.(Orange color are the visited cells and blue color are the cells not visited) 
    }
    \label{fig:KD-Tree_for_Range_Query_Algorithm}
\end{figure}


% TODO: THIS ALGORITHM IS STILL NOT COMPLETE, and hard to follow.

% DONE: It is finished now.

\begin{algorithm}[H]
    \SetAlgoLined
    \SetKwInOut{Input}{Input}
    \SetKwInOut{Output}{Output}
    \Input{\texttt{Rectangle range (Lower;l and upper;u) $\mathcal{Q}(\boldsymbol{l}, \boldsymbol{u}); [l \in \mathbb{R};u \in \mathbb{R}]$}}
    \Output{\texttt{List of points within $\mathcal{Q}(\boldsymbol{l}, \boldsymbol{u})$}}
    % \texttt{SEARCH\_RANGE\_QUERY($\mathcal{Q}(\boldsymbol{l}, \boldsymbol{u})$)}\\
    \eIf{\texttt{node.leftChild is leaf}}
        {
            \If {$cell(v) \subseteq \mathcal{Q}(\boldsymbol{l}, \boldsymbol{u})$}
                {\texttt{add all points within cell to list}}
            \eIf {$cell(v) \cap \mathcal{Q}(\boldsymbol{l}, \boldsymbol{u}) = \varnothing$}
                {\texttt{add nothing to list}}
            {\texttt{add all points reported within $cell(v) \cap \mathcal{Q}(\boldsymbol{l}, \boldsymbol{u})$}}
        }
        {
            % \texttt{SEARCH\_RANGE\_QUERY($\mathcal{Q}(\boldsymbol{l}, \boldsymbol{u})$)} //Recursively function is called\\
            \texttt{Search subtree of $v$ recursively.}
        }
        
    \texttt{Similarly, steps are repeated for node.rightChild}
    
    \caption{Range Query Algorithm for $K$D-Tree}
    \label{Range_Query_Algorithm_$K$D-Tree}
\end{algorithm}

In algorithm \ref{Range_Query_Algorithm_$K$D-Tree} 

\begin{enumerate}
    \item On line $1$, start with the root of tree in the function SEARCH\_RANGE\_QUERY(). We pass $\mathcal{Q}(\boldsymbol{l})$ and $\mathcal{Q}(\boldsymbol{u})$ and check if the point at the root falls within $\mathcal{Q}(\boldsymbol{l}, \boldsymbol{u})$. If the point falls within $\mathcal{Q}(\boldsymbol{l}, \boldsymbol{u})$ then the point is added to the result list.
    
    \item Let $v$, $w$ be left and right children nodes.(Refer fig \ref{fig:KD-Tree_for_Range_Query_Algorithm} for space partition of space)
    
    \item On line $2$, after checking the root, $v$ of root is searched to see if it is a leaf node. Since $v$ is not a leaf node, a decision to traverse the right or left subtree is made by checking its x-axis, y-axis and $\mathcal{S}$. 
    
    \item On line $3$, check if the entire cell of $v$ lie within the range of $\mathcal{Q}(\boldsymbol{l}, \boldsymbol{u})$. If it does then add all the points to the result.
    
    \item On line $6$, check if there is no intersection of the cell of $v$ with the $\mathcal{Q}(\boldsymbol{l}, \boldsymbol{u})$. If there is no intersection then nothing is added to the result list.
    
    \item On line $8$, since there must be a partial intersection of the cell of $v$, add the points from the cell that lie within $\mathcal{Q}(\boldsymbol{l}, \boldsymbol{u})$. 
    
    \item On line $11$, keep traversing within the subtree as in above steps until we get all the points within $\mathcal{Q}(\boldsymbol{l}, \boldsymbol{u})$
    % \item In fig \ref{fig:KD-Tree_for_Range_Query_Algorithm} we can see that all cells that intersect $\mathcal{Q}(\boldsymbol{l}, \boldsymbol{u})$ are checked and all the cells that are not intersected by $\mathcal{Q}(\boldsymbol{l}, \boldsymbol{u})$ are pruned. This improves the performance of the range query and hence results in much faster query.\\
\end{enumerate}

There are two cases in range query search:
\begin{enumerate}
    \item\textbf{Case 1}: When an entire subtree lie within $\mathcal{Q}(\boldsymbol{l}, \boldsymbol{u})$.
    \item\textbf{Case 2}: When only a part of the subtree lie within $\mathcal{Q}(\boldsymbol{l}, \boldsymbol{u})$.
\end{enumerate}

\begin{mscexample}

    
    \begin{minipage}[t]{\linewidth}
    \centering
    \includegraphics[width=6cm]{graphs/Range_Query_Tree.png}
    \label{fig:KD-Tree_for_Range Query}
    \hfill
    \includegraphics[width=6cm]{graphs/Range_Query_plot.png}
    \label{fig:KD_Tree_Range_Query_Plot}
    % \caption{Case 1: $K$D-Tree for Range Query (Case 1). Points highlighted in orange are returned in the query.}
    \end{minipage}


    
\begin{enumerate}
    \item\textbf{Case 1} : For example we have a tree with Point list as 
	$$[(5,6),(4,2),(7,9),(3,1),(5,5),(10,7),(2,11)]$$
	
	 with $\mathcal{Q}(\boldsymbol{l}) = (2,3)$ and $\mathcal{Q}(\boldsymbol{u}) = (6,7)$. \\
	 We follow below steps in order to get the results:
% 	 We can see the points along with $\mathcal{Q}(\boldsymbol{l}, \boldsymbol{u})$ plotted in \ref{fig:KD_Tree_Range_Query_Plot}.
	 \begin{enumerate}
	 
        \item Root point is checked and since its x-axis and y-axis both lie within $\mathcal{Q}(\boldsymbol{l}, \boldsymbol{u})$ i.e., $2 > 5 > 6$ and $3 > 6 > 7$, it is added to result list.
        
         \item In order to check whether to traverse left or right, we check if the x-axis($\mathcal{Q} = 0$) of root is greater than or equal to the $\mathcal{Q}(\boldsymbol{l}$) x-axis. Since this value is larger ($5 > 2$), it will then traverse to the left. 
         
         \item Increase $\mathcal{S}$ to $1$.
         
         \item We have root.leftChild to be $(4,2)$. Check if both the x-axis and y-axis lie within  $\mathcal{Q}(\boldsymbol{l}, \boldsymbol{u})$. Since the y-axis doesn't lie in $\mathcal{Q}(\boldsymbol{l}, \boldsymbol{u})$, this point is not added ($2 \notin [3,7]$). 
         
         \item Similarly, it keeps adding points to list while it recursively traverses the tree and checks if the points lie within $\mathcal{Q}(\boldsymbol{l}, \boldsymbol{u})$ until it reaches a leaf.
         
	\end{enumerate}
	\begin{minipage}[t]{\linewidth}
        \centering
        \includegraphics[width=6cm]{graphs/Range_Query_Tree_02.png}
        % \caption{Case 2: $K$D-Tree for Range Query. Points highlighted in orange are returned in the query.}
        \label{fig:KD-Tree_for_Range_Query_Case2}
        \hfill
        \includegraphics[width=6cm]{graphs/Range_Query_plot_02.png}
        % \caption{Case 2: $K$D-Tree Range Query Plot on 2-dimensional plane. Points $(4,2)$, $(3,1)$ and $(5,5)$ lie within $\mathcal{Q}(\boldsymbol{l}, \boldsymbol{u})$}
        \label{fig:KD_Tree_Range_Query_Plot_Case2}
        \end{minipage}
    
	 \item\textbf{Case 2} : For example, we have a tree with Point list as: 

	$$[(5,6),(4,2),(7,9),(3,1),(5,5),(10,7),(2,11)]$$
	
	with $\mathcal{Q}(\boldsymbol{l}) = (2,0.5)$ and $\mathcal{Q}(\boldsymbol{u}) = (6,4.75)$. \\
	We follow below steps in order to get the results:
% 	We can see the points along with $\mathcal{Q}(\boldsymbol{l}, \boldsymbol{u})$ plotted in \ref{fig:KD_Tree_Range_Query_Plot_Case2}.
	\begin{enumerate}
    	\item Root point is checked and since both x-axis and y-axis do not lie within $\mathcal{Q}(\boldsymbol{l}, \boldsymbol{u})$ i.e., $6 \notin [0.5, 4.75]$ it is not added to the list.
    	
    	\item In order to check whether to traverse left or right of the tree, we check if the x-axis of root is greater than or equal to $\mathcal{Q}(\boldsymbol{l})$ x-axis. Since the value is larger ($5 > 2$), it will then traverse to the left. Root.leftChild node is $(4,2)$. Since both, x-axis and y-axis lie within $\mathcal{Q}(\boldsymbol{l}, \boldsymbol{u})$, it is added to the result list. 
    	
    	\item Since all the points in this cell lie within $\mathcal{Q}(\boldsymbol{l}, \boldsymbol{u})$, they are all added to the list.
    	
    \end{enumerate}	
\end{enumerate}
\end{mscexample}

\begin{figure*}[t]
    \centering
    \includegraphics[width=1\textwidth]{graphs/range_query_lisa.png}
    \caption{Range Query Search in Lisa.
    1) Find the cells that overlap with $\mathcal{Q}(\boldsymbol{l}, \boldsymbol{u})$. \\
    2) Decompose $\mathcal{Q}(\boldsymbol{l}, \boldsymbol{u})$ into the unions of smaller query rectangles, each of which intersect only one cell. \\
    3) Find shards corresponding to lower and upper coordinates for each $\mathcal{Q}(\boldsymbol{l}, \boldsymbol{u})$, and perform a sequential search. }
    \label{fig:Range_Query_Lisa}
\end{figure*}


\subsubsection{Range Query with LISA}


For a range query $\mathcal{Q}(\boldsymbol{l},\boldsymbol{u})$, we first find the cells that overlap with $\mathcal{Q}$. Then we decompose $\mathcal{Q}$ into the union of smaller query rectangles $\bigcup \mathcal{Q}_i$ such that each smaller query rectangles intersects only one cell, as shown in the Fig. \ref{fig:Range_Query_Lisa}.



 Suppose that $\mathcal{Q}=\bigcup \mathcal{Q}_i$ where $\mathcal{Q}_i=[l_{i_0}, u_{i_o})\times [l_{i_1}, u_{i_1})$, i.e. we have $\mathcal{Q}_i$ representing the $i$th smaller query rectangles of one cell $C_j$.
 
 Then we can calculate the mapped values of $\mathcal{Q}_i$, i.e. $\mathcal{M}(l_{i_0}, l_{i_1})$ and $\mathcal{M}(u_{i_0}, u_{i_1})$. For simplicity, we use $m_l^{(i)}$ and $m_u^{(i)}$ to denote $\mathcal{M}(l_{i_0}, l_{i_1})$ and $\mathcal{M}(u_{i_0}, u_{i_1})$ respectively.
 
After creating corresponding mapped values, we then apply the shard prediction function $\mathcal{SP}(m_{l}^{i})$ and $\mathcal{SP}(m_{u}^{i})$ to predict the shard that could possibly contain keys that lie in the query rectangle $\mathcal{Q}_i$. Then in each shard, we perform a sequential search to find the desired keys. 



\subsection{$K$NN Query}

$K$-Nearest Neighbours ($K$NN), as the name suggests, is the process of finding $K$ nearest neighbours to a given query point. In this project, $K$NN query is only performed on $2$-dimensional data. We use $\ell_2$ norm as the distance metric. A $K$NN query will be formalised as $\mathcal{K}(\boldsymbol{X})$ where $\boldsymbol{X}\in\mathbb{R}^2$.

\subsubsection{$K$NN query with $K$D-Tree}
As a baseline, we first perform $K$NN query with $K$D-tree.\\
The main advantage of $K$D-Tree is that we can exploit the tree structure and prune points we don't think will have distance smaller than the ones we have already calculated. This improves the time complexity as compared to finding the distance of point with every other point in space to get the closest neighbours. 


\begin{algorithm}[H]
    \SetAlgoLined
    \SetKwInOut{Input}{Input}
    \SetKwInOut{Output}{Output}
    \Input{$K$; \texttt{Number of nearest neighbour, List of TestPoints; $\mathcal{P}(\boldsymbol{x}, \boldsymbol{y}); [x \in \mathbb{R};y \in \mathbb{R}]$}}
    \Output{\texttt{List of $K$ nearest points(ResultList)}}
    \For{$i\gets0$ \KwTo $len(TestPoints)$}
    {
        \texttt{Start at root}\\
        \texttt{Traverse subtree where $\mathcal{P}(\boldsymbol{x}, \boldsymbol{y})_i$ can be added.}\\
        \texttt{Find the leaf; Calculate the distance and store it as $\mathcal{D}$}\\
        \eIf{$len(ResultList) < K$}
            {
                \eIf {\texttt{Perpendicular distance of Parent with $\mathcal{P}(\boldsymbol{x}, \boldsymbol{y}) <= \mathcal{D}$}}
                    {
                        \texttt{Go on the other side of subtree} 
                    }
                        {
                            \texttt{Go up another $level$}
                        }
            }
            {
                \texttt{return ResultList}
            }
        
    }
    \caption{$K$NN Query Algorithm for $K$D-Tree}
    \label{$K$NN_Query_Algorithm_$K$D-Tree}
\end{algorithm}

In algorithm \ref{$K$NN_Query_Algorithm_$K$D-Tree},
\begin{enumerate}
    \item Start with the root to traverse tree until we reach the leaf. We find the subtree where the new $\mathcal{P}(\boldsymbol{x}, \boldsymbol{y})$ could be added and finally reach the leaf of this subtree. 

    \item Calculate the square of the euclidean distance of this point from the $\mathcal{P}(\boldsymbol{x}, \boldsymbol{y})$. We add it to the list and push and pop values from the list depending on the distances we calculate while traversing the tree upwards from here.
    
    \item From the leaf we could either go up another level or go the other side of the subtree to get a point that could have a distance smaller than the last best calculated distance in the list. 
    
    \item Once we make a decision in the above step, we can recursively traverse the tree upwards until we reach the root. 
\end{enumerate}


\begin{mscexample}


    \begin{minipage}[t]{\linewidth}
        \centering
        \includegraphics[width=6cm]{graphs/KD-Tree_KNN_Tree.png}
        % \caption{$K$D-Tree for KNN Query}
        \label{fig:$K$D-Tree_for_KNN Query}
        \hfill
        \includegraphics[width=6cm]{graphs/KD-Tree_KNN_plot.png}
        % \caption{$K$D-Tree KNN Plot on 2-dimentional plane}
        \label{fig:KD_Tree_KNN_Plot}
    \end{minipage}
	For example, we have Point list as $$((5,4),(2,6),(13,3),(8,7),(3,1),(10,2))]$$ 
	
	\textbf{Test point; $\mathcal{P}(\boldsymbol{x}, \boldsymbol{y})$} = $(9,4)$\\
	
% 	we will have a tree structure as shown in \ref{fig:$K$D-Tree_for_KNN Query} and it's plot on $2$-dimensional plane is shown in \ref{fig:KD_Tree_KNN_Plot}. *****\ref{fig:KD_Tree_KNN_Plot} \\
% 	In the fig above, we can see that although point $(8,7)$ is the leaf we will reach when we traverse the tree to search where test point $(9,4)$ can be added, it is not in fact the nearest point to the test point. \\
	
	Below are the steps followed to get the $4$ nearest neighbours:
	\begin{enumerate}
    	\item Traverse to $(8,7)$ by searching for a location where $\mathcal{P}(\boldsymbol{x}, \boldsymbol{y})$ could be added.
    	
    	\item Add $(8,7)$ to result list.
    	
    	\item Calculate the distance of $(8,7)$ and $\mathcal{P}(\boldsymbol{x}, \boldsymbol{y})$. Save the distance as $\mathcal{D}$
    	
    	\item Make a decision whether to traverse to the other side of the subtree to point $(10,2)$ by checking the perpendicular distance of $(13,3)$ with $\mathcal{P}(\boldsymbol{x}, \boldsymbol{y})$ and compare this with $\mathcal{D}$.(We do this to verify if there is even a possibility to find a point smaller than the last best distance on the other side of the subtree.) 
    	
    	\item Since the perpendicular distance is smaller than the best calculated $\mathcal{D}$ ($A > B$), we will check the distance of $\mathcal{P}(\boldsymbol{x}, \boldsymbol{y})$ and $(10,2)$. This distance in our case is indeed smaller than the best calculated distance of $\mathcal{P}(\boldsymbol{x}, \boldsymbol{y})$ with $(8,7)$ so far.
    	
    	\item Add $(10,2)$ to result list.
    	
    	\item Similarly, we traverse until we have $4$ nearest neighbour to $\mathcal{P}(\boldsymbol{x}, \boldsymbol{y})$ in the list.
	\end{enumerate}
\end{mscexample}



\subsubsection{$K$NN Query with LISA}
%TODO: What does this paragraph mean?
%We do not know the analytical representation of shards, as we use machine learning model $ \mathcal{SP}$ to generate shards. Thus, 
It is difficult to apply traditional $K$NN query pruning strategies applicable for $K$D-Trees, to LISA model as it doesn't maintain a tree like structure with all nodes and
entries based on MBRs (minimum bounding rectangle) and parent-children relationships. Shard boundaries are learned per mapped interval and no data structure is maintained to refer to shards in adjacent mapped intervals. The key idea in the $K$NN query is to convert it into a range query by estimating an appropriate query range. LISA paper suggests a learning model to learn an appropriate distance bound from underlying training data for every query point and specific value of K. However, we used empirically estimates to learn this distance bound for different values of $K$. This distance bound is used to convert the $K$NN query to range query.The query range is augmented if less than K neighbors are found in a range query. 

Consider a query point $q_{knn}=(x_{0},x_{1})$, let $x^{'} \in V$ be the $K$th nearest key to $x$ in database at a distance value $\delta = \| x^{'}-q_{knn}\|_{2} $. Lets define $ \mathcal{Q}(q_{knn},\delta) \triangleq [x_{0}-\delta, x_{0}+\delta) \times[x_{1}-\delta, x_{1}+\delta)$ and $\mathcal{B}(q_{knn}, \delta)  \triangleq \{p \in V \mid \| q_{knn}-p\|_{2} \leq \delta \} $. We can create a query rectangle $qr =  \mathcal{Q}(q_{knn}, \delta + \epsilon)$ where $\epsilon \rightarrow 0$. As shown in Fig. \ref{fig:KNN_Query_Lisa}, K nearest keys to $q_{knn}$ are all in $\mathcal{B}(q_{knn}, \delta)$ and thus in $\mathcal{Q}$. $K$NN query can be solved using the range query if we can estimate an appropriate distance bound $\delta$ for every query point.

\begin{figure*}[t]
    \centering
    \includegraphics[width=0.7\textwidth]{graphs/KNN_Query_Lisa.png}
    \caption{KNN Query Implementation in Lisa(K=3)\\
    1)$q_{knn}$ represents the query point, $ \mathcal{Q}(x,\delta) \triangleq [x_{0}-\delta, x_{0}+\delta) \times[x_{1}-\delta, x_{1}+\delta)$, represents query rectangle and $ \mathcal{B}(x, \delta)$ represents the key space at distance $\delta$ containing K nearest keys.\\
    2)KNN query can be solved by range query if we can estimate an appropriate distance bound $\delta$ for every query point\\
    }
    \label{fig:KNN_Query_Lisa}
\end{figure*}
In our experiments, we find the $\delta$ empirically. We try with different values of $\delta$ and choose the one for which we get the best results. 

\chapter{Evaluation}

\begin{mscsummary}
	In this chapter, we describe how we evaluate the database indexes that we have implemented in previous chapter. For both one and two dimensional data, we use manually synthesised dataset that are generated from a certain distribution as our dataset. This chapter is organised into two sections, where the first section describes the experiment settings and results for one-dimensional data and indexes and the second section describes the two-dimensional data.
\end{mscsummary}

\section{One Dimensional Data and Indexes}

For one dimensional data, the evaluation covers the following tasks:

\begin{itemize}
	\item Compares the performance between baseline model, recursive model and traditional B-Tree.
\end{itemize}

\section{Two Dimensional Data and Indexes}

For two dimensional data, the evaluation covers the following tasks:

\begin{itemize}
 	\item Find hyper-parameters for the LISA Baseline model empirically.
	\item Find the hyper-parameters for the LISA model empirically.
	\item Compares the performance between $K$D-tree, LISA Baseline and LISA models for point query.
	\item Compare the performance between $K$D-tree, LISA Baseline and LISA models for range query.
	\item Compare the performance between $K$D-tree and LISA models for KNN query. KNN Query has not been implemented for LISA Baseline as there is no description of KNN Query for Baseline model in the paper. 
\end{itemize}

\subsection{Dataset}

For two dimensional case, we manually generate three columns of the data:

\begin{itemize}
	\item The first two columns contain the  2 dimensional keys $\boldsymbol{X} \in \mathbb{R}^{2}$, which are independently sampled from a given distribution. %% TOOD: What distributions?
	\item Then we assign the keys into different pages according to a preset parameter $N_{page}$ for page size. Specifically, the first $N_{page}$ keys will be assigned into the first page, the second $N_{page}$ keys will be assigned into the second page and so on so forth. After the assignments, we set the second column $Y$ to be the page index of the corresponding $x$.
\end{itemize}

\subsection{Task 1 : Hyper-parameters Search }
After generating dataset as mentioned in previous section, we sample a smaller subset from it. We repeat our experiments for 3 different sample sizes of 10000, 100000 and 1000000 points. Test data is a copy of training data for all our experiments. For Baseline and Lisa models, final prediction is given by linear search through a range of values (identified as a Cell for Baseline and Shard for LISA model) and mean square error (MSE) is zero as test points are already learned during training. This is where Learned Index models differ from traditional machine learning models where model performance is evaluated on unseen data. 

\subsubsection {Hyper-parameter search for the LISA Baseline implementation}
Baseline model has one hyper-parameter: N (Number of cells specifying the number of equal length intervals into which mapped values are divided). As discussed in section, the query search consists of two parts, first is binary search to locate the cell into which the query key is located, followed by sequentially comparison of the query key value with keys in the found cell until a match is found. The time complexity of first search is $log_{2}N_{1}$, where $N_{1}$ is the number of cells. The time complexity of second search is  $ \left \lceil {N_{2} / 2}\right \rceil $, where $N_{2}$ is the number of keys per cell. Optimum value of hyper-parameter N is to assign 1 key per page to mimimize the sequential search cost. 

\begin{table}
	\centering
	\begin{tabular}{||p{0.15\textwidth}<{\centering}|p{0.2\textwidth}<{\centering}| p{0.1\textwidth}<{\centering}|p{0.15\textwidth}<{\centering}|p{0.15\textwidth}<{\centering}|p{0.15\textwidth}<{\centering}||}
		\hline
		Training/Test Data Size& Model & $N$ & Build Time (ms) & Avg Query Time (ms) & Memory Size (KB)\\ [0.5ex] 
		\hline
		\hline
		%10,000& Lisa Baseline & 10 & 11.171 & 4.3426 & 313.77\\
		%\hline
		10,000& LISA Baseline & 100 & 11.25 & 0.7189 & 315\\
		%\hline
		%10,000& LISA Baseline & 1000 & 13.542 & 0.32806 & 336\\
		\hline
		10,000& LISA Baseline & 10000 &26.83 & 0.1985 & 547\\
		\hline
		100,000& LISA Baseline & 1000 & 111.97 & 0.7271 & 3149\\
		\hline
		100,000& LISA Baseline & 100000 & 272.93 & 0.2381 & 5469\\
		\hline
		1,000,000& LISA Baseline & 1000 & 1104.65 & 4.4732 & 31274\\
		\hline
		1,000,000& LISA Baseline & 1000000 & 2717.65 & 0.2436 & 54688\\
		\hline
		\hline
	\end{tabular}
    \caption{Hyper-parameters Search LISA Baseline Model\\
    a) For any training size, optimum value of $N$ will be equal to the number of keys in training data.\\
    b) If $N$ is assigned as numbers of keys in database, number of keys per cell will be 1, and query search cost will be reduced to $log_{2}N$.}
    \label{small_lognormal_lisa_baseline_10000}
\end{table}
\comm{
\begin{table}[ht]
	\centering
	\begin{tabular}{||p{0.15\textwidth}<{\centering}|p{0.2\textwidth}<{\centering}| p{0.1\textwidth}<{\centering}|p{0.15\textwidth}<{\centering}|p{0.15\textwidth}<{\centering}|p{0.15\textwidth}<{\centering}||}
		\hline
		Training/Test Data Size & Model & No. of cells & Build Time (ms) & Avg Query Time (ms) & Memory Size (KB)\\ [0.5ex] 
		\hline
		\hline
		%100,000& Lisa Baseline & 10 & 109.287 & 46.7233 & 3126.2\\
		%\hline
		%100,000& Lisa Baseline & 100 & 111.596 & 4.8086 & 3128.3\\
		
		100,000& LISA Baseline & 1000 & 111.978 & 0.7271 & 3149\\
		\hline
		100,000& LISA Baseline & 10000 & 128.496 & 0.3301 & 3360\\
		\hline
		100,000& LISA Baseline & 100000 & 272.933 & 0.2381 & 5469\\
		\hline
		\hline
	\end{tabular}
   \caption{Hyper-parameters Search LISA Baseline Model: Training Size:100,000 Points}
    \label{small_lognormal_lisa_baseline_100000}
\end{table}
}

\textbf{Conclusion} As shown in table \ref{small_lognormal_lisa_baseline_10000}, following conclusions can be drawn:
\begin{enumerate}
	\item Build time : Build time increases with increase in value of N, as metadata for additional cells needs to be calculated. 
	\item Average Query Time :  Average Query Time decreases with increase in value of N as number of keys per cell decreases.
	\item Memory Size :  Memory requirements of the model increases with increase in value of N, as metadata for additional cells needs to be stored. Increase  in memory size is not significant with increase in N as we maintain only two values per cell, mapped value of first key in the cell and mapped value of last key in the cell.
\end{enumerate}
\comm{
\begin{table}[ht]
	\centering
	\begin{tabular}{||p{0.15\textwidth}<{\centering}|p{0.2\textwidth}<{\centering}| p{0.1\textwidth}<{\centering}|p{0.15\textwidth}<{\centering}|p{0.15\textwidth}<{\centering}|p{0.15\textwidth}<{\centering}||}
		\hline
		Training/Test Data Size& Model & No. of cells & Build Time(ms) & Avg Query Time(ms) & Memory Size(KB)\\ [0.5ex] 
		\hline
		\hline
		%1,000,000& Lisa Baseline & 10 & 1094.99 & 347.561 & 31251.3\\
		%\hline
		%1,000,000& Lisa Baseline & 100 &1099.68 &40.145 & 31253.4\\
		%\hline
		1,000,000& LISA Baseline & 1000 & 1104.65 & 4.473 & 31274.5\\
		\hline
		1,000,000& LISA Baseline & 10000 & 1143.73 & 0.669 & 31485.4\\
		\hline
		1,000,000& LISA Baseline & 100000 & 1273.56 & 0.294 & 33594.8\\
		\hline
		1,000,000& LISA Baseline & 1000000 & 2717.65 & 0.243 & 54688.5\\
		\hline
		\hline
	\end{tabular}
    
	\caption{Hyper-parameters Search LISA Baseline Model: Training Size:1,000,000 Points}
	\label{small_lognormal_lisa_baseline_1000000}
\end{table}
}


\subsubsection {Hyper-parameter search for the LISA implementation}
For LISA model, we have 3 hyper parameters:
\begin{enumerate}
	\item $G$ : The size of the grid cell. Number of grid cells into which the key space is divided. In our implementation, we use a square grid cell, and total number of cells is given by $G$ $\times G$.
	\item $N$ : Number of equal length intervals into which mapped value range is divided. During our experiments, we found that shard prediction algorithm gives best performance if mapped interval boundaries are aligned to grid cell boundaries. That's why this parameter is initialized to $N$=$G$ $\times G$   
	\item $S$ : Number of shards to learn per mapped interval. 
\end{enumerate}

\textbf{Conclusion} Experiments results shown in tables \ref{small_lognormal_lisa_10000}, \ref{small_lognormal_lisa_100000} and \ref{small_lognormal_lisa_1000000} are consistent across all 3 training sizes and have following interpretation. 
\begin{enumerate}

    \item For a particular value of $G$, average query time decreases and memory size increases with increase in value of S.
    
    \item Average query time decreases and memory size increases with increase in values of $G$ and $S$. 
	
	\item We need to choose S such that there are at least 35 keys per shard. We see mean square errors(mse) if number of keys per shard are less than 35 for following reasons. 
	\begin{enumerate}
	    %\item Total number of keys in training set are first divided equally among grid cells.For each cell, its corresponding keys are further divided equally among number of shards.
	    \item For point query search, we first predict a shard and then sequentially compare the query point key values with all the keys in the predicted shard until a match is found
		\item For query points near the shard boundaries, there can be a mismatch in groundtruth shardId and predicted shardId.If the query point is not found in the predicted shard, we continue our search in adjacent left and right shards in an empirically found range.
	\end{enumerate}
	
	During test experiments we found that if shard size is less than 35 keys, then sometimes shard prediction error can be greater than 1 and point query search can fail resulting in mse errors.  
\end{enumerate}

\begin{table}
	\centering
	\begin{tabular}{||p{0.14\textwidth}<{\centering}|p{0.08\textwidth}<{\centering}|p{0.15\textwidth}<{\centering}| p{0.07\textwidth}<{\centering}|p{0.1\textwidth}<{\centering}|p{0.125\textwidth}<{\centering}|p{0.125\textwidth}<{\centering}|p{0.07\textwidth}<{\centering}||}
		\hline
		Training/Test Data Size& Model & G & S& Build Time(s) & Avg Query Time(ms) & Memory Size(KB) &mse\\ [0.5ex] 
		\hline
		\hline
		10,000& LISA& 4*4=16 & 5& 4.335& 1.13135 & 324.72&0\\
		\hline
		10,000& LISA& 4*4=16 & 10& 3.370& 0.96036 & 329.07&0\\
		\hline
		10,000& LISA& 4*4=16 & 20&1.127& 0.86184 & 337.85&0\\
		\hline
		10,000& LISA& 4*4=16 & 30&3.478& 0.74339 & 346.63&5729\\
		\hline
	    \hline
	\end{tabular}

    \caption{Hyper-parameters Search LISA Model: Training Size:10,000 Points. \\
    a) For the last row, Numbers of keys= 10000 \\
    b) Keys per cell= $10000 \setminus (4\times4) = 625$\\
    c) Keys per shard = $625\setminus30=20$ keys per shard, resulting in mse errors}
	\label{small_lognormal_lisa_10000}
\end{table}

\begin{table}
	\centering
	\begin{tabular}{||p{0.14\textwidth}<{\centering}|p{0.08\textwidth}<{\centering}|p{0.15\textwidth}<{\centering}| p{0.07\textwidth}<{\centering}|p{0.1\textwidth}<{\centering}|p{0.125\textwidth}<{\centering}|p{0.125\textwidth}<{\centering}|p{0.08\textwidth}<{\centering}||}
		\hline
		Training/Test Data Size& Model & GridCellSize & No of Shards & Build Time(s) & Avg Query Time(ms) & Memory Size(KB)&mse\\ [0.5ex] 
		\hline
		\hline
	    %100,000& Lisa& 4*4=16 & 5& 45.846& 5.93345 & 3137.2&0\\
		%\hline
		%100,000& Lisa& 4*4=16 & 10& 42.398& 3.29308 & 3141.6&0\\
		%\hline
		%100,000& LISA& 4*4=16 & 20& 59.036&1.52851 & 3150.3&0\\
		\hline
		100,000& LISA& 4*4=16 & 50& 122.64& 1.51173 & 3176.6&0\\
		\hline
		100,000& Lisa& 4*4=16 & 100& 30.211& 1.44084 & 3220.3&0\\
		\hline
		100,000& LISA& 4*4=16 & 150& 142.13&1.15491 & 3264.1&297234\\
		%\hline
		%100,000& Lisa& 6*6=36 & 20&33.637&1.59742 & 3178.9&0\\
		\hline
		100,000& Lisa& 6*6=36 & 50& 66.375& 1.55903 & 3238.1&0\\
		\hline
		100,000& LISA& 6*6=36 & 75& 72.491& 1.43043 & 3287.2&0\\
		\hline
		100,000& Lisa& 6*6=36 & 100& 60.929& 1.64881 & 3336.4&5.6e+07\\
		\hline
		100,000& LISA& 8*8=64 & 20& 35.638& 1.54029 & 3218.7&0\\
		\hline
		100,000& LISA& 8*8=64 & 50& 45.014& 1.52117 & 3323.6&0\\
		\hline
		\hline
	\end{tabular}
    \caption{Hyper-parameters Search LISA Model: Training Size:100,000 Points}
	\label{small_lognormal_lisa_100000}
\end{table}

\comm{
\begin{table}
	\centering
\centering
	\begin{tabular}{||p{0.14\textwidth}<{\centering}|p{0.08\textwidth}<{\centering}|p{0.15\textwidth}<{\centering}| p{0.07\textwidth}<{\centering}|p{0.1\textwidth}<{\centering}|p{0.125\textwidth}<{\centering}|p{0.125\textwidth}<{\centering}|p{0.08\textwidth}<{\centering}||}
		\hline
		Training/Test Data Size& Model & GridCellSize & No of Shards& Build Time(s) & Avg Query Time(ms) & Memory Size(KB)&mse\\ [0.5ex] 
		\hline
		\hline
	 	%1,000,000&Lisa& 10*10=100 & 5& 122.64& 1.51173 & 3176.6&0\\
		%\hline
		%1,000,000& Lisa& 10*10=100 & 10& 30.211& 1.44518 & 3220.3&0\\
		%\hline
		%1,000,000& Lisa& 10*10=100 & 20& 24.428&3.19491 & 3149.4&0\\
		%\hline
		1,000,000& LISA& 10*10=100 & 50&743.29&1.77751 & 31558.9&0\\
		\hline
		1,000,000& Lisa& 10*10=100 & 100& 1077.89& 1.63397 & 31832.3&0\\
		\hline
		1,000,000& LISA& 20*20=400 & 25& 365.49& 2.53317 & 31930.8&0\\
		\hline
		1,000,000& LISA& 20*20=400 & 50& 609.32& 1.44526 & 32477.6&0\\
		\hline
		1,000,000& LISA& 25*25=625 & 25& 240.22& 1.56227& 32779.8&0\\
		\hline
		1,000,000& LISA& 30*30=900 & 25& 205.18& 1.79839 & 33010.3&0\\
		\hline
		\hline
	\end{tabular}
    \caption{Hyper-parameters Search LISA Model: Training Size:1,000,000 Points}
	\label{small_lognormal_lisa_1000000}
\end{table}
}
\subsection{Performance Comparison for $K$D-Tree, LISA-Baseline and LISA Models }
During following experiments, for each training data size, we have used hyper-parameters optimized for that particular data set size. 

\subsubsection{Point Query Comparison for $K$D-Tree, LISA-Baseline and LISA Models }
Table \ref{Point_Query_Comparision} shows the performance evaluation for $K$D-Tree, LISA-Baseline and LISA Models for different training data sizes. For a given training set, we perform point query evaluation for every point in the dataset and take the average. As shown in the Fig. \ref{fig:Point_Query_Comparision}, LISA outperforms $K$D-tree in terms of average query time and memory requirements, however its build time is significantly higher than $K$D-Tree 

\begin{table}
	\centering
\centering
	\begin{tabular}{||p{0.18\textwidth}<{\centering}|p{0.15\textwidth}<{\centering}|p{0.15\textwidth}<{\centering}|p{0.15\textwidth}<{\centering}|p{0.15\textwidth}<{\centering}|p{0.05\textwidth}<{\centering}||}
		\hline
		Training/Test Data Size& Model & Build Time(s) & Avg Query Time(ms) & Memory Size(KB)&mse\\ [0.5ex] 
		\hline
		\hline
	 	10,000& KD-Tree & 0.023 & 4.363 & 2890 & 0\\
	 	\hline
	 	10,000& Baseline & 0.026 & 0.198 & 547&0\\
	 	\hline
	 	10,000& LISA & 1.127&0.861 & 337&0\\
		\hline
	 	100,000& KD-Tree & 0.340 & 6.176 & 28906 &0\\
	 	\hline
	 	100,000& Baseline & 0.324 & 0.241 & 5469&0\\
	 	\hline
	 	100,000& LISA& 22.491& 1.43 & 3169&0\\
		\hline
	 	1,000,000& KD-Tree& 4.124 & 9.254 & 289062 &0\\
	 	\hline
	 	1,000,000& Baseline& 2.718 & 0.343 & 54688&0\\
	 	\hline
	 	1,000,000& LISA& 445.324&1.445 & 32477&0\\
	 	
	 	%\hline
	 	%10,000,000& KD-Tree& 77.924 & 10.254 & 2421880 &0\\
	 	%\hline
	 	%10,000,000& Baseline& 13.118 & 3.873 & 312736&0\\
	 	%\hline
	 	%10,000,000& LISA& 445.324&1.445 & 32477&0\\
		\hline
		\hline
	\end{tabular}
	\caption{Point Query experimental results for KDTree, Baseline and LISA models}
	\label{Point_Query_Comparision}
\end{table}

\begin{figure*}[t]
    \centering
    \includegraphics[width=1.1\textwidth]{graphs/evaluation/PointQueryPlot.pdf}
    \caption{Point Query experimental results for $K$D-Tree, Baseline and LISA models\\
    LISA outperforms $K$D-Tree in terms of average query time and memory requirements, however its build time is significantly higher than $K$D-Tree }
    \label{fig:Point_Query_Comparision}
\end{figure*}
\subsubsection {Range Query Experiments}
Table \ref{Range_Query_Experimental_Results} shows evaluation results for LISA,Baseline and $K$D-tree models for range sizes of 10, 100, 1000 for different training sizes. For a given range query size, we perform 20 trials and take the average. For each trial, we sample a random point from the test set and find the range from sampled point to the range query size. Average query time for each range is further divided by the range size to compare the query time across various ranges. As shown in the Fig. \ref{fig:Range_Query_Comparision}, LISA outperforms $K$D-tree for range query size of 10000 for all training sizes, however its range query time for smaller range sizes is significantly higher than $K$D-Tree.

\begin{figure*}
    \centering
    \includegraphics[width=1.1\textwidth]{graphs/evaluation/RangeQueryPlot.pdf}
    \caption{Range Query experimental results for $K$D-Tree, Baseline and LISA models\\
    a) Plot A shows average range query time for a fixed training size of 1M points. LISA outperforms $K$D-Tree for larger range queries. \\
     b) Plot B shows average range query time for a fixed range query of size 10000 for various training sizes. Lisa outperforms $K$D-Tree for all training data sizes for range queries of size 10000 }
    \label{fig:Range_Query_Comparision}
\end{figure*}

\begin{table}
	\centering
\centering
	\begin{tabular}{||p{0.15\textwidth}<{\centering}|p{0.15\textwidth}<{\centering}|p{0.22\textwidth}<{\centering}|p{0.25\textwidth}<{\centering}|p{0.20\textwidth}<{\centering}||}
		\hline
		Training/Test Data Size& Range Query Size & Avg Query Time(ms)($K$D-tree) & Avg Query Time(ms)(Baseline)&Avg Query Time(ms)(LISA)\\ [0.5ex] 
		\hline
		\hline
	 	10,000& 10& 0.1361 & 0.1113& 0.8204 \\
	 	\hline
	 	10,000& 100& 0.0533 & 0.0451& 0.1201 \\
	 	\hline
	 	10,000& 1000& 0.0438 & 0.0399& 0.0294 \\
 	 	\hline
 	 	10,000& 10000& 0.0648&0.0382&0.0061 \\
	 	\hline
	 	100,000& 10& 0.1392 & 0.1298&2.8961 \\
	 	\hline
	 	100,000& 100& 0.0539 & 0.0505&0.2792 \\
	 	\hline
	 	100,000& 1000& 0.043 & 0.0428&  0.0563 \\
 	 	\hline
 	 	100,000& 10000&0.0718& 0.0392& 0.0129 \\
	 	\hline
	    1,000,000& 10& 0.2238 & 0.2661& 3.5181 \\
	 	\hline
	 	1,000,000& 100& 0.0922 & 0.0617&0.6263 \\
	 	\hline
	 	1,000,000& 1000& 0.0744 & 0.0437 &0.0939 \\
	 	\hline
	 	1,000,000& 10000& 0.0735 & 0.0412 &0.0186 \\
	 	%\hline
	 	%10,000,000& 10& 0.2285 & 0.037188 &12.891 \\
	 	%\hline
	 	%10,000,000& 100& 0.0779 & 0.037188 &0.1611 \\
	 	%\hline
	 	%10,000,000& 1000& 0.0740 &  0.037188 &0.0353 \\
	 	
	 	%\hline
	 	%10,000,000& 10000& 0.0773 &  0.037188 &0.0 \\
	 
		\hline
		\hline
	\end{tabular}
	\caption{Range Query experimental results for $K$D-tree, Baseline and LISA models}
	\label{Range_Query_Experimental_Results}

\end{table}

\begin{figure*}
    \centering
    \includegraphics[width=1\textwidth]{graphs/evaluation/one-d/KNN Query.pdf}
    \caption{KNN Query experimental results for $K$D-Tree and LISA models\\
    a) Plot A shows average KNN query time for a fixed training size of $1M$ points for different values of K. LISA outperforms $K$D-Tree for all values of K. \\
     b) Plot B shows average KNN query time for various training sizes with K = 10. Lisa outperforms $K$D-Tree for all training data sizes. }
    \label{fig:KNN_Query_Comparision}
\end{figure*}

\subsubsection {$\boldsymbol{K}$NN Query Experiments}
Table \ref{KNN_Query_Experimental_Results} shows evaluation results for LISA and $K$D-tree models for $\boldsymbol{K}$NN Queries for various value of K and training sizes. For a given $\boldsymbol{K}$ value, we perform 20 trials and take the average of query time. For each trial, we sample a random point from the test set and find K neighbours around that point. Average query Time is further divided by $\boldsymbol{K}$ to compare the Query time across various values of $\boldsymbol{K}$.As shown in the Fig. \ref{fig:KNN_Query_Comparision}, LISA outperforms $K$D-tree for different training sizes and K values.

\begin{table}
	\centering
\centering
	\begin{tabular}{||p{0.15\textwidth}<{\centering}|p{0.25\textwidth}<{\centering}|p{0.25\textwidth}<{\centering}|p{0.25\textwidth}<{\centering}||}
		\hline
		Training/Test Data Size& K & Avg Query Time(ms)($K$D-tree) & Avg Query Time(ms)(LISA)\\ [0.5ex] 
		\hline
		\hline
	 	10,000& 3& 1.4069 &0.2020 \\
	 	\hline
	 	10,000& 5& 0.8753 &0.1181\\
	 	\hline
	 	10,000& 7& 0.6333 &0.0811 \\
	 	\hline
	 	10,000 & 10& 0.4368 &0.0549 \\
	 	\hline
	 	100,000 & 3& 2.0325 &0.5867 \\
	 	\hline
	 	100,000 & 5& 1.2004 &0.3549 \\
	 	\hline
	 	100,000 & 7& 0.8812 &0.2779 \\
	 	\hline
	 	100,000 & 10&  0.6618 &0.2161 \\
	 	\hline
	    1,000,000 & 3& 2.7865 & 1.218 \\
	 	\hline
	 	1,000,000 & 5& 1.7072 &0.9414 \\
	 	\hline
	 	1,000,000 & 7& 1.3255 &0.7681 \\
	 	\hline
	 	1,000,000 & 10& 0.9172 &0.5779 \\
	 	%\hline
	    %1,000,000 & 3& 3.888 & 3.361 \\
	 	%\hline
	 	%1,000,000 & 5& 2.3613 &3.192 \\
	 	%\hline
	 	%1,000,000 & 7& 1.6812 & 1.748 \\
	 	%\hline
	 	%1,000,000 & 10& 1.2151 & 1.378 \\
		\hline
		\hline
	\end{tabular}
	\caption{KNN Query experimental results for $K$D-tree and LISA model}
	\label{KNN_Query_Experimental_Results}

\end{table}


\chapter{Insights and Findings}

\section{General Discussions}


\subsubsection{Limitations}

Though the learned index model, especially the recursive model has a potential to greatly reduce the memory usage and cost less time in making the query. It is still limited in several perspective.

\begin{itemize}
    \item \textbf{Read-only Database}. Current recursive model index assumes that the data is a static, read-only array. Only when this assumption is hold, we can regard the database index as the CDF. However, in reality, we usually need to insert and delete the data in the array and violates this assumption.
    \item \textbf{Sorted Keys}. The recursive model and baseline model assume that the keys are sorted in ascending order, so that the CDF assumption applies.
    \item \textbf{In-Memory Database}. In our implementations, we only consider the case where all the keys are stored in the memory. 
\end{itemize}

To apply the learned indexes into a general-purpose database, we will need to overcome these limitations. For example, the model needs to be trained again in order to support the read-and-write database. 

\section{One Dimensional Learned Index}

\subsection{Baseline Learned Index}

\subsubsection{Activation Functions}

From our observations, activation functions determines the shape of the fully connected neural network. With the one-dimensional data, the input and output of a neural network is always a scalar, which reveals interesting relations between the activation function and the output of neural network. We use two different activations functions to describe this relation.

\begin{figure}[!htb]
\begin{subfigure}[b]{0.5\textwidth}
	\centering
	\includegraphics[width=8cm]{graphs/insights/identity}
	\caption{Identity Activation}
	\label{fig:id_act}
\end{subfigure}
\hfill
\begin{subfigure}[b]{0.5\textwidth}
	\centering
	\includegraphics[width=8cm]{graphs/insights/relu}	
	\caption{ReLU Activation}
	\label{fig:relu_act}
\end{subfigure}
\caption{The predictions of neural networks with different activation functions. The blue line represents the ground truth and the orange line represents the predicted output.}
\label{fig:relation_of_activation_function}
\end{figure}

\begin{itemize}
\item
  If we use identity activation function, i.e.$z^{(i)}(x)=x$, then no
  matter how many layers are there, the fully connected neural network
  falls back to a linear regression.

  \textbf{Proof:} The output of the first layer, with identity activation function, will be $o^{(1)}=z^{(1)}(w^{(1)}x+b^{(1)})=w^{(1)}x+b^{(1)}$. Then the output will be the input of the next layer, and hence the output of the second layer will be $o^{(2)}=z^{(2)}(w^{(2)}(w^{(1)}x+b^{(1)})+b^{(2)})=w^{(2)}w^{(1)}x+w^{(2)}b^{(1)}+b^{(2)}$. Hence if we use identity activation, the trained neural network will become a linear regression. The predicted output of a neural network with identity activation is illustrated in Fig \ref{fig:id_act}, where we could verify that the predicted output is a line.
\item
  With ReLU (Rectified Linear Unit) as activation function, i.e. $z^{(i)}(x)=\text{max}(0,x)$, then the fully connected neural network becomes a piecewise linear function.

\textbf{Proof:} In the first layer, the output of a neural network with ReLU activation will be $o^{(1)}=z^{(1)}(w^{(1)}x+b^{(1)})=\text{max}(w^{(1)}x+b^{(1)},0)$. There will be two cases for this function:

\begin{equation}
	o^{(1)}=\begin{cases}	
		w^{(1)}x+b & w^{(1)}x+b>0 \\
		0 & \text{otherwise}
	\end{cases}
\end{equation}

The output of the first layer will be the input of the second layer, which uses the same ReLU activation function. Hence the output will be $o^{(2)}=z^{(2)}(w^{(2)}o^{(1)}+b^{(2)})=z^{(2)}(w^{(2)}\text{max}(w^{(1)}x+b^{(1)},0)+b^{(2)})$.

\begin{equation}
	o^{(2)}= \begin{cases}
		w^{(2)}b^{(2)} & w^{(1)}x+b^{(1)}<0 \\
		0 & w^{(2)}(\text{max}(w^{(1)}x+b^{(1)},0))+b^{(2)}<0 
	\end{cases}	
\end{equation}
\end{itemize}

\section{Two Dimensional Learned Index}

It is difficult to apply traditional $K$NN query pruning strategies applicable for $K$D-Trees, to LISA model as it doesn't maintain a tree like structure with all nodes and
entries based on MBRs (minimum bounding rectangle) and parent-children relationships. %Shard boundaries are learned per mapped interval and no data structure is maintained to refer to shards in adjacent mapped intervals. 
The key idea in LISA paper $K$NN query implementation is to convert it into a range query by estimating an appropriate query range. LISA paper suggests a learning model to learn an appropriate distance bound from underlying training data for every query point and specific value of K. However, we have used empirically estimates to learn this distance bound for different values of $K$. This distance bound is used to convert the $K$NN query to range query.The query range is augmented if less than K neighbors are found in a range query. 

Consider a query point $q_{knn}=(x_{0},x_{1})$, let $x^{'} \in V$ be the $K$th nearest key to $x$ in database at a distance value $\delta = \| x^{'}-q_{knn}\|_{2} $. Lets define $ \mathcal{Q}(q_{knn},\delta) \triangleq [x_{0}-\delta, x_{0}+\delta) \times[x_{1}-\delta, x_{1}+\delta)$ and $\mathcal{B}(q_{knn}, \delta)  \triangleq \{p \in V \mid \| q_{knn}-p\|_{2} \leq \delta \} $. We can create a query rectangle $qr =  \mathcal{Q}(q_{knn}, \delta + \epsilon)$ where $\epsilon \rightarrow 0$. As shown in Fig. \ref{fig:KNN_Query_Lisa}, K nearest keys to $q_{knn}$ are all in $\mathcal{B}(q_{knn}, \delta)$ and thus in $\mathcal{Q}$. $K$NN query can be solved using the range query if we can estimate an appropriate distance bound $\delta$ for every query point.

\begin{figure*}[t]
    \centering
    \includegraphics[width=0.7\textwidth]{graphs/KNN_Query_Lisa.png}
    \caption{KNN Query Implementation in Lisa(K=3)\\
    1)$q_{knn}$ represents the query point, $ \mathcal{Q}(x,\delta) \triangleq [x_{0}-\delta, x_{0}+\delta) \times[x_{1}-\delta, x_{1}+\delta)$, represents query rectangle and $ \mathcal{B}(x, \delta)$ represents the key space at distance $\delta$ containing K nearest keys.\\
    2)KNN query can be solved by range query if we can estimate an appropriate distance bound $\delta$ for every query point\\
    }
    \label{fig:KNN_Query_Lisa}
\end{figure*}
In our experiments, we find the $\delta$ empirically. We try with different values of $\delta$ and choose the one for which we get the best results. 

\section{Future Work}
\subsubsection{Future work}

In current work, we have implemented RMI and LISA, two novel learned index structures
for one and two dimensional data respectively. This work opens up several directions for future research on learned indexes for database systems. We are listing some of them here. 

\begin{enumerate}
	\item Read only and in-memory database are two major constraints applicable to our LISA implementation that are supported by the original paper. Adding support for insertion, deletion and disk resident training data will be taken in next phase.
	\item LISA paper supports Lattice Regression model to learn an appropriate distance bound from underlying training data for every query point and specific value of $K$. This distance bound is used to convert the $K$NN query to range query. It will be interesting to add support for the Lattice Regression Implementation in future work. 
	\item It will be interesting to study other query types (e.g., spatial joins and closest pairs) using LISA
	
\end{enumerate}
\chapter{Convolution and CNN for Learned Indexes}


\subsubsection{Limitations}

Though the learned index model, especially the recursive model has a potential to greatly reduce the memory usage and cost less time in making the query. It is still limited in several perspective.

\begin{itemize}
    \item \textbf{Read-only Database}. Current recursive model index assumes that the data is a static, read-only array. Only when this assumption is hold, we can regard the database index as the CDF. However, in reality, we usually need to insert and delete the data in the array and violates this assumption.
    \item \textbf{Sorted Keys}. The recursive model and baseline model assume that the keys are sorted in ascending order, so that the CDF assumption applies.
    \item \textbf{In-Memory Database}. In our implementations, we only consider the case where all the keys are stored in the memory. 
\end{itemize}

To apply the learned indexes into a general-purpose database, we will need to overcome these limitations. For example, the model needs to be trained again in order to support the read-and-write database. 

\chapter{Conclusion}

In this project, we reviewed and implemented two classic tree structures, \textbf{B-Tree} and \textbf{$K$D-Tree}, used as database indexes. The tree structures are capable of finding elements precisely as they will need to traverse all possible nodes. The shortcomings of these tree structures also come from this property: the tree needs to save and traverse the possible nodes, which yields a space complexity that is proportional to the number of records. In the meanwhile, it yields a query time complexity that has a positive correlation with the number of records. As the volume of data is increasing rapidly, the time and space complexity becomes huge and the tree structures become a bottleneck of applications.

We then implement two kinds of learned indexes: the \textbf{recursive model index} for one-dimensional data and the \textbf{LISA} model for two-dimensional data. We conclude that the recursive model and its baseline model have bounded time and space complexities.

% ADD some advantages of LISA. 

Having said these advantages of learned indexes, they all have their shortcomings.

\begin{enumerate}
	\item Even though the learned indexes have a constant time complexity for queries, the constant is relatively large. Therefore, if the number of records is not huge, the learned indexes will not outperform classic tree structures.
	\item The recursive model, baseline model and the convolutional model are prone to error. It may not be a big issue with the in-memory databases, but it will cost much more time when searching between different disk pages is needed, especially with the traditional hard disk drive (HDD).
\end{enumerate}

\section*{Acknowledgement}

We would like to express our sincere gratitude to Prof. Dr. Michael Böhlen, and Mr. Qing Chen for their commitment in supervising this project. Our appreciation extends to Dr. Sven Helmer in reading our report and arranging discussion and presentation of this project.

\begin{appendices}
\chapter{Appendix}
\begingroup
\fontsize{8pt}{10pt}\selectfont
\begin{landscape}
\begin{table}[]
\footnotesize
\begin{tabular}{|c|c|c|c|c|l|l|l|l|}
\hline
\textbf{\# id} & \textbf{Distributions} & \textbf{root model} & \textbf{second model} & \textbf{third models} & \multicolumn{1}{c|}{\textbf{Build Time (s)}} & \multicolumn{1}{c|}{\textbf{Query Time (ms)}} & \multicolumn{1}{c|}{\textbf{Evaluation   Error (MSE)}} & \multicolumn{1}{c|}{\textbf{Memory Size   (KB)}} \\ \hline
\multicolumn{1}{|l|}{1} &  & fcn & 200 fcn & 2000 fcn & 418.9493798 & 0.970932583 & 653.8536667 & 7487.059896 \\ \cline{1-1} \cline{3-9} 
\multicolumn{1}{|l|}{2} &  & fcn & 200 fcn & 4000 fcn & 1141.521194 & 0.9675528 & \textbf{1.134166667} & 24440.75523 \\ \cline{1-1} \cline{3-9} 
\multicolumn{1}{|l|}{3} &  & fcn & 200 fcn & 6000 fcn & 688.8004486 & 1.07512705 & 196.9116667 & 13034.22656 \\ \cline{1-1} \cline{3-9} 
\multicolumn{1}{|l|}{4} &  & fcn & 400 fcn & 2000 fcn & 483.1734781 & 1.158343717 & 113246.196 & 9208.992183 \\ \cline{1-1} \cline{3-9} 
\multicolumn{1}{|l|}{5} &  & fcn & 400 fcn & 4000 fcn & 636.8463397 & 1.339095933 & 113652.3212 & 12695.55731 \\ \cline{1-1} \cline{3-9} 
\multicolumn{1}{|l|}{6} &  & fcn & 400 fcn & 6000 fcn & 742.0712694 & 1.243333667 & 51.00183333 & 15434.78905 \\ \cline{1-1} \cline{3-9} 
\multicolumn{1}{|l|}{7} &  & fcn & 600 fcn & 2000 fcn & 504.959355 & 1.06512235 & 113246.2647 & 9745.335942 \\ \cline{1-1} \cline{3-9} 
\multicolumn{1}{|l|}{8} &  & fcn & 600 fcn & 4000 fcn & 879.6010201 & 0.973031833 & 18.99766667 & 20434.90626 \\ \cline{1-1} \cline{3-9} 
\multicolumn{1}{|l|}{9} &  & fcn & 600 fcn & 6000 fcn & 373.6126809 & 1.11725315 & 142041.6877 & 8118.023442 \\ \cline{1-1} \cline{3-9} 
\multicolumn{1}{|l|}{10} &  & lr & 200 lr & 2000 lr & 262.5089284 & 1.280502367 & 8246.633985 & 4348.463542 \\ \cline{1-1} \cline{3-9} 
\multicolumn{1}{|l|}{11} &  & lr & 200 lr & 4000 lr & 869.7494701 & 1.304096217 & 7326.238372 & 18769.81252 \\ \cline{1-1} \cline{3-9} 
\multicolumn{1}{|l|}{12} &  & lr & 200 lr & 6000 lr & 655.0431077 & 1.318176683 & 6276.09111 & 13297.72135 \\ \cline{1-1} \cline{3-9} 
\multicolumn{1}{|l|}{13} &  & lr & 400 lr & 2000 lr & 275.3925674 & 1.31789575 & 120427.9247 & 5143.059892 \\ \cline{1-1} \cline{3-9} 
\multicolumn{1}{|l|}{14} &  & lr & 400 lr & 4000 lr & 601.7362665 & 1.453903583 & 6783.428749 & 12864.80731 \\ \cline{1-1} \cline{3-9} 
\multicolumn{1}{|l|}{15} &  & lr & 400 lr & 6000 lr & 388.5866734 & 1.623972083 & 5998.720313 & 8041.416654 \\ \cline{1-1} \cline{3-9} 
\multicolumn{1}{|l|}{16} &  & lr & 600 lr & 2000 lr & 267.8966881 & 1.861582733 & 121932.5051 & 4986.927088 \\ \cline{1-1} \cline{3-9} 
\multicolumn{1}{|l|}{17} &  & lr & 600 lr & 4000 lr & 558.531068 & 1.52717965 & 8434.091306 & 13843.60678 \\ \cline{1-1} \cline{3-9} 
\multicolumn{1}{|l|}{18} &  & lr & 600 lr & 6000 lr & 337.0881814 & 1.28034995 & 35342.6365 & 8366.570317 \\ \cline{1-1} \cline{3-9} 
\multicolumn{1}{|l|}{19} &  & lr & 200 lr & 4000 fcn & 34.86059083 & 1.63086885 & 14478.05283 & \textbf{220.4973958} \\ \cline{1-1} \cline{3-9} 
\multicolumn{1}{|l|}{20} &  & lr & 200 fcn & 4000 fcn & 410.2378013 & 1.653916983 & 13656.1507 & 9318.697933 \\ \cline{1-1} \cline{3-9} 
\multicolumn{1}{|l|}{21} &  & fcn & 200 fcn & 4000 lr & 38.131223 & 1.667702583 & 12191.8397 & 602.2005208 \\ \cline{1-1} \cline{3-9} 
\multicolumn{1}{|l|}{22} &  & fcn & 200 lr & 4000 lr & 238.9569197 & 1.663567483 & 13403.64758 & 5229.914058 \\ \cline{1-1} \cline{3-9} 
\multicolumn{1}{|l|}{23} &  & lr & 200 fcn & 4000 lr & 290.6430138 & 1.657428583 & 12567.93278 & 6588.58335 \\ \cline{1-1} \cline{3-9} 
\multicolumn{1}{|l|}{24} & \multirow{-24}{*}{log\_normal} & fcn & 200 lr & 4000 fcn & 352.6849412 & 1.909310833 & 11572.93555 & 8292.059883 \\ \hline
\end{tabular}
\label{appendix_1: experiments summary of small lognormal}
\caption{}
\end{table}
\end{landscape}
\endgroup

\begin{table}
	\centering
	\begin{tabular}{||p{0.15\textwidth}<{\centering}|p{0.2\textwidth}<{\centering}| p{0.1\textwidth}<{\centering}|p{0.15\textwidth}<{\centering}|p{0.15\textwidth}<{\centering}|p{0.15\textwidth}<{\centering}||}
		\hline
		Training/Test Data Size& Model & $N$ & Build Time (ms) & Avg Query Time (ms) & Memory Size (KB)\\ [0.5ex] 
		\hline
		\hline
		10,000& LISA Baseline & 10 & 11.17 & 4.3426 & 313\\
		\hline
		10,000& LISA Baseline & 100 & 11.25 & 0.7189 & 315\\
		\hline
		10,000& LISA Baseline & 1000 & 13.54 & 0.3283 & 336\\
		\hline
		10,000& LISA Baseline & 10000 &26.83 & 0.2415 & 547\\
		\hline
		100,000& LISA Baseline & 10 & 109.28 & 46.7173 & 3126\\
		\hline
		100,000& LISA Baseline & 100 & 111.59& 4.8086 & 3128\\
		\hline
		100,000& LISA Baseline & 1000 & 111.97 & 0.7271 & 3149\\
		\hline
		100,000& LISA Baseline & 10000 & 128.49 & 0.3301 & 3360\\
		\hline
		100,000& LISA Baseline & 100000 & 272.93 & 0.2381 & 5469\\
		\hline
		1,000,000& LISA Baseline & 10 & 1094.85 & 347.5613 & 31251\\
		\hline
		1,000,000& LISA Baseline & 100 & 1099.38 & 40.1451 & 31253\\
		\hline
		1,000,000& LISA Baseline & 1000 & 1104.65 & 4.4732 & 31274\\
		\hline
		1,000,000& LISA Baseline & 10000 & 1143.65 & 0.6697 & 31485\\
		\hline
		1,000,000& LISA Baseline & 100000 & 1273.56 &  0.2944 &33594 \\
		\hline
		1,000,000& LISA Baseline & 1000000 & 2717.65 & 0.2436 & 54688\\
		\hline
		\hline
	\end{tabular}
    \caption{Hyper-parameters Search LISA Baseline Model for training sizes $10K$, $100K$ and $1M$ \\
    %a) For any training size, optimum value of $N$ will be equal to the number of keys in training data.\\
    %b) If $N$ is assigned as numbers of keys in database, number of keys per cell will be 1, and
    %query search cost will be reduced to $log_{2}N$.}
    }
    \label{small_lognormal_lisa_baseline_10000}
\end{table}

\begin{table}
	\centering
	\begin{tabular}{||p{0.15\textwidth}<{\centering}|p{0.2\textwidth}<{\centering}| p{0.1\textwidth}<{\centering}|p{0.15\textwidth}<{\centering}|p{0.15\textwidth}<{\centering}|p{0.15\textwidth}<{\centering}||}
		\hline
		Training/Test Data Size& Model & $N$ & Build Time(ms) & Avg Query Time(ms) & Memory Size(KB)\\ [0.5ex] 
		\hline
		\hline
		10,000& LISA Baseline Optimized& 10 & 11.1208 & 0.2841 & 313\\
		\hline
		10,000& LISA Baseline Optimized & 100 & 12.0108 & 0.2779 & 315\\
		\hline
		10,000& LISA Baseline Optimized & 1000 & 12.7589 & 0.2765 & 336\\
		\hline
		10,000& LISA Baseline Optimized & 10000 & 25.8732 & 0.2752 & 547\\
		\hline
		100,000& LISA Baseline Optimized & 10 & 112.973 & 0.2855 & 3126\\
		\hline
		100,000& LISA Baseline Optimized & 100 & 114.318 & 0.2823 & 3128\\
		\hline
		100,000& LISA Baseline Optimized & 1000 & 116.699 & 0.2806 & 3149\\
		\hline
		100,000& LISA Baseline Optimized & 10000 & 129.514 & 0.2794 & 3360\\
		
		\hline
		1,000,000& LISA Baseline Optimized & 10 & 1116.51 & 0.2905 & 31251\\
		\hline
		1,000,000& LISA Baseline Optimized & 100 & 1118.85 &0.2858 & 31253\\
		\hline
		1,000,000& LISA Baseline Optimized & 1000 & 1134.88 & 0.2844 & 31274\\
		\hline
		1,000,000& LISA Baseline Optimized & 10000 & 1134.88 & 0.2831 & 31485\\
		
		\hline
		\hline
	\end{tabular}
	\caption{Experimental results for LISA Baseline model with search optimization}
	\label{baseline_search_optimization_10000}

\end{table}


\begin{table}
	\centering
	\begin{tabular}{||p{0.14\textwidth}<{\centering}|p{0.08\textwidth}<{\centering}|p{0.15\textwidth}<{\centering}| p{0.07\textwidth}<{\centering}|p{0.1\textwidth}<{\centering}|p{0.125\textwidth}<{\centering}|p{0.125\textwidth}<{\centering}|p{0.07\textwidth}<{\centering}||}
		\hline
		Training/Test Data Size& Model & G & S& Build Time(s) & Avg Query Time(ms) & Memory Size(KB) &mse\\ [0.5ex] 
		\hline
		\hline
		10,000& LISA& 4*4=16 & 5& 4.335& 1.13135 & 324.72&0\\
		\hline
		10,000& LISA& 4*4=16 & 10& 3.370& 0.96036 & 329.07&0\\
		\hline
		10,000& LISA& 4*4=16 & 20&1.127& 0.86184 & 337.85&0\\
		\hline
		10,000& LISA& 4*4=16 & 30&3.478& 0.74339 & 346.63&5729\\
		\hline
	    \hline
	\end{tabular}

    \caption{Hyper-parameters Search LISA Model: Training Size:10,000 Points. \\
    a) For the last row, Numbers of keys= 10000 \\
    b) Keys per cell= $10000 \setminus (4\times4) = 625$\\
    c) Keys per shard = $625\setminus30=20$ keys per shard, resulting in mse errors}
	\label{small_lognormal_lisa_10000}
\end{table}

\begin{table}
	\centering
	\begin{tabular}{||p{0.14\textwidth}<{\centering}|p{0.08\textwidth}<{\centering}|p{0.15\textwidth}<{\centering}| p{0.07\textwidth}<{\centering}|p{0.1\textwidth}<{\centering}|p{0.125\textwidth}<{\centering}|p{0.125\textwidth}<{\centering}|p{0.08\textwidth}<{\centering}||}
		\hline
		Training/Test Data Size& Model & GridCellSize & No of Shards & Build Time(s) & Avg Query Time(ms) & Memory Size(KB)&mse\\ [0.5ex] 
		\hline
		\hline
	    %100,000& LISA& 4*4=16 & 5& 45.846& 5.93345 & 3137.2&0\\
		%\hline
		%100,000& LISA& 4*4=16 & 10& 42.398& 3.29308 & 3141.6&0\\
		%\hline
		%100,000& LISA& 4*4=16 & 20& 59.036&1.52851 & 3150.3&0\\
		\hline
		100,000& LISA& 4*4=16 & 50& 122.64& 1.51173 & 3176.6&0\\
		\hline
		100,000& LISA& 4*4=16 & 100& 30.211& 1.44084 & 3220.3&0\\
		\hline
		100,000& LISA& 4*4=16 & 150& 142.13&1.15491 & 3264.1&297234\\
		%\hline
		%100,000& LISA& 6*6=36 & 20&33.637&1.59742 & 3178.9&0\\
		\hline
		100,000& LISA& 6*6=36 & 50& 66.375& 1.55903 & 3238.1&0\\
		\hline
		100,000& LISA& 6*6=36 & 75& 72.491& 1.43043 & 3287.2&0\\
		\hline
		100,000& LISA& 6*6=36 & 100& 60.929& 1.64881 & 3336.4&5.6e+07\\
		\hline
		100,000& LISA& 8*8=64 & 20& 35.638& 1.54029 & 3218.7&0\\
		\hline
		100,000& LISA& 8*8=64 & 50& 45.014& 1.52117 & 3323.6&0\\
		\hline
		\hline
	\end{tabular}
    \caption{Hyper-parameters Search LISA Model: Training Size:100,000 Points}
	\label{small_lognormal_lisa_100000}
\end{table}

\begin{table}
	\centering
\centering
	\begin{tabular}{||p{0.14\textwidth}<{\centering}|p{0.08\textwidth}<{\centering}|p{0.15\textwidth}<{\centering}| p{0.07\textwidth}<{\centering}|p{0.1\textwidth}<{\centering}|p{0.125\textwidth}<{\centering}|p{0.125\textwidth}<{\centering}|p{0.08\textwidth}<{\centering}||}
		\hline
		Training/Test Data Size& Model & GridCellSize & No of Shards& Build Time(s) & Avg Query Time(ms) & Memory Size(KB)&mse\\ [0.5ex] 
		\hline
		\hline
	 	%1,000,000&LISA& 10*10=100 & 5& 122.64& 1.51173 & 3176.6&0\\
		%\hline
		%1,000,000& LISA& 10*10=100 & 10& 30.211& 1.44518 & 3220.3&0\\
		%\hline
		%1,000,000& LISA& 10*10=100 & 20& 24.428&3.19491 & 3149.4&0\\
		%\hline
		1,000,000& LISA& 10*10=100 & 50&743.29&1.77751 & 31558.9&0\\
		\hline
		1,000,000& LISA& 10*10=100 & 100& 1077.89& 1.63397 & 31832.3&0\\
		\hline
		1,000,000& LISA& 20*20=400 & 25& 365.49& 2.53317 & 31930.8&0\\
		\hline
		1,000,000& LISA& 20*20=400 & 50& 609.32& 1.44526 & 32477.6&0\\
		\hline
		1,000,000& LISA& 25*25=625 & 25& 240.22& 1.56227& 32779.8&0\\
		\hline
		1,000,000& LISA& 30*30=900 & 25& 205.18& 1.79839 & 33010.3&0\\
		\hline
		\hline
	\end{tabular}
    \caption{Hyper-parameters Search LISA Model: Training Size:1,000,000 Points}
	\label{small_lognormal_lisa_1000000}
\end{table}

\begin{table}
	\centering
\centering
	\begin{tabular}{||p{0.20\textwidth}<{\centering}|p{0.15\textwidth}<{\centering}|p{0.20\textwidth}<{\centering}|p{0.20\textwidth}<{\centering}|p{0.21\textwidth}<{\centering}||}
		\hline
		Training/Test Data Size& Model & Build Time (s) & Avg Query Time (ms) & Memory Size (KB)\\ [0.5ex] 
		\hline
		\hline
	 	10,000& KD-Tree & 0.023 & 4.363 & 2890 \\
	 	\hline
	 	10,000& Baseline & 0.026 & 0.198 & 547\\
	 	\hline
	 	10,000& LISA & 1.127&0.861 & 337\\
		\hline
	 	100,000& KD-Tree & 0.340 & 6.176 & 28906 \\
	 	\hline
	 	100,000& Baseline & 0.324 & 0.241 & 5469\\
	 	\hline
	 	100,000& LISA& 22.491& 1.43 & 3169\\
		\hline
	 	1,000,000& KD-Tree& 4.124 & 9.254 & 289062 \\
	 	\hline
	 	1,000,000& Baseline& 2.718 & 0.343 & 54688\\
	 	\hline
	 	1,000,000& LISA& 445.324&1.445 & 32477 \\
		\hline
		\hline
	\end{tabular}
	\caption{Point Query experimental results for KDTree, Baseline and LISA models}
	\label{Point_Query_Comparision}
\end{table}


\begin{table}
	\centering
\centering
	\begin{tabular}{||p{0.15\textwidth}<{\centering}|p{0.15\textwidth}<{\centering}|p{0.22\textwidth}<{\centering}|p{0.25\textwidth}<{\centering}|p{0.20\textwidth}<{\centering}||}
		\hline
		Training/Test Data Size& Range Query Size & Avg Query Time (ms) ($K$D-tree) & Avg Query Time (ms) (Baseline) &Avg Query Time (ms) (LISA)\\ [0.5ex] 
		\hline
		\hline
	 	10,000& 10& 0.1361 & 0.1113& 0.8204 \\
	 	\hline
	 	10,000& 100& 0.0533 & 0.0451& 0.1201 \\
	 	\hline
	 	10,000& 1000& 0.0438 & 0.0399& 0.0294 \\
 	 	\hline
 	 	10,000& 10000& 0.0648&0.0382&0.0061 \\
	 	\hline
	 	100,000& 10& 0.1392 & 0.1298&2.8961 \\
	 	\hline
	 	100,000& 100& 0.0539 & 0.0505&0.2792 \\
	 	\hline
	 	100,000& 1000& 0.043 & 0.0428&  0.0563 \\
 	 	\hline
 	 	100,000& 10000&0.0718& 0.0392& 0.0129 \\
	 	\hline
	    1,000,000& 10& 0.2238 & 0.2661& 3.5181 \\
	 	\hline
	 	1,000,000& 100& 0.0922 & 0.0617&0.6263 \\
	 	\hline
	 	1,000,000& 1000& 0.0744 & 0.0437 &0.0939 \\
	 	\hline
	 	1,000,000& 10000& 0.0735 & 0.0412 &0.0186 \\
	 	%\hline
	 	%10,000,000& 10& 0.2285 & 0.037188 &12.891 \\
	 	%\hline
	 	%10,000,000& 100& 0.0779 & 0.037188 &0.1611 \\
	 	%\hline
	 	%10,000,000& 1000& 0.0740 &  0.037188 &0.0353 \\
	 	
	 	%\hline
	 	%10,000,000& 10000& 0.0773 &  0.037188 &0.0 \\
	 
		\hline
		\hline
	\end{tabular}
	\caption{Range Query experimental results for $K$D-tree, Baseline and LISA models}
	\label{Range_Query_Experimental_Results}

\end{table}

\begin{table}
	\centering
\centering
	\begin{tabular}{||p{0.15\textwidth}<{\centering}|p{0.25\textwidth}<{\centering}|p{0.25\textwidth}<{\centering}|p{0.25\textwidth}<{\centering}||}
		\hline
		Training/Test Data Size& $K$ & Avg Query Time(ms)($K$D-tree) & Avg Query Time(ms)(LISA)\\ [0.5ex] 
		\hline
		\hline
	 	10,000& 3& 4.2207 &0.6020 \\
	 	\hline
	 	10,000& 5& 4.3765 &0.6084\\
	 	\hline
	 	10,000& 7& 4.4331 &0.6129 \\
	 	\hline
	 	10,000 & 10&4.3682 &0.6493 \\
	 	\hline
	 	100,000 & 3& 6.0021 &1.7601 \\
	 	\hline
	 	100,000 & 5& 6.0975 &1.7745 \\
	 	\hline
	 	100,000 & 7& 6.1684 &1.9453 \\
	 	\hline
	 	100,000 & 10& 6.6183 &2.1617 \\
	 	\hline
	    1,000,000 & 3& 8.3595 & 3.6549 \\
	 	\hline
	 	1,000,000 & 5& 8.5362 &4.7073 \\
	 	\hline
	 	1,000,000 & 7& 9.2785 &5.3767 \\
	 	\hline
	 	1,000,000 & 10& 9.1726 &5.7799 \\
	 	%\hline
	    %1,000,000 & 3& 3.888 & 3.361 \\
	 	%\hline
	 	%1,000,000 & 5& 2.3613 &3.192 \\
	 	%\hline
	 	%1,000,000 & 7& 1.6812 & 1.748 \\
	 	%\hline
	 	%1,000,000 & 10& 1.2151 & 1.378 \\
		\hline
		\hline
	\end{tabular}
	\caption{KNN Query experimental results for $K$D-tree and LISA model}
	\label{KNN_Query_Experimental_Results}

\end{table}




\end{appendices}


\bibliographystyle{plain}
\bibliography{refs}

\end{document}
