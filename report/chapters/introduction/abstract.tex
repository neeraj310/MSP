Databases use indexes to find records efficiently. Among these indexes, B-tree and KD-tree are two successful indexes used for 1-dimensional and 2-dimensional data. In this project, we implement both the learned index \cite{kraska2018case} for 1-D data and the learned index, named LISA \cite{li2020lisa} for 2-D data from scratch using Python. Afterwards, we have conducted sanity check to ensure that our implementations are correct. In addition to the implementation and evaluation, we have theoretically analyse some properties that the learned indexes hold. Beyond that, we also explore and discuss the properties that the learned indexes should hold.

As an extension to the existing learned indexes, we explore the possibilities of using convolution operation and convolutional neural network as learned indexes and report our results. 