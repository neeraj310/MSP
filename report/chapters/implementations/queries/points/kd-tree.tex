Similar to search with binary search tree, we also need to traverse the tree in order to perform point query. However, we need to switch the dimensions when we compare the values between the query key and the values in the nodes.

\begin{algorithm}[H]
    \SetAlgoLined
    \SetKwInOut{Input}{Input}
    \SetKwInOut{Output}{Output}
    \Input{\texttt{t}: The node being searched; \texttt{x}: The query key; \texttt{cd}: Current dimension}
    \Output{\texttt{n}: the node that contains the query key}
    \texttt{DIM=2}\\
    \uIf{\texttt{t==NULL}} {
    	\Return \texttt{NULL}
    }
    \uIf{\texttt{x[0]==t.data[0]} and \texttt{x[1]==t.data[1]}} {
    	\Return \texttt{t}
    }
    \uElseIf{\texttt{x[cd]<t.data}} {
    	\Return \texttt{pointSearch(t.left, x, (cd+1) \% DIM)}
    }
    \uElseIf{\texttt{x[cd]>t.data}} {
    	\Return \texttt{pointSearch(t.right, x, (cd+1) \% DIM)}
    }
    \caption{Point Query with $K$D-Tree}
    \label{algo: point_query_kdtree}
\end{algorithm}

The point query works in the following steps:

\begin{enumerate}
	\item From Line $2$ to $3$, we first check if current node is \texttt{NULL}. If so, that means that we have already traversed all the possible nodes and found nothing. In this case, the query returns \texttt{NULL}.
	\item From Line $4$ to $5$, we check if the current node contains the same key as the query key. If so, the current node is the node that we are looking for. Hence, we return the current node in this case.
	\item Otherwise, from Line $6$ to $9$, we check if the current dimension of the query key is smaller, larger or equal to the current dimension of the data in the node.
	\begin{enumerate}
	\item If it is smaller, then we search on the left subtree of current node, with the same query key and switched dimension.
	\item If it is larger, then we search on the right subtree of current node, with the same query key and switched dimension.
	\end{enumerate}
\end{enumerate}

\begin{mscexample}    
	In the previous figure \ref{fig:kd_tree_example}, we showed an example $K$D-tree. If we want to search for $(50, 30)$ in this tree, we would follow the following steps:
	\begin{enumerate}
		\item We first check the root node and compares the $x$-coordinate. As $50>30$, we go to the right subtree of the root node.
		\item Then in the subtree, we compare the $y$-coordinate. As $50<70$, we go to the left subtree of this node.
		\item Then in the left subtree, the termination condition is reached, hence we return this node as result.
	\end{enumerate}
\end{mscexample}
