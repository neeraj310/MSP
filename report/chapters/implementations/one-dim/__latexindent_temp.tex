B-Tree and its variants have been widely used as indexes in databases. B-Trees can be considered as a generalisation of binary search tree: In binary search tree, there is only one key and two children at most in the internal node. B-Tree extends the nodes such that each node can contain several keys and children. The keys in a node serve as dividing points and separate the range of keys. With this structure, we make a multi-way decision based on comparisons with the keys stored at the node $x$.

In this section, we introduce the construction and search processes
of B-Trees and then analyse their properties.

\subsubsection{Attributes and Properties}

Each node in a B-Tree has the following attributes:

\begin{itemize}
\item
  $x.n$: the number of keys currently stored in the node $x$.
\item Inside each node, the keys are sorted in non decreasing order, so that we have $x.keys_1\leq x.keys_2\leq\cdots\leq x.keys_{x.n}$.
\item
  $x.leaf$, a Boolean value determines if current node is a leaf node.
\end{itemize}

With these properties, A B-Tree $T$ whose root is $T.root$ have the
following properties:

\begin{itemize}
\item
  Each internal node $x$ contains $x.n+1$ children. We assume the
  children are $x.c_1,\cdots,x.c_{x.n+1}$.
\item
  The nodes in the tree have lower and upper bounds on the number of
  keys that can contain. These bounds can be expressed in terms of a
  fixed integer $t$.
\end{itemize}


\subsubsection{Insertion in a B-Tree}

