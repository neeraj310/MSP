The point query of recursive model is a top-down process. With a given $x$, the root model will first predict an output that represents the probability that $F(X\leq x)$. Then we map this output into the index of models in the next stage. Afterwards, we use that model to predict an output with the given $x$. We iterate these steps until the last stage in which we use the output as the final output. The above process is described in Algorithm. \ref{algo:point_query_rmi}


\begin{algorithm}[H]
    \SetAlgoLined
    \SetKwInOut{Input}{input}
    \Input{\texttt{x; models; num\_of\_stages}}
     \texttt{stage$\gets 0$} \\
 	 \texttt{next\_model$\gets 0$} \\
     \While{\texttt{stage} \textless \texttt{num\_of\_stages}}{
     	\texttt{model=models[stage][next\_model]}\\
        \texttt{output = model.predict(x)} \\
      \uIf{\texttt{stage==num\_of\_stages-1}} {
		\texttt{y = output}
      }\Else {
      	\texttt{next\_model=output*len(models[stage+1])}\\
      	\texttt{stage = stage+1}
      }
     }
     \Return \texttt{y}
     \caption{Point Query With Recursive Model Index}
     \label{algo:point_query_rmi}
\end{algorithm}

In the query algorithm, we have three inputs: $x$ as the query key, trained models and the number of stages.

On line \texttt{1-2}, we first initialise the $stage$ and $next\_model$ to be $0$, so that we use the root model at the very beginning. Then on line \texttt{3}, we iterate over all stages. In each stage, we perform the following actions:
\begin{enumerate}
	\item On line \texttt{4}, we access the model at $stage$ whose index is $next\_model$.
	\item On line \texttt{5}, we perform the prediction with the query key $x$ and the model selected by the previous step.
	\item On line \texttt{6}, we check if current stage is the last stage. 
	\begin{enumerate}
		\item If it is, then we get the final output, which equals to the output from line \texttt{5}.
		\item If it is not the last stage, then we map the output from previous step into the index of the model in the next stage. As there are \texttt{len(models[stage+1])} models in the next stage and the output represents some probability (hence, $\texttt{output}\in [0,1]$), we multiply them and find the \texttt{next\_model}. In the meanwhile, we add $1$ to the stage.
	\end{enumerate}
	\item At the end, we return the final output, which is the output from the model in the last stage.
\end{enumerate}

After calculating the output as described in the Algorithm. \ref{algo:point_query_rmi}, we calculate the page index in a same way as we described in the baseline model. 
