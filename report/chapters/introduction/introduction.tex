Over the years, indexes have been widely used in databases to improve the speed of data retrieval. In the past decades, the database indexes generally fall into hand-engineered data structures and algorithms, such as B-Tree, KD-Tree, Hash Table, etc. These indexes have played an important role in databases and have been widely used in modern data management systems (DBMS). Despite their success, they do not consider the distribution of the database entries, which might be helpful in designing faster indexes.

For example, if the dataset contains integers from $1$ to $1$ million, the key can be used directly as an offset. With the key used as an offset, the values with the key can be retrieved in $\mathcal{O}(1)$ time complexity. Compared with B-Tree, which always takes $\mathcal{O}(\log n)$ time complexity for the same query. At the same time, by using the key as an offset directly, we do not need any extra overhead regarding memory space, where the B-Tree needs extra $\mathcal{O}(n)$ space complexity to save the tree.

From the above example, we found there are two promising advantages of learned indexes over hand-engineered indexes:
\begin{enumerate}
  \item Learned indexes may be faster when performing queries, especially when the number of entries in the database are extremely huge.
  \item Learned indexes may take less memory space, as we only need to save the model with constant size.
  \end{enumerate}
  
 We will explore and analyse these two advantages qualitatively in the \textit{chapter 5}. 

Nowadays, to leverage these two advantages, researchers proposed learned indexes \cite{kraska2018case}, where machine learning techniques are applied to automatically learn the distribution of the database entries and build the data-driven indexes. This approach has been shown to be powerful and competitive compared with hand-engineered indexes, such as B-Tree.

In this report, we explore the development of database indexes, from hand-engineered indexes to the learned index. After that, we explore the possibilities of using complex convolutional neural networks as database indexes. This report is organised into the following chapters:

\begin{enumerate}
	\item \textbf{Introduction}. In this chapter, we illustrate the organisation of this report. Besides, we go through the modern computer systems and introduce the general information about database indexes.
	\item \textbf{Implementation}. In this chapter, we thoroughly describe the implementation of one and two dimensional indexes, including B-Tree, baseline learned index, recursive model, KD-Tree and LISA.
	\item \textbf{Evaluation}. In this chapter, we perform evaluation among the indexes we implemented with different evaluation dataset. 
	\item \textbf{Insights and Findings}. We demonstrate our findings during the implementation in this chapter. Besides, we also discuss the advantages and disadvantages of different indexes.
	\item \textbf{Conclusions}. 
\end{enumerate}

\section{Notations}

In this report, we will use the following notations:

\begin{table}[h]
\begin{tabularx}{\textwidth}{@{}XX@{}}
\toprule
  \underline{Sets and Spaces} \\
  $\mathbb{R}$ & The set of real numbers \\
  $\mathbb{R}^d$ & The set of $d$ dimensional real space \\
  \underline{Random Variables} \\
  $\boldsymbol{X}$ & A vector or matrix \\
  $x$ & A single value in $\textbf{X}$ \\
  $(x,y)$ & A tuple contains two values \\	
  \underline{Hyper-Parameters} \\
  N   & A pre-set hyper parameter \\
  \underline{Functions} \\
  $\mathcal{LR}$ & Linear Regression Function\\
  $\mathcal{P}$ & Polynomial Function\\
  $\mathcal{M}$ & Mapping Function\\
  $\mathcal{O}$ & Big-O notation for complexity\\
  $\mathcal{SP}$ & Shard Prediction Function\\
  $\mathcal{Q}$ & Range Query \\
  $\mathcal{K}$ & $K$NN Query \\
  \underline{Others} \\
  $\clubsuit$ & End of Example \\
  $\blacksquare$ & End of Proof \\
\bottomrule
\end{tabularx}
\end{table}


\section{Terminologies}

In the following chapters, we will use the following terminologies

\textbf{Index model} is a function that maps the index of a row of data into the location (e.g. page index) of the data. For example, in one-dimensional case, the index models include B-Tree, Linear Regression models, etc.

\textbf{Key} is a special attribute in the database that could identify a record. In our work, the key could be a scalar in one-dimensional case, or a $(x,y)$ pair in two-dimensional case.

\textbf{Order of a tree} is the maximum number of children that a node can have.

\textbf{Internal node} is any node of a tree that has child nodes and is not a root node.

\textbf{Leaf node} is any node that does not have child nodes.

\textbf{Level} of a node is defined as the number of edges between this node and the root node.


\section{Motivation}

In traditional database indexes, the complexity for locating an item is usually bounded by some function related to the total number of elements. For example, with a B-Tree, an item can be found within $\mathcal{O}(\log n)$ time complexity. In the meantime, saving a B-Tree as index takes $n$ space complexity. With the rapid growing of the volume of data, $n$ becomes much larger than ever before. Hence, the big data era is calling for a database index that have constant complexity in both time and space.

To achieve such a goal, the distribution of the data is important. For example, assume that the data is fixed-length records over a set of continuous integers from 1 to 100 million, the conventional B-Tree index can be replaced by the keys themselves, making the query time complexity an $\mathcal{O}(1)$ rather than $\mathcal{O}(\log n)$. Similarly, the space complexity would be reduced from $\mathcal{O}(n)$ to $\mathcal{O}(1)$. This example shows that with the knowledge of the distribution of the data, it is possible to locate the item in database in constant time.

Formally, we define the index of each record as $x$ and the corresponding location as $y$ and we represent the whole data as $(X, Y)$ pairs with the total number of pairs defined as $N$. We could then normalise the $Y$ into $\tilde{Y}\in[0,1]$ so that the $\tilde{y}$ represents the portion of the $y$ among the whole $Y$. With these definitions, we can then define a function $F:X\to \tilde{Y}$ that maps the index into the portion of the $y$. We have $y=F(x)* N$. As the output of this function can be considered as the probability of $X\leq x$, we can regard this function $F(x)$ as the cumulative distribution function (CDF) of $X$, i.e. $F(x)=\mathbb{P}(X\leq x)$. Now that $N$ is determined by the length of data records, we only need to learn such CDF and we called the learned CDF function as \textbf{learned index model}.

From the perspective of the distribution of data records, our previous example can be rephrased as following. Our data records are $(X, Y)$ pairs with a linear relation, i.e. $y=x, \forall y\in Y$. We are looking for a function $F$ such that $y=x=F(x)* N$, and hence we end up with $F(x)=\frac{1}{N}*x$. If we use this linear function $F(x)$ as the index model, then we could locate the data within $\mathcal{O}(1)$ time complexity and we only need to store the total number of records as the only parameter. Compared with B-Tree and other indexes, the advantages are enormous.

Even though there might be potential advantages, the learned index model has several assumptions, as listed below.
\begin{enumerate}
	\item All data records are stored in memory. 
	\item All data records are sorted by $X$.
	\item All data records are stored statically in database, hence we do not take insertion and deletion into consideration.
\end{enumerate}


