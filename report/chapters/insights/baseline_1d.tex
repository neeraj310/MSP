\subsubsection{Activation Functions}
\begin{itemize}
\item
  If we use identity activation function, i.e.$z^{(i)}(x)=x$, then no
  matter how many layers are there, the fully connected neural network
  falls back to a linear regression.

  \textbf{Proof:} The output of the first layer, with identity
  activation function, will be
  \(z^{(1)}(w^{(1)}x+b^{(1)})=w^{(1)}x+b^{(1)}\). Then the output will
  be the input of the next layer, and hence the output of the second
  layer will be
  \(z^{(2)}(w^{(2)}(w^{(1)}x+b^{(1)})+b^{(2)})=w^{(2)}w^{(1)}x+w^{(2)}b^{(1)}+b^{(2)}\).
  Similar induction can be obtained for multiple layers. Hence if we use
  identity activation, the trained neural network will fall back to a
  linear regression. 
\item
  With ReLU (Rectified Linear Unit) as activation function
  i.e.~\(z^{(i)}(x)=\text{max}(0,x)\), then the fully connected neural
  network falls back to a piecewise linear function.

  \textbf{Proof:} The output with ReLU activation function, will be
  \(z^{(1)}(w^{(1)}x+b^{(1)})=\text{max}(w^{(1)}x+b^{(1)}, 0)\). Then
  the output will be the input of the next layer, and hence the output
  of the second layer will be
  \(z^{(2)}(w^{(2)}(w^{(1)}x+b^{(1)})+b^{(2)})=\text{max}(w^{(2)}w^{(1)}x+w^{(2)}b^{(1)}+b^{(2)},0)\).
  Similar induction can be obtained for multiple layers. Hence if we use
  identity activation, the trained neural network will fall back to a
  piecewise linear function. The visualization below shows our lemma is
  correct.
\end{itemize}