Databases use indexes to efficiently find records. B-tree and KD-tree are the two of the indexes used for 1-dimensional and 2-dimensional data. In this project, we first implement these two indexes from scratch and then we implemented the learned indexes, including a fully connected neural network and a recursive model for 1-dimensional data \cite{kraska2018case} and the for 2-dimensional data \cite{li2020lisa}. Afterwards we conduct several experiments to evaluate the performance of learned indexes compared with traditional B-Tree and KD-Tree. In addition to the implementation and evaluation, we then theoretically analyse the properties that the learned indexes hold and should hold.

As extension to the existing learned indexes, we also explore the possibilities of using convolution operation and convolutional neural network to improve the performance of learned indexes.