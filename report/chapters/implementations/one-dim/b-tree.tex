B-tree and its variants have been widely used as indexes in databases. B-trees can be considered as a generalisation of binary search tree: In binary search tree, there is only one key and two children at most in the internal node. B-tree extends the nodes such that each node can contain several keys and children. The keys in a node serve as dividing points and separate the range of keys. With this structure, we make a multi-way decision based on comparisons with the keys stored at the node $x$.

In this section, we introduce the construction process of B-trees and then analyse its properties.

\subsubsection{Attributes and Properties}

Each node \texttt{x} in a B-tree has the following attributes:

\begin{itemize}
\item \texttt{x.n}: the number of keys currently stored in the node $x$.
\item \texttt{x.keys}: the stored keys of this node.
\item \texttt{x.leaf}: a bool value that determines if current node is a leaf node.
\item \texttt{x.children}: a list of its children. If \texttt{x} is a leaf node who has no children at all, then the list will be empty. We assume the children are \texttt{x.$c_1$,$\cdots$,x.$c_{x.n+1}$}, i.e. there will be $\texttt{x.n}+1$ children at most.
\end{itemize}

\noindent
With these attributes, a B-tree has the following properties:

\begin{itemize}
\item The number of children of a node is always $1$ bigger than the number of keys in a node.
\item Nodes in this tree have lower and upper bounds on the number of keys they can contain. These two bounds can be expressed in terms of a fixed integer $t$, which we call the \textbf{degree} of this tree.
	\begin{enumerate}
		\item Each node, other than the root node, must contain at least $t-1$ keys. The root of the tree must have at least one key if the tree is not empty.
		\item Each node can contain at most $2t-1$ keys. A node is called \textbf{full} if it contains exactly $2t-1$ keys.
	\end{enumerate}
\item Inside each node, the keys are sorted in the non-decreasing order, so that we have \texttt{x.keys$_1\leq $} \texttt{x.keys$_2\leq \cdots \leq$} \texttt{x.keys$_{\texttt{x.n}}$}.
\item The keys \texttt{x.key$_i$} separate the ranges of keys stored in each subtree: if $k_i$ is any key stored in the subtree with a root \texttt{x.c$_i$}, then we have $k_1\leq$\texttt{x.keys$_1\leq $}$ k_2\leq$\texttt{x.keys$_2\leq $} $\cdots\leq$ \texttt{x.keys$_n\leq $}$k_{\texttt{x.n}+1}$.
\end{itemize}

In Fig. \ref{fig: B-tree}, we demonstrate an example B-tree whose degree is $2$. In the following section, we will illustrate how to construct and insert keys into a B-tree.

\begin{figure}
\centering
B-tree and its variants have been widely used as indexes in databases. B-trees can be considered as a generalisation of binary search tree: In binary search tree, there is only one key and two children at most in the internal node. B-tree extends the nodes such that each node can contain several keys and children. The keys in a node serve as dividing points and separate the range of keys. With this structure, we make a multi-way decision based on comparisons with the keys stored at the node $x$.

In this section, we introduce the construction process of B-trees and then analyse its properties.

\subsubsection{Attributes and Properties}

Each node \texttt{x} in a B-tree has the following attributes:

\begin{itemize}
\item \texttt{x.n}: the number of keys currently stored in the node $x$.
\item \texttt{x.keys}: the stored keys of this node.
\item \texttt{x.leaf}: a bool value that determines if current node is a leaf node.
\item \texttt{x.children}: a list of its children. If \texttt{x} is a leaf node who has no children at all, then the list will be empty. We assume the children are \texttt{x.$c_1$,$\cdots$,x.$c_{x.n+1}$}, i.e. there will be $\texttt{x.n}+1$ children at most.
\end{itemize}

\noindent
With these attributes, a B-tree has the following properties:

\begin{itemize}
\item The number of children of a node is always $1$ bigger than the number of keys in a node.
\item Nodes in this tree have lower and upper bounds on the number of keys they can contain. These two bounds can be expressed in terms of a fixed integer $t$, which we call the \textbf{minimum degree} of this tree.
	\begin{enumerate}
		\item Each node, other than the root node, must contain at least $t-1$ keys. The root of the tree must have at least one key if the tree is not empty.
		\item Each node can contain at most $2t-1$ keys. A node is called \textbf{full} if it contains exactly $2t-1$ keys.
	\end{enumerate}
\item Inside each node, the keys are sorted in the non-decreasing order, so that we have \texttt{x.keys$_1\leq $} \texttt{x.keys$_2\leq \cdots \leq$} \texttt{x.keys$_{\texttt{x.n}}$}.
\item The keys \texttt{x.key$_i$} separate the ranges of keys stored in each subtree: if $k_i$ is any key stored in the subtree with a root \texttt{x.c$_i$}, then we have $k_1\leq$\texttt{x.keys$_1\leq $}$ k_2\leq$\texttt{x.keys$_2\leq $} $\cdots\leq$ \texttt{x.keys$_n\leq $}$k_{\texttt{x.n}+1}$.
\end{itemize}

In Fig. \ref{fig: B-tree}, we demonstrate an example B-tree whose minimum degree is $2$. In the following section, we will illustrate how to construct and insert keys into a B-tree.

\begin{figure}
\centering
B-tree and its variants have been widely used as indexes in databases. B-trees can be considered as a generalisation of binary search tree: In binary search tree, there is only one key and two children at most in the internal node. B-tree extends the nodes such that each node can contain several keys and children. The keys in a node serve as dividing points and separate the range of keys. With this structure, we make a multi-way decision based on comparisons with the keys stored at the node $x$.

In this section, we introduce the construction process of B-trees and then analyse its properties.

\subsubsection{Attributes and Properties}

Each node \texttt{x} in a B-tree has the following attributes:

\begin{itemize}
\item \texttt{x.n}: the number of keys currently stored in the node $x$.
\item \texttt{x.keys}: the stored keys of this node.
\item \texttt{x.leaf}: a bool value that determines if current node is a leaf node.
\item \texttt{x.children}: a list of its children. If \texttt{x} is a leaf node who has no children at all, then the list will be empty. We assume the children are \texttt{x.$c_1$,$\cdots$,x.$c_{x.n+1}$}, i.e. there will be $\texttt{x.n}+1$ children at most.
\end{itemize}

\noindent
With these attributes, a B-tree has the following properties:

\begin{itemize}
\item The number of children of a node is always $1$ bigger than the number of keys in a node.
\item Nodes in this tree have lower and upper bounds on the number of keys they can contain. These two bounds can be expressed in terms of a fixed integer $t$, which we call the \textbf{minimum degree} of this tree.
	\begin{enumerate}
		\item Each node, other than the root node, must contain at least $t-1$ keys. The root of the tree must have at least one key if the tree is not empty.
		\item Each node can contain at most $2t-1$ keys. A node is called \textbf{full} if it contains exactly $2t-1$ keys.
	\end{enumerate}
\item Inside each node, the keys are sorted in the non-decreasing order, so that we have \texttt{x.keys$_1\leq $} \texttt{x.keys$_2\leq \cdots \leq$} \texttt{x.keys$_{\texttt{x.n}}$}.
\item The keys \texttt{x.key$_i$} separate the ranges of keys stored in each subtree: if $k_i$ is any key stored in the subtree with a root \texttt{x.c$_i$}, then we have $k_1\leq$\texttt{x.keys$_1\leq $}$ k_2\leq$\texttt{x.keys$_2\leq $} $\cdots\leq$ \texttt{x.keys$_n\leq $}$k_{\texttt{x.n}+1}$.
\end{itemize}

In Fig. \ref{fig: B-tree}, we demonstrate an example B-tree whose minimum degree is $2$. In the following section, we will illustrate how to construct and insert keys into a B-tree.

\begin{figure}
\centering
B-tree and its variants have been widely used as indexes in databases. B-trees can be considered as a generalisation of binary search tree: In binary search tree, there is only one key and two children at most in the internal node. B-tree extends the nodes such that each node can contain several keys and children. The keys in a node serve as dividing points and separate the range of keys. With this structure, we make a multi-way decision based on comparisons with the keys stored at the node $x$.

In this section, we introduce the construction process of B-trees and then analyse its properties.

\subsubsection{Attributes and Properties}

Each node \texttt{x} in a B-tree has the following attributes:

\begin{itemize}
\item \texttt{x.n}: the number of keys currently stored in the node $x$.
\item \texttt{x.keys}: the stored keys of this node.
\item \texttt{x.leaf}: a bool value that determines if current node is a leaf node.
\item \texttt{x.children}: a list of its children. If \texttt{x} is a leaf node who has no children at all, then the list will be empty. We assume the children are \texttt{x.$c_1$,$\cdots$,x.$c_{x.n+1}$}, i.e. there will be $\texttt{x.n}+1$ children at most.
\end{itemize}

\noindent
With these attributes, a B-tree has the following properties:

\begin{itemize}
\item The number of children of a node is always $1$ bigger than the number of keys in a node.
\item Nodes in this tree have lower and upper bounds on the number of keys they can contain. These two bounds can be expressed in terms of a fixed integer $t$, which we call the \textbf{minimum degree} of this tree.
	\begin{enumerate}
		\item Each node, other than the root node, must contain at least $t-1$ keys. The root of the tree must have at least one key if the tree is not empty.
		\item Each node can contain at most $2t-1$ keys. A node is called \textbf{full} if it contains exactly $2t-1$ keys.
	\end{enumerate}
\item Inside each node, the keys are sorted in the non-decreasing order, so that we have \texttt{x.keys$_1\leq $} \texttt{x.keys$_2\leq \cdots \leq$} \texttt{x.keys$_{\texttt{x.n}}$}.
\item The keys \texttt{x.key$_i$} separate the ranges of keys stored in each subtree: if $k_i$ is any key stored in the subtree with a root \texttt{x.c$_i$}, then we have $k_1\leq$\texttt{x.keys$_1\leq $}$ k_2\leq$\texttt{x.keys$_2\leq $} $\cdots\leq$ \texttt{x.keys$_n\leq $}$k_{\texttt{x.n}+1}$.
\end{itemize}

In Fig. \ref{fig: B-tree}, we demonstrate an example B-tree whose minimum degree is $2$. In the following section, we will illustrate how to construct and insert keys into a B-tree.

\begin{figure}
\centering
\input{graphs/implementation/one-dim/b-tree}
\caption{An example of B-tree with the minimum degree $t=2$.}
\label{fig: B-tree}
\end{figure}

\subsubsection{Insertion in a B-tree}

With a B-tree, we cannot simply create a new leaf node and insert the new key as we do with a binary search tree, because the resulting tree will fail to be a valid B-tree. Instead, we need to insert the new key into an existing leaf node. If the node is not full, we can safely insert the new key. Otherwise, we will need to split the node around the median of its keys into two new nodes and promote the median key into its parent. In this process, we need to split the parent if its parent is also full.

In the insertion, we travel down the tree and search for the position where the key should be inserted. During the traverse, we split each full node along the way. By doing so, whenever we want to split a full node, we are assured that its parent is not full. The overall algorithm is shown in Algo. \ref{algo:b-tree-insertion}, which contains methods \texttt{splitChild} and \texttt{InsertNonFull} as described in Algo. \ref{algo:b-tree-split-child} and Algo. \ref{algo:b-tree-insert-nonfull} respectively.

\begin{algorithm}[H]
\SetAlgoLined
\SetKwInOut{Input}{input}
\Input{\texttt{T}: The tree with the root \texttt{T.root}; \texttt{k}: The key to be inserted}
\KwResult{\texttt{T}: The tree with the inserted key \texttt{k}}
	\texttt{r=T.root} \\
 \uIf{\texttt{T.n==2t-1}} {
  \texttt{s = NewNode()} \\
  \texttt{T.root = s} \\
  \texttt{s.leaf = False} \\
  \texttt{s.n = 0} \\
  \texttt{s.c$_1$ = r} \\
  \texttt{SplitChild(s, 1)} \\
  \texttt{InsertNonFull(s, k)} \\
 }\uElse{
   \texttt{InsertNonFull(r, k)}
  }
 \caption{Insert}
 \label{algo:b-tree-insertion}
\end{algorithm}

In the Algo. \ref{algo:b-tree-insertion}, we first check if the root node \texttt{r} is full. If it is full, then the root splits and a new node \texttt{s} becomes the root. Then we insert the key \texttt{k} into the tree rooted at the non-full root node, i.e. \texttt{s} or \texttt{r}.

In the Algo. \ref{algo:b-tree-split-child}, the node \texttt{y} originally has $2t$ children (i.e. $2t-1$ keys) and is full. We take the following steps to split it:

\begin{enumerate}
	\item We first (from Line $1$ to Line $11$) create a new node \texttt{z} and give it the largest $t-1$ keys and the corresponding $t$ children of \texttt{y}.
	\item Then we adjust the count of keys for \texttt{y} on Line $12$: after the split, \texttt{y} will have $t-1$ keys.
	\item After that, from Line $13$ to Line $21$, we insert \texttt{z} as a child of \texttt{x}, move the median key from \texttt{y} up to \texttt{x}, and adjust the key count in \texttt{x}.
\end{enumerate}

\begin{algorithm}[H]
\SetAlgoLined
\SetKwInOut{Input}{input}
\Input{\texttt{x}: The node whose children are being split; \texttt{i}: The index of \texttt{x}'s child who is full originally}
\KwResult{\texttt{x}: The parent node whose children are not full}
\texttt{z = NewNode()} \\
\texttt{y = x.c$_i$}  \\
\texttt{z.leaf = y.leaf} \\
\texttt{z.n = t-1}\\
\For{$j\gets1$ \KwTo $t-1$}{
  \texttt{z.keys$_j$ = y.keys$_{j+t}$}\\
}
\uIf{not \texttt{y.leaf}} {
	\For{$j\gets 1$\KwTo $t$} {
		\texttt{z.c$_j$ = y.c$_{j+t}$} \\
	}
}
\texttt{y.n = t-1}\\
\For{$j\gets \texttt{x.n}$ \KwTo $i+1$} {
	\texttt{x.c$_{j+1}$ = x.c$_j$}
}
\texttt{x.c$_{i+1}$ = z} \\
\For{$j\gets \texttt{x.n}$ \KwTo $i$} {
	\texttt{x.keys$_{j+1}$=x.keys$_j$}
}
\texttt{x.key$_i$ = y.key$_t$}\\
\texttt{x.n = x.n+1}
\caption{SplitChild}
\label{algo:b-tree-split-child}	
\end{algorithm}

The Algo. \ref{algo:b-tree-insert-nonfull} works as follows:

\begin{enumerate}
	\item From Line $3$ to Line $6$, We first check if \texttt{x} is a leaf. If it is a leaf, then we insert the key \texttt{k} into \texttt{x}.
	\item If \texttt{x} is not a leaf, then we must insert \texttt{k} into the appropriate leaf node in the subtree rooted at internal node \texttt{x}. From Line $8$ to Line $11$, we traverse the subtree rooted at \texttt{x} and determine the child of \texttt{x} to which the recursion descends. Then we check on Line $12$ if the child where the recursion descends is a full node.
	\item If the child is a full node, we then split the child on Line $13$ into two non-full children. We then determine from Line $14$ to Line $15$ which of the two children is the appropriate node to insert.
	\item At the last, on Line $16$ we look into the $i$th children of \texttt{x} and recursively insert the key \texttt{k} into it.
\end{enumerate}

\begin{algorithm}[H]
\SetAlgoLined
\SetKwInOut{Input}{input}
\Input{\texttt{x}: The node to be inserted; \texttt{k}: The key to be inserted}
\KwResult{\texttt{x}: The node with the inserted key \texttt{k}}
\texttt{i=x.n} \\
\uIf{\texttt{x.leaf}} {
	\While{\texttt{i $\geq$ 1 and k < x.keys$_i$}} {
		\texttt{x.key$_{i+1}$=k} \\
		\texttt{x.n = x.n+1} \\
	}
}
\uElse{
	\While{\texttt{i $\geq$ 1 and k < x.keys$_i$}} {
		\texttt{i=i-1}\\
	}
	\texttt{i=i+1} \\
	\uIf{\texttt{x.c$_i$.n==2t-1}} {
		\texttt{SplitChild(x,i)} \\
		\uIf{\texttt{k>x.key$_i$}} {
			\texttt{i=i+1} \\
		}
	}
	\texttt{InsertNonFull(x.c$_i$, k)}
}
\caption{InsertNonFull}
\label{algo:b-tree-insert-nonfull}	
\end{algorithm}

\caption{An example of B-tree with the minimum degree $t=2$.}
\label{fig: B-tree}
\end{figure}

\subsubsection{Insertion in a B-tree}

With a B-tree, we cannot simply create a new leaf node and insert the new key as we do with a binary search tree, because the resulting tree will fail to be a valid B-tree. Instead, we need to insert the new key into an existing leaf node. If the node is not full, we can safely insert the new key. Otherwise, we will need to split the node around the median of its keys into two new nodes and promote the median key into its parent. In this process, we need to split the parent if its parent is also full.

In the insertion, we travel down the tree and search for the position where the key should be inserted. During the traverse, we split each full node along the way. By doing so, whenever we want to split a full node, we are assured that its parent is not full. The overall algorithm is shown in Algo. \ref{algo:b-tree-insertion}, which contains methods \texttt{splitChild} and \texttt{InsertNonFull} as described in Algo. \ref{algo:b-tree-split-child} and Algo. \ref{algo:b-tree-insert-nonfull} respectively.

\begin{algorithm}[H]
\SetAlgoLined
\SetKwInOut{Input}{input}
\Input{\texttt{T}: The tree with the root \texttt{T.root}; \texttt{k}: The key to be inserted}
\KwResult{\texttt{T}: The tree with the inserted key \texttt{k}}
	\texttt{r=T.root} \\
 \uIf{\texttt{T.n==2t-1}} {
  \texttt{s = NewNode()} \\
  \texttt{T.root = s} \\
  \texttt{s.leaf = False} \\
  \texttt{s.n = 0} \\
  \texttt{s.c$_1$ = r} \\
  \texttt{SplitChild(s, 1)} \\
  \texttt{InsertNonFull(s, k)} \\
 }\uElse{
   \texttt{InsertNonFull(r, k)}
  }
 \caption{Insert}
 \label{algo:b-tree-insertion}
\end{algorithm}

In the Algo. \ref{algo:b-tree-insertion}, we first check if the root node \texttt{r} is full. If it is full, then the root splits and a new node \texttt{s} becomes the root. Then we insert the key \texttt{k} into the tree rooted at the non-full root node, i.e. \texttt{s} or \texttt{r}.

In the Algo. \ref{algo:b-tree-split-child}, the node \texttt{y} originally has $2t$ children (i.e. $2t-1$ keys) and is full. We take the following steps to split it:

\begin{enumerate}
	\item We first (from Line $1$ to Line $11$) create a new node \texttt{z} and give it the largest $t-1$ keys and the corresponding $t$ children of \texttt{y}.
	\item Then we adjust the count of keys for \texttt{y} on Line $12$: after the split, \texttt{y} will have $t-1$ keys.
	\item After that, from Line $13$ to Line $21$, we insert \texttt{z} as a child of \texttt{x}, move the median key from \texttt{y} up to \texttt{x}, and adjust the key count in \texttt{x}.
\end{enumerate}

\begin{algorithm}[H]
\SetAlgoLined
\SetKwInOut{Input}{input}
\Input{\texttt{x}: The node whose children are being split; \texttt{i}: The index of \texttt{x}'s child who is full originally}
\KwResult{\texttt{x}: The parent node whose children are not full}
\texttt{z = NewNode()} \\
\texttt{y = x.c$_i$}  \\
\texttt{z.leaf = y.leaf} \\
\texttt{z.n = t-1}\\
\For{$j\gets1$ \KwTo $t-1$}{
  \texttt{z.keys$_j$ = y.keys$_{j+t}$}\\
}
\uIf{not \texttt{y.leaf}} {
	\For{$j\gets 1$\KwTo $t$} {
		\texttt{z.c$_j$ = y.c$_{j+t}$} \\
	}
}
\texttt{y.n = t-1}\\
\For{$j\gets \texttt{x.n}$ \KwTo $i+1$} {
	\texttt{x.c$_{j+1}$ = x.c$_j$}
}
\texttt{x.c$_{i+1}$ = z} \\
\For{$j\gets \texttt{x.n}$ \KwTo $i$} {
	\texttt{x.keys$_{j+1}$=x.keys$_j$}
}
\texttt{x.key$_i$ = y.key$_t$}\\
\texttt{x.n = x.n+1}
\caption{SplitChild}
\label{algo:b-tree-split-child}	
\end{algorithm}

The Algo. \ref{algo:b-tree-insert-nonfull} works as follows:

\begin{enumerate}
	\item From Line $3$ to Line $6$, We first check if \texttt{x} is a leaf. If it is a leaf, then we insert the key \texttt{k} into \texttt{x}.
	\item If \texttt{x} is not a leaf, then we must insert \texttt{k} into the appropriate leaf node in the subtree rooted at internal node \texttt{x}. From Line $8$ to Line $11$, we traverse the subtree rooted at \texttt{x} and determine the child of \texttt{x} to which the recursion descends. Then we check on Line $12$ if the child where the recursion descends is a full node.
	\item If the child is a full node, we then split the child on Line $13$ into two non-full children. We then determine from Line $14$ to Line $15$ which of the two children is the appropriate node to insert.
	\item At the last, on Line $16$ we look into the $i$th children of \texttt{x} and recursively insert the key \texttt{k} into it.
\end{enumerate}

\begin{algorithm}[H]
\SetAlgoLined
\SetKwInOut{Input}{input}
\Input{\texttt{x}: The node to be inserted; \texttt{k}: The key to be inserted}
\KwResult{\texttt{x}: The node with the inserted key \texttt{k}}
\texttt{i=x.n} \\
\uIf{\texttt{x.leaf}} {
	\While{\texttt{i $\geq$ 1 and k < x.keys$_i$}} {
		\texttt{x.key$_{i+1}$=k} \\
		\texttt{x.n = x.n+1} \\
	}
}
\uElse{
	\While{\texttt{i $\geq$ 1 and k < x.keys$_i$}} {
		\texttt{i=i-1}\\
	}
	\texttt{i=i+1} \\
	\uIf{\texttt{x.c$_i$.n==2t-1}} {
		\texttt{SplitChild(x,i)} \\
		\uIf{\texttt{k>x.key$_i$}} {
			\texttt{i=i+1} \\
		}
	}
	\texttt{InsertNonFull(x.c$_i$, k)}
}
\caption{InsertNonFull}
\label{algo:b-tree-insert-nonfull}	
\end{algorithm}

\caption{An example of B-tree with the minimum degree $t=2$.}
\label{fig: B-tree}
\end{figure}

\subsubsection{Insertion in a B-tree}

With a B-tree, we cannot simply create a new leaf node and insert the new key as we do with a binary search tree, because the resulting tree will fail to be a valid B-tree. Instead, we need to insert the new key into an existing leaf node. If the node is not full, we can safely insert the new key. Otherwise, we will need to split the node around the median of its keys into two new nodes and promote the median key into its parent. In this process, we need to split the parent if its parent is also full.

In the insertion, we travel down the tree and search for the position where the key should be inserted. During the traverse, we split each full node along the way. By doing so, whenever we want to split a full node, we are assured that its parent is not full. The overall algorithm is shown in Algo. \ref{algo:b-tree-insertion}, which contains methods \texttt{splitChild} and \texttt{InsertNonFull} as described in Algo. \ref{algo:b-tree-split-child} and Algo. \ref{algo:b-tree-insert-nonfull} respectively.

\begin{algorithm}[H]
\SetAlgoLined
\SetKwInOut{Input}{input}
\Input{\texttt{T}: The tree with the root \texttt{T.root}; \texttt{k}: The key to be inserted}
\KwResult{\texttt{T}: The tree with the inserted key \texttt{k}}
	\texttt{r=T.root} \\
 \uIf{\texttt{T.n==2t-1}} {
  \texttt{s = NewNode()} \\
  \texttt{T.root = s} \\
  \texttt{s.leaf = False} \\
  \texttt{s.n = 0} \\
  \texttt{s.c$_1$ = r} \\
  \texttt{SplitChild(s, 1)} \\
  \texttt{InsertNonFull(s, k)} \\
 }\uElse{
   \texttt{InsertNonFull(r, k)}
  }
 \caption{Insert}
 \label{algo:b-tree-insertion}
\end{algorithm}

In the Algo. \ref{algo:b-tree-insertion}, we first check if the root node \texttt{r} is full. If it is full, then the root splits and a new node \texttt{s} becomes the root. Then we insert the key \texttt{k} into the tree rooted at the non-full root node, i.e. \texttt{s} or \texttt{r}.

In the Algo. \ref{algo:b-tree-split-child}, the node \texttt{y} originally has $2t$ children (i.e. $2t-1$ keys) and is full. We take the following steps to split it:

\begin{enumerate}
	\item We first (from Line $1$ to Line $11$) create a new node \texttt{z} and give it the largest $t-1$ keys and the corresponding $t$ children of \texttt{y}.
	\item Then we adjust the count of keys for \texttt{y} on Line $12$: after the split, \texttt{y} will have $t-1$ keys.
	\item After that, from Line $13$ to Line $21$, we insert \texttt{z} as a child of \texttt{x}, move the median key from \texttt{y} up to \texttt{x}, and adjust the key count in \texttt{x}.
\end{enumerate}

\begin{algorithm}[H]
\SetAlgoLined
\SetKwInOut{Input}{input}
\Input{\texttt{x}: The node whose children are being split; \texttt{i}: The index of \texttt{x}'s child who is full originally}
\KwResult{\texttt{x}: The parent node whose children are not full}
\texttt{z = NewNode()} \\
\texttt{y = x.c$_i$}  \\
\texttt{z.leaf = y.leaf} \\
\texttt{z.n = t-1}\\
\For{$j\gets1$ \KwTo $t-1$}{
  \texttt{z.keys$_j$ = y.keys$_{j+t}$}\\
}
\uIf{not \texttt{y.leaf}} {
	\For{$j\gets 1$\KwTo $t$} {
		\texttt{z.c$_j$ = y.c$_{j+t}$} \\
	}
}
\texttt{y.n = t-1}\\
\For{$j\gets \texttt{x.n}$ \KwTo $i+1$} {
	\texttt{x.c$_{j+1}$ = x.c$_j$}
}
\texttt{x.c$_{i+1}$ = z} \\
\For{$j\gets \texttt{x.n}$ \KwTo $i$} {
	\texttt{x.keys$_{j+1}$=x.keys$_j$}
}
\texttt{x.key$_i$ = y.key$_t$}\\
\texttt{x.n = x.n+1}
\caption{SplitChild}
\label{algo:b-tree-split-child}	
\end{algorithm}

The Algo. \ref{algo:b-tree-insert-nonfull} works as follows:

\begin{enumerate}
	\item From Line $3$ to Line $6$, We first check if \texttt{x} is a leaf. If it is a leaf, then we insert the key \texttt{k} into \texttt{x}.
	\item If \texttt{x} is not a leaf, then we must insert \texttt{k} into the appropriate leaf node in the subtree rooted at internal node \texttt{x}. From Line $8$ to Line $11$, we traverse the subtree rooted at \texttt{x} and determine the child of \texttt{x} to which the recursion descends. Then we check on Line $12$ if the child where the recursion descends is a full node.
	\item If the child is a full node, we then split the child on Line $13$ into two non-full children. We then determine from Line $14$ to Line $15$ which of the two children is the appropriate node to insert.
	\item At the last, on Line $16$ we look into the $i$th children of \texttt{x} and recursively insert the key \texttt{k} into it.
\end{enumerate}

\begin{algorithm}[H]
\SetAlgoLined
\SetKwInOut{Input}{input}
\Input{\texttt{x}: The node to be inserted; \texttt{k}: The key to be inserted}
\KwResult{\texttt{x}: The node with the inserted key \texttt{k}}
\texttt{i=x.n} \\
\uIf{\texttt{x.leaf}} {
	\While{\texttt{i $\geq$ 1 and k < x.keys$_i$}} {
		\texttt{x.key$_{i+1}$=k} \\
		\texttt{x.n = x.n+1} \\
	}
}
\uElse{
	\While{\texttt{i $\geq$ 1 and k < x.keys$_i$}} {
		\texttt{i=i-1}\\
	}
	\texttt{i=i+1} \\
	\uIf{\texttt{x.c$_i$.n==2t-1}} {
		\texttt{SplitChild(x,i)} \\
		\uIf{\texttt{k>x.key$_i$}} {
			\texttt{i=i+1} \\
		}
	}
	\texttt{InsertNonFull(x.c$_i$, k)}
}
\caption{InsertNonFull}
\label{algo:b-tree-insert-nonfull}	
\end{algorithm}

\caption{An example of B-tree with the degree $t=2$.}
\label{fig: B-tree}
\end{figure}

\subsubsection{Insertion in a B-tree}

With a B-tree, we cannot simply create a new leaf node and insert the new key as we do with a binary search tree, because the resulting tree will fail to be a valid B-tree. Instead, we need to insert the new key into an existing leaf node. If the node is not full, we can safely insert the new key. Otherwise, we will need to split the node around the median of its keys into two new nodes and promote the median key into its parent. In this process, we need to split the parent if its parent is also full.

In the insertion, we travel down the tree and search for the position where the key should be inserted. During the traverse, we split each full node along the way. By doing so, whenever we want to split a full node, we are assured that its parent is not full. The overall algorithm is shown in Algo. \ref{algo:b-tree-insertion}, which contains methods \texttt{splitChild} and \texttt{InsertNonFull} as described in Algo. \ref{algo:b-tree-split-child} and Algo. \ref{algo:b-tree-insert-nonfull} respectively.

\begin{algorithm}[H]
\SetAlgoLined
\SetKwInOut{Input}{input}
\Input{\texttt{T}: The tree with the root \texttt{T.root}; \texttt{k}: The key to be inserted}
\KwResult{\texttt{T}: The tree with the inserted key \texttt{k}}
	\texttt{r=T.root} \\
 \uIf{\texttt{T.n==2t-1}} {
  \texttt{s = NewNode()} \\
  \texttt{T.root = s} \\
  \texttt{s.leaf = False} \\
  \texttt{s.n = 0} \\
  \texttt{s.c$_1$ = r} \\
  \texttt{SplitChild(s, 1)} \\
  \texttt{InsertNonFull(s, k)} \\
 }\uElse{
   \texttt{InsertNonFull(r, k)}
  }
 \caption{B-tree Insertion}
 \label{algo:b-tree-insertion}
\end{algorithm}

In the Algo. \ref{algo:b-tree-insertion}, we first check if the root node \texttt{r} is full. If it is full, then the root splits and a new node \texttt{s} becomes the root. Then we insert the key \texttt{k} into the tree rooted at the non-full root node, i.e. \texttt{s} or \texttt{r}.

In the Algo. \ref{algo:b-tree-split-child}, the node \texttt{y} originally has $2t$ children (i.e. $2t-1$ keys) and is full. We take the following steps to split it:

\begin{enumerate}
	\item We first (from Line $1$ to Line $11$) create a new node \texttt{z} and give it the largest $t-1$ keys and the corresponding $t$ children of \texttt{y}.
	\item Then we adjust the count of keys for \texttt{y} on Line $12$: after the split, \texttt{y} will have $t-1$ keys.
	\item After that, from Line $13$ to Line $21$, we insert \texttt{z} as a child of \texttt{x}, move the median key from \texttt{y} up to \texttt{x}, and adjust the key count in \texttt{x}.
\end{enumerate}

\begin{algorithm}[H]
\SetAlgoLined
\SetKwInOut{Input}{input}
\Input{\texttt{x}: The node whose children are being split; \texttt{i}: The index of \texttt{x}'s child who is full originally}
\KwResult{\texttt{x}: The parent node whose children are not full}
\texttt{z = NewNode()} \\
\texttt{y = x.c$_i$}  \\
\texttt{z.leaf = y.leaf} \\
\texttt{z.n = t-1}\\
\For{$j\gets1$ \KwTo $t-1$}{
  \texttt{z.keys$_j$ = y.keys$_{j+t}$}\\
}
\uIf{not \texttt{y.leaf}} {
	\For{$j\gets 1$\KwTo $t$} {
		\texttt{z.c$_j$ = y.c$_{j+t}$} \\
	}
}
\texttt{y.n = t-1}\\
\For{$j\gets \texttt{x.n}$ \KwTo $i+1$} {
	\texttt{x.c$_{j+1}$ = x.c$_j$}
}
\texttt{x.c$_{i+1}$ = z} \\
\For{$j\gets \texttt{x.n}$ \KwTo $i$} {
	\texttt{x.keys$_{j+1}$=x.keys$_j$}
}
\texttt{x.key$_i$ = y.key$_t$}\\
\texttt{x.n = x.n+1}
\caption{Split a Child Node in B-Tree}
\label{algo:b-tree-split-child}	
\end{algorithm}

The Algo. \ref{algo:b-tree-insert-nonfull} works as follows:

\begin{enumerate}
	\item From Line $3$ to Line $6$, We first check if \texttt{x} is a leaf. If it is a leaf, then we insert the key \texttt{k} into \texttt{x}.
	\item If \texttt{x} is not a leaf, then we must insert \texttt{k} into the appropriate leaf node in the subtree rooted at internal node \texttt{x}. From Line $8$ to Line $11$, we traverse the subtree rooted at \texttt{x} and determine the child of \texttt{x} to which the recursion descends. Then we check on Line $12$ if the child where the recursion descends is a full node.
	\item If the child is a full node, we then split the child on Line $13$ into two non-full children. We then determine from Line $14$ to Line $15$ which of the two children is the appropriate node to insert.
	\item At the last, on Line $16$ we look into the $i$th children of \texttt{x} and recursively insert the key \texttt{k} into it.
\end{enumerate}

\begin{algorithm}[H]
\SetAlgoLined
\SetKwInOut{Input}{input}
\Input{\texttt{x}: The node to be inserted; \texttt{k}: The key to be inserted}
\KwResult{\texttt{x}: The node with the inserted key \texttt{k}}
\texttt{i=x.n} \\
\uIf{\texttt{x.leaf}} {
	\While{\texttt{i $\geq$ 1 and k < x.keys$_i$}} {
		\texttt{x.key$_{i+1}$=k} \\
		\texttt{x.n = x.n+1} \\
	}
}
\uElse{
	\While{\texttt{i $\geq$ 1 and k < x.keys$_i$}} {
		\texttt{i=i-1}\\
	}
	\texttt{i=i+1} \\
	\uIf{\texttt{x.c$_i$.n==2t-1}} {
		\texttt{SplitChild(x,i)} \\
		\uIf{\texttt{k>x.key$_i$}} {
			\texttt{i=i+1} \\
		}
	}
	\texttt{InsertNonFull(x.c$_i$, k)}
}
\caption{Insert into a Non-Full Node in B-Tree}
\label{algo:b-tree-insert-nonfull}	
\end{algorithm}
