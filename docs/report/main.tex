\documentclass[a4paper,12pt]{scrreprt}
\usepackage{scrlayer}
\DeclareNewLayer[
    foreground,
    contents={%
      \parbox[b][\layerheight][c]{\layerwidth}
        {\centering (This page intentionally left blank)}%
    }
  ]{blankpage.fg}
\DeclarePageStyleByLayers{blank}{blankpage.fg}

\usepackage{scrhack}
\usepackage[utf8]{inputenc}
\usepackage[T1]{fontenc}
\usepackage{silence}
\WarningFilter{scrreprt}{Usage of package `fancyhdr'}
\usepackage{graphicx}
\usepackage{tikz}
\usepackage{times}
\usepackage{listings}
\usepackage{amssymb}
\usepackage{amsfonts}
\usepackage{amsmath}
\usepackage[english]{babel}
\usepackage[utf8]{inputenc}
\usepackage[noend]{algpseudocode}
\usepackage[ruled,vlined,linesnumbered]{algorithm2e}
\usepackage{tabularx}
\usepackage{booktabs}
\usepackage[labelfont=bf,format=plain,justification=raggedright,singlelinecheck=false]{caption}
\usepackage{mathtools}
%\usepackage[ruled, vlined, linesnumbered, algo2e]{algorithm2e}
\DeclarePairedDelimiter\ceil{\lceil}{\rceil}
\DeclarePairedDelimiter\floor{\lfloor}{\rfloor}

\titlehead{Institut für Informatik, Universität Zürich}
\subject{\vspace*{2cm}MSc Project Report}
\title{Implementing Learned Indexes on 1 and 2 Dimensional Data}
\author{
  Neeraj Kumar, Nivedita Nivedita, Xiaozhe Yao\\[-5pt]
  %TODO: Add your matri.no and email here
  \scriptsize Matrikelnummer: 19-759-570\\[-5pt]
  \scriptsize Email: \texttt{xiaozhe.yao@uzh.ch}
}
\date{\vspace*{2cm}January 11, 2010}
\publishers{
  \small supervised by \\ 
  Prof.\ Dr.\ Michael H. Böhlen and \\ Mr. Qing\ Chen \\[5cm]
  \begin{tikzpicture}[overlay]
    \node at (-3,-3) {\includegraphics[height=1.5cm]{IFIlogo}};
    \node at (7,-3) {\includegraphics[height=1.5cm]{dbtgBW}};
  \end{tikzpicture}
}
 
\newtheorem{definition}{Definition}
\newtheorem{example}{Example}
\newtheorem{theorem}{Theorem}
\newtheorem{lemma}{Lemma}
\newcommand{\comment}[1]{}
\begin{document}

\begingroup
\let\newpage\relax%
\maketitle
\newpage\null\thispagestyle{blank}\newpage
\setcounter{page}{0}
\endgroup

\begin{abstract}
Databases use indexes to find records efficiently. Among these indexes, B-tree and KD-tree are two successful indexes used for 1-dimensional and 2-dimensional data. In this project, we implement both the learned index \cite{kraska2018case} for 1-D data and the learned index, named LISA \cite{li2020lisa} for 2-D data from scratch using Python. Afterwards, we have conducted sanity check to ensure that our implementations are correct. In addition to the implementation and evaluation, we have theoretically analyse some properties that the learned indexes hold. Beyond that, we also explore and discuss the properties that the learned indexes should hold.

As an extension to the existing learned indexes, we explore the possibilities of using convolution operation and convolutional neural network as learned indexes and report our results. 
\end{abstract}

\setcounter{tocdepth}{2}
\tableofcontents 

\chapter{Introduction}

Over the years, indexes have been widely used in databases to improve the speed of data retrieval. In the past decades, the database indexes generally fall into hand-engineered data structures and algorithms, such as B-Tree, KD-Tree, Hash Table, etc. These indexes have played an important role in databases and have been widely used in modern data management systems (DBMS). Despite their success, they do not consider the distribution of the database entries, which might be helpful in designing faster indexes.

For example, if the dataset contains integers from $1$ to $1$ million, the key can be used directly as an offset. With the key used as an offset, the values with the key can be retrieved in $\mathcal{O}(1)$ time complexity. Compared with B-Tree, which always takes $\mathcal{O}(\log n)$ time complexity for the same query. At the same time, by using the key as an offset directly, we do not need any extra overhead regarding memory space, where the B-Tree needs extra $\mathcal{O}(n)$ space complexity to save the tree.

From the above example, we found there are two promising advantages of learned indexes over hand-engineered indexes:
\begin{enumerate}
  \item Learned indexes may be faster when performing queries, especially when the number of entries in the database are extremely huge.
  \item Learned indexes may take less memory space, as we only need to save the model with constant size.
  \end{enumerate}
  
 We will explore and analyse these two advantages qualitatively in the \textit{chapter 5}. 

Nowadays, to leverage these two advantages, researchers proposed learned indexes \cite{kraska2018case}, where machine learning techniques are applied to automatically learn the distribution of the database entries and build the data-driven indexes. This approach has been shown to be powerful and competitive compared with hand-engineered indexes, such as B-Tree.

In this report, we explore the development of database indexes, from hand-engineered indexes to the learned index. After that, we explore the possibilities of using complex convolutional neural networks as database indexes. This report is organised into the following chapters:

\begin{enumerate}
	\item \textbf{Introduction}. In this chapter, we illustrate the organisation of this report. Besides, we go through the modern computer systems and introduce the general information about database indexes.
	\item \textbf{Implementation}. In this chapter, we thoroughly describe the implementation of one and two dimensional indexes, including B-Tree, baseline learned index, recursive model, KD-Tree and LISA.
	\item \textbf{Evaluation}. In this chapter, we perform evaluation among the indexes we implemented with different evaluation dataset. 
	\item \textbf{Insights and Findings}. We demonstrate our findings during the implementation in this chapter. Besides, we also discuss the advantages and disadvantages of different indexes.
	\item \textbf{Conclusions}. 
\end{enumerate}

\section{Notations}

In this report, we will use the following notations:

\begin{table}[h]
\begin{tabularx}{\textwidth}{@{}XX@{}}
\toprule
  \underline{Sets and Spaces} \\
  $\mathbb{R}$ & The set of real numbers \\
  $\mathbb{R}^d$ & The set of $d$ dimensional real space \\
  \underline{Random Variables} \\
  $\textbf{X}$ & A vector or matrix \\
  $x$ & A single value in $\textbf{X}$ \\
  $(x,y)$ & A tuple contains two values \\	
  \underline{Hyper-Parameters} \\
  N   & A pre-set hyper parameter \\
  \underline{Functions} \\
  $\mathcal{LR}$ & Linear Regression Function\\
  $\mathcal{P}$ & Polynomial Function\\
  $\mathcal{M}$ & Mapping Function\\
  $\mathcal{O}$ & Big-O notation for complexity\\
  \underline{Others} \\
  $\clubsuit$ & End of Example \\
  $\blacksquare$ & End of Proof \\
\bottomrule
\end{tabularx}
\end{table}


\section{Terminologies}

In the following chapters, we will use the following terminologies

\textbf{Index model} is a function that maps the index of a row of data into the location (e.g. page index) of the data. For example, in one-dimensional case, the index models include B-Tree, Linear Regression models, etc.

\textbf{Key} is a special attribute in the database that could identify a record. In our work, the key could be a scalar in one-dimensional case, or a $(x,y)$ pair in two-dimensional case.

\textbf{Order of a tree} is the maximum number of children that a node can have.

\textbf{Internal node} is any node of a tree that has child nodes and is not a root node.

\textbf{Leaf node} is any node that does not have child nodes.

\textbf{Level} of a node is defined as the number of edges between this node and the root node.


\section{Motivation}

In traditional database indexes, the complexity for locating an item is usually bounded by some function related to the total number of elements. For example, with a B-Tree, an item can be found within $\mathcal{O}(\log n)$ time complexity. In the meantime, saving a B-Tree as index takes $n$ space complexity. With the rapid growing of the volume of data, $n$ becomes much larger than ever before. Hence, the big data era is calling for a database index that have constant complexity in both time and space.

To achieve such a goal, the distribution of the data is important. For example, assume that the data is fixed-length records over a set of continuous integers from 1 to 100 million, the conventional B-Tree index can be replaced by the keys themselves, making the query time complexity an $\mathcal{O}(1)$ rather than $\mathcal{O}(\log n)$. Similarly, the space complexity would be reduced from $\mathcal{O}(n)$ to $\mathcal{O}(1)$. This example shows that with the knowledge of the distribution of the data, it is possible to locate the item in database in constant time.

Formally, we define the index of each record as $x$ and the corresponding location as $y$ and we represent the whole data as $(X, Y)$ pairs with the total number of pairs defined as $N$. We could then normalise the $Y$ into $\tilde{Y}\in[0,1]$ so that the $\tilde{y}$ represents the portion of the $y$ among the whole $Y$. With these definitions, we can then define a function $F:X\to \tilde{Y}$ that maps the index into the portion of the $y$. We have $y=F(x)* N$. As the output of this function can be considered as the probability of $X\leq x$, we can regard this function $F(x)$ as the cumulative distribution function (CDF) of $X$, i.e. $F(x)=\mathbb{P}(X\leq x)$. Now that $N$ is determined by the length of data records, we only need to learn such CDF and we called the learned CDF function as \textbf{learned index model}.

From the perspective of the distribution of data records, our previous example can be rephrased as following. Our data records are $(X, Y)$ pairs with a linear relation, i.e. $y=x, \forall y\in Y$. We are looking for a function $F$ such that $y=x=F(x)* N$, and hence we end up with $F(x)=\frac{1}{N}*x$. If we use this linear function $F(x)$ as the index model, then we could locate the data within $\mathcal{O}(1)$ time complexity and we only need to store the total number of records as the only parameter. Compared with B-Tree and other indexes, the advantages are enormous.

Even though there might be potential advantages, the learned index model has several assumptions, as listed below.
\begin{enumerate}
	\item All data records are stored in memory. 
	\item All data records are sorted by $X$.
	\item All data records are stored statically in database, hence we do not take insertion and deletion into consideration.
\end{enumerate}




\chapter{Implementation}

\section{One Dimensional Data}

\subsection{B-Tree}

B-Tree and its variants have been widely used as indexes in
databases. For example, the PostgreSQL uses B-Tree as its index. B-Trees
can be considered as a natural generalisation of binary search tree. In
binary search tree, there is only one key and two possible children in
the internal node. However, an internal node of B-Tree can contain
several keys and children. The keys in a node serve as dividing points
and separate the range of keys. With this structure, we make a
multi-way decision based on comparisons with the keys stored at the node
$x$. The image below illustrates a simple B-Tree.

In this section, we introduce the construction and query processes
of B-Trees and then analyse their properties.

\subsubsection{Motivation}

In computers, the memories are organised in an hierarchical way. For example, a classical computer system consists three layers of memory: the CPU cache, main memory and the hard disk. In such a system, the CPU cache is the fastest but the most expensive while and hard disk is the cheapest but also the slowest. When querying for an item, the CPU will first try to fetch it from the CPU cache. If not there the CPU will then try to fetch it from the main memory, and then the hard disk.

%TODO: Some statements are needed to show the data is stored in blocks in the memory.

At the same time, the traditional hard disk drive (HDD) is made by a moving mechanical structure.

%TODO: Illustrate the mechanical structure.
% Should we be writing this since we are using in-memory ?

In summary, there are two properties in classical computer systems that we need to take into account:

\begin{enumerate}
	\item The memory is not flat, meaning that memory references are not equally expensive. 
	\item 
\end{enumerate}

\subsubsection{Definition and Terms}
\begin{figure}[htp]
    \centering
    \includegraphics[width=0.6\textwidth]{graphs/B-Tree_definitions.png}
    \caption{B-Tree}
    \label{fig:B-Tree}
\end{figure}

Before we formally define B-Trees, we assume the following terms:

\begin{itemize}
\item
  \textbf{Keys}: The key in a database is a special attribute that could identify a row in the database. In our work, each key corresponds to a 1-dimensional
  \textbf{value} and forms a key-value pair.
\item
  \textbf{Order}: The Order of a B-Tree is the maximum number of children that a node can have. Number of keys in a node is always one less than the order of the tree at the maximum.
\item
  \textbf{Internal Node}: An internal node is any node of the tree that has child nodes and is not a root node.
\item
  \textbf{Leaf Node}: A leaf node is any node that does not have child nodes.
\end{itemize}

Each node in a B-Tree has the following attributes:

\begin{itemize}
\item
  $x.n$ is the number of keys currently stored in the node $x$.
\item
  Inside each node, the keys are sorted in non decreasing order, so that
  we have $x.keys_1\leq x.keys_2\leq\cdots\leq x.keys_{x.n}$.
\item
  $x.leaf$, a Boolean value determines if current node is a leaf node.
\end{itemize}

With these properties, A B-Tree $T$ whose root is $T.root$ have the
following properties:

\begin{itemize}
\item
  Each internal node $x$ contains $x.n+1$ children. We assume the
  children are $x.c_1,\cdots,x.c_{x.n+1}$.
\item
  The nodes in the tree have lower and upper bounds on the number of
  keys that can contain. These bounds can be expressed in terms of a
  fixed integer $t$.
\end{itemize}

\subsubsection{Insertion of B-Tree}

When inserting keys into a binary search tree, we search for the leaf
position at which to insert the new key. However, with B-Tree, we cannot
simply find the position, create a new node and insert the value because
the tree will be imbalanced again. Hence, in this section, we illustrate
an operation that splits a full node around its median key

\begin{algorithm}[H]
    \SetAlgoLined
    \SetKwInOut{Input}{Input}
     \Input{\texttt{$m$:order\_of\_tree ,$(k,v)$:(key, value), $N$:Node;}}
    \SetKwInOut{Output}{Output}
     \Output{\texttt{B-Tree}}
     \eIf{$N$ is a leaf and not yet full}  
     {
        \texttt{insert $(k,v)$ into $N$}
     }
     {
        \texttt{create new Node N'}\\
        \texttt{Find the median of the node}\\
        \texttt{Add the value at the median location to the new node N'}\\
     }
    \If{$N$ is root with no children and not full}
    {
    
        \texttt{insert $(k,v)$ into $N$}
        
    }
     \caption{Algorithm for B-Tree insertion}
     \label{B-Tree Insertion}
\end{algorithm}


\begin{figure}[htp]
    \centering
    \includegraphics[width=0.6\textwidth]{graphs/B-Tree_example01.png}
    \caption{B-Tree key insertion}
    \label{fig:B-Tree key insertion1}
\end{figure}

\begin{figure}[htp]
    \centering
    \includegraphics[width=0.3\textwidth]{graphs/B-Tree_example02.png}
    \caption{B-Tree key insertion}
    \label{fig:B-Tree key insertion2}
\end{figure}

\subsubsection{Insertion in a B-Tree}

There are two conditions in insertion:
\begin{itemize}
\item
\textbf{When the node is empty or not created at all}:
In the algorithm, initially when the first key is inserted and there is no root to the tree it will check the condition if there are any nodes and if not create a new one. It will insert the new key in the node and keeps inserting the new keys until it is one less than the order of the B-Tree being created. 
\item
\textbf{When the node is full}:
As soon as the maximum number of allowed children has reached for the root node a new empty node is created. Suppose we have inserted [1,11,21] to a node of a B-Tree with degree 4. Now if we want to insert a new value 31 into the tree, since it has reached the maximum number of children it will find the median of the existing node [1,11,21] which is 11 and increase the level of the B-tree to 2 and make [1] and [21] child nodes of [11]. So now if a new value is to be inserted it can be inserted. Splitting of the node happens each time it reaches it's maximum allowed keys. Now 31 will be compared with 11 and since 31 > 11 it will be inserted in the right child and it will have an updated value of [21,31]. More keys can then be inserted until it reaches it's maximum and splits again. Once the node is split it's parent is also updated to the median value i.e., [11] in this case for nodes [1] and [21,31,41] as can be seen in \ref{fig:B-Tree key insertion2}.
\end{itemize}

\subsubsection{Search in a B-Tree}

\begin{algorithm}[H]
    \SetAlgoLined
    \SetKwInOut{Input}{Input}
     \Input{\texttt{k ; key, root ; Root of the B-Tree}}
    \SetKwInOut{Output}{Output}
     \Output{\texttt{Value associated with key}}
     \For{$i\gets0$ \KwTo $len(Root Node)$}{
         \If{keys in root greater than k}
         {
            \texttt{SEARCH\_CHILD(key)} //Search linearly the child associated with the key location one before\\
            \eIf{Child is a leaf}
            {
                \texttt{Linearly search until the key is reached}
            }
            {
                \texttt{SEARCH\_CHILD(key)}
            }
            
         }
    }
     \caption{Algorithm for B-Tree Search}
     \label{B-Tree Search}
\end{algorithm}

Search in a B-Tree is basically the comparison of the value of the key that needs to be searched with the keys in the node. It first linearly checks the value in the root key and looks for a key value which is greater than the searched key. As soon as it finds a key greater than the searched key it will search in the child of the key before it. For example if we were to search of key [41] in the example of the B-Tree used above, we would first check the value [11] and since it smaller than [31] and there are no keys greater than this in the root node it will search in the right child of the root node. It will then linearly search the node and locate it's associated value and return it. 

\subsection{Baseline Learned Index}

\subsubsection{Overview}

The B-Tree can be regarded as a function $\mathcal{F}$ that maps the key $x$ into its corresponding page index $y$. It is known to us that the pages are allocated in a way that the every $S$ entries are allocated in a page where $S$ is a pre-defined parameter. For example, if we set $S$ to be 10 items per page, then the first page will contain the first 10 keys and their corresponding values. Similarly, the second 10 keys and their corresponding values will be allocated to the second page.

If we know the CDF of $X$ as $F(X\leq x)$ and the total number of entries $N$, then the position of $x$ can be estimated as $p=F(x)*N$ and the page index where it should be allocated to is given by

$$y=\floor{\frac{p}{S}}=\floor{\frac{F(x)*N}{S}}$$  

For example, if the keys are uniformly distributed from $0$ to $1000$, i.e. the CDF of $X$ is defined as $F(X\leq x)=\frac{x}{1000}$ and we set $S=10, N=1001$. Then for any key $x$, we immediately know it will be allocated into $y=\floor{\frac{1000}{10}*\frac{x}{1000}}=\floor{\frac{x}{10}}$. Assume that we have a key $698$, then we can calculate $y=\floor{\frac{698}{10}}=69$. By doing so, the page index is calculated in constant time and space.

In this example, we see that the distribution of $X$ is essential and our goal of learned index in one-dimensional data is to learn such distribution. To do so, we apply two different techniques as the baseline, the polynomial regression and fully connected neural network.

To train such a learned index, we first manually generate the $X$ with respect to a certain distribution. We then save the generated $X$ into a dense array with the length $N$. Then we use the proportional index, i.e. the index of each $x$ divided by $N$ as the expected output $y$.

\subsubsection{Polynomial Regression}
 
The polynomial regression model with degree $m$ can be formalised as 

$$ \hat{y_i}= \beta_0+\beta_1x_i+\beta_2x_i^2+\cdots+\beta_mx_i^m$$ and it can be expressed in a matrix form as below

$$
\begin{bmatrix}
y_1 \\ y_2\\ \vdots \\ y_n 
\end{bmatrix}=\begin{bmatrix}
1 & x_1 & x_1^2 &\cdots & x_1^m \\ 
1 & x_2 & x_2^2 &\cdots & x_2^m \\ 
\vdots \\ 
1 & x_n & x_n^2 &\cdots & x_n^m \\ 
\end{bmatrix}\begin{bmatrix}
\beta_0 \\ \beta_1 \\ \vdots \\ \beta_m 
\end{bmatrix}
$$ which can be written as $Y=\boldsymbol{X}\boldsymbol{\beta}$. 
 
 Our goal is to find $\beta$ such that the sum of squared error, i.e. $\text{S}(\boldsymbol{\beta})=\sum_{i=1}^n(\hat{y}-y)^2$ is minimal. This optimisation problem can be resolved by ordinary least square estimation as shown below.
 
 First we have the error as
 
 \begin{equation}
 \begin{split}
 \text{S}(\boldsymbol{\beta})=||\boldsymbol{y}-\boldsymbol{X} \boldsymbol{\beta}||& =(\boldsymbol{y}-\boldsymbol{X}\boldsymbol{\beta})^T(\boldsymbol{y}-\boldsymbol{X}\boldsymbol{\beta})\\
 	& =\boldsymbol{y}^T\boldsymbol{y}-\boldsymbol{\beta}^T\boldsymbol{X}^T\boldsymbol{y}-\boldsymbol{y}^T\boldsymbol{X}\boldsymbol{\beta}+\boldsymbol{\beta}^T\boldsymbol{X}^T\boldsymbol{X}\boldsymbol{\beta}
\end{split}
 \end{equation}
 
 Here we know that $(\boldsymbol{\beta}^T\boldsymbol{X}^T\boldsymbol{y})^T=\boldsymbol{y}^T\boldsymbol{X}\boldsymbol{\beta}$ is a $1\times 1$ matrix, i.e. a scalar. Hence it is equal to its own transpose. As a result we could simplify the error as
 
 \begin{equation}
 	\begin{split}
 		\text{S}(\boldsymbol{\beta})=\boldsymbol{y}^T\boldsymbol{y}-2\boldsymbol{\beta}^T\boldsymbol{X}^T\boldsymbol{y}+\boldsymbol{\beta}^T\boldsymbol{X}^T\boldsymbol{X}\boldsymbol{\beta}
 	\end{split}
 \end{equation}
 
 In order to find the minimum of $S(\boldsymbol{\beta})$, we differentiate it with respect to $\boldsymbol{\beta}$ as 
 
 \begin{equation}
 	\nabla_{\boldsymbol{\beta}}S=-2\boldsymbol{X}^T\boldsymbol{y}+2(\boldsymbol{X}^T\boldsymbol{X})\boldsymbol{\beta}
 \end{equation}
 
 By let it to be zero, we end up with 
 
 \begin{equation}
 \begin{split}
 	 &	-\boldsymbol{X}^T\boldsymbol{y}+(\boldsymbol{X}^T\boldsymbol{X})\boldsymbol{\beta}=0 \\
 	& \implies \boldsymbol{\beta}= (\boldsymbol{X}^T\boldsymbol{X})^{-1}\boldsymbol{X}^T\boldsymbol{y}
 \end{split}
 \end{equation}
 
\subsubsection{Fully Connected Neural Network}

\subsection{Recursive Model Index}

%TODO: there should be some graph to demonstrate the last-mile problem.

In our baseline models, it is not very difficult to reduce the mean square error from millions to thousands. However, it is much harder to reduce it from thousands to tens. This is the so called last-mile problem.

In order to solve this problem, recursive model index was proposed \cite{kraska2018case}. The idea is to split the whole set of data into smaller pieces and assign each piece an index model. By doing so, each model is only responsible for a small range of keys. Ideally, in each smaller range, the keys are distributed in a way that is easier to be learned by our index models, such as polynomial model, fully connected model or even traditional B-Tree model.

As shown in Fig. \ref{rmi_structure}. A recursive model can be regarded as a tree structure, which contains a root model that receives the full dataset for training. Then the root model will split the dataset into several parts. Each sub-model will then receive one part of the full dataset. Then we train the sub-models one by one with the partial training dataset. 

\begin{mscexample}
	For example, in the Fig. \ref{rmi_structure}, the full dataset will be split into three parts and each sub-model receives one part. To train this recursive model, we first train the root model with the whole dataset. Then the root model will split the dataset into 3 parts according to the predicted value of each data point in the dataset. Then each sub-model will receive one part and we train the sub-model accordingly.
\end{mscexample}

\begin{figure*}[h]
\centering
\includegraphics[scale=0.4]{graphs/implementation/one-dim/rmi_demo.pdf}
\caption{An example recursive model index with one root model and three leaf model.}
\label{rmi_structure}
\end{figure*}

\subsubsection{Properties}

Similar to a tree, we define the following terms in a recursive model:

\begin{enumerate}
	\item \textbf{Node Model}. Every node is responsible for making decisions with given input data. In one dimensional case, it can be regarded as a function $f:\mathbb{R}\to\mathbb{R}, x\to y$ where $x$ is the input index and $y$ is the corresponding page block. In principle, each node can be implemented as any machine learning model, from linear regression to neural network, or a traditional tree-based model, such as B-Tree.
	\item \textbf{Internal Node Model}. Internal nodes are all nodes except for leaf nodes and the root node. Every internal node receives a certain part of training data from the full dataset, and train a model on it. 
\end{enumerate}

In the following sections, we will use the notations defined below:
\begin{enumerate}
	\item $N_M^{(i)}$ is the number of models in the $i$th stage.
	%TODO: more notations
	%TODO: modify algorithms accordingly
\end{enumerate}


\subsubsection{Training}

In order to construct a recursive model, we need to have several parameters listed below:
\begin{enumerate}
	\item The training dataset, notated as $(X, Y)$ with entries notated as $(x,y)$.
	\item The number of stages, notated as $N_S$. It is an integer variable.
	\item The number of models at each stage, notated as $N_M$. It is a list of integer variable. $N_M^{(i+1)}$ represents the number of models in the $i$th stage.
\end{enumerate}

The training process of recursive model is an up-bottom process. There will be only one root model that receives the whole training data. After the root model is trained, we iterate over all the training data and predict the page by the root model. After the iteration, we get a new set of pairs $(X, Y_0)$. Then we map $\forall y_0\in Y_0$ into the selected model id in next stage by $\texttt{next}=y_0 * N_M^{(i+1)}/\texttt{max(Y)}$.

% REVISE THIS
\begin{algorithm}[H]
    \SetAlgoLined
    \SetKwInOut{Input}{input}
    \Input{\texttt{$N_S$: A scalar representing the number of stages; \\ $N_M$: An array representing the number of models at each stage; \\ x; y}}
     \texttt{trainset=[[(x,y)]]} \\
     \texttt{stage$\gets 0$} \\
     \While{\texttt{stage} \textless \texttt{$N_S$}}{
      \While{\texttt{model} \textless \texttt{$N_M$[stage]}} {
        \texttt{model.train(trainset[stage][model])} \\
        \texttt{models[stage].append(model)}
      }
      \uIf{\texttt{stage<$N_S$-1}} {
        \For{\texttt{$i\gets0$ \KwTo $len(x)$}}{
        	\texttt{next\_model = 0}\\
        	\For{\texttt{$j\gets 0$ \KwTo stage-1}} {
        		\texttt{output = models[stage][next\_model]} \\
        		\texttt{next\_model = output * $N_M$[stage+1]/max\_y}\\
        	}
            	\texttt{model = models[next\_model]} \\
            	\texttt{output = model.predict(x[i])} \\
            	\texttt{next = output * $N_M$[stage+1]/max\_y} \\
            	\texttt{trainset[stage+1][next].add((x[i],y[i]))}
        }
      }
     \texttt{stage=stage+1}
     }
     \caption{Training of Recursive Model Index}
\end{algorithm}

\subsubsection{Polynomial Internal Models}

In the recursive model index, we use internal models to learn the CDF of a part of the full training data. In order to learn the CDF, we need to know or assume the distribution of a specific part of the data. In this report, we support the following distributions.

\begin{table}[h]
  \begin{tabularx}{\textwidth}{@{}XX@{}}
  \toprule
    Linear Regression & $wx+b$ \\
    Quadratic Regression & $ax^2+bx+c$ \\
    B-Tree & N/A \\
    Fully Connected Neural Network & N/A \\
  \bottomrule
  \end{tabularx}
  \end{table}

Here we describe how we fit a polynomial model.

The polynomial regression model with degree $m$ can be formalised as 

$$ \hat{y_i}= \beta_0+\beta_1x_i+\beta_2x_i^2+\cdots+\beta_mx_i^m$$ and it can be expressed in a matrix form as below

$$
\begin{bmatrix}
y_1 \\ y_2\\ \vdots \\ y_n 
\end{bmatrix}=\begin{bmatrix}
1 & x_1 & x_1^2 &\cdots & x_1^m \\ 
1 & x_2 & x_2^2 &\cdots & x_2^m \\ 
\vdots \\ 
1 & x_n & x_n^2 &\cdots & x_n^m \\ 
\end{bmatrix}\begin{bmatrix}
\beta_0 \\ \beta_1 \\ \vdots \\ \beta_m 
\end{bmatrix}
$$ which can be written as $Y=\boldsymbol{X}\boldsymbol{\beta}$. 
 
 \begin{mscproof}
 	 Our goal is to find $\beta$ such that the sum of squared error, i.e. $$\text{S}(\boldsymbol{\beta})=\sum_{i=1}^n(\hat{y}-y)^2$$ is minimal. This optimisation problem can be resolved by ordinary least square estimation as shown below.
 
 First we have the error as
 
 \begin{equation}
 \begin{split}
 \text{S}(\boldsymbol{\beta})=||\boldsymbol{y}-\boldsymbol{X} \boldsymbol{\beta}||& =(\boldsymbol{y}-\boldsymbol{X}\boldsymbol{\beta})^T(\boldsymbol{y}-\boldsymbol{X}\boldsymbol{\beta})\\
 	& =\boldsymbol{y}^T\boldsymbol{y}-\boldsymbol{\beta}^T\boldsymbol{X}^T\boldsymbol{y}-\boldsymbol{y}^T\boldsymbol{X}\boldsymbol{\beta}+\boldsymbol{\beta}^T\boldsymbol{X}^T\boldsymbol{X}\boldsymbol{\beta}
\end{split}
 \end{equation}
 
 Here we know that $(\boldsymbol{\beta}^T\boldsymbol{X}^T\boldsymbol{y})^T=\boldsymbol{y}^T\boldsymbol{X}\boldsymbol{\beta}$ is a $1\times 1$ matrix, i.e. a scalar. Hence it is equal to its own transpose. As a result we could simplify the error as
 
 \begin{equation}
 	\begin{split}
 		\text{S}(\boldsymbol{\beta})=\boldsymbol{y}^T\boldsymbol{y}-2\boldsymbol{\beta}^T\boldsymbol{X}^T\boldsymbol{y}+\boldsymbol{\beta}^T\boldsymbol{X}^T\boldsymbol{X}\boldsymbol{\beta}
 	\end{split}
 \end{equation}
 
 In order to find the minimum of $S(\boldsymbol{\beta})$, we differentiate it with respect to $\boldsymbol{\beta}$ as 
 
 \begin{equation}
 	\nabla_{\boldsymbol{\beta}}S=-2\boldsymbol{X}^T\boldsymbol{y}+2(\boldsymbol{X}^T\boldsymbol{X})\boldsymbol{\beta}
 \end{equation}
 
 By let it to be zero, we end up with 
 
 \begin{equation}
 \begin{split}
 	 &	-\boldsymbol{X}^T\boldsymbol{y}+(\boldsymbol{X}^T\boldsymbol{X})\boldsymbol{\beta}=0 \\
 	& \implies \boldsymbol{\beta}= (\boldsymbol{X}^T\boldsymbol{X})^{-1}\boldsymbol{X}^T\boldsymbol{y}
 \end{split}
 \end{equation}
 
 \end{mscproof}

\section{Two Dimensional Data}

\subsection{KD-Tree}

Similar to search with binary search tree, we also need to traverse the tree in order to perform point query. However, we need to switch the dimensions when we compare the values between the query key and the values in the nodes.

\begin{algorithm}[H]
    \SetAlgoLined
    \SetKwInOut{Input}{Input}
    \SetKwInOut{Output}{Output}
    \Input{\texttt{t}: The node being searched; \texttt{x}: The query key; \texttt{cd}: Current dimension}
    \Output{\texttt{n}: the node that contains the query key}
    \texttt{DIM=2}\\
    \uIf{\texttt{t==NULL}} {
    	\Return \texttt{NULL}
    }
    \uIf{\texttt{x[0]==t.data[0]} and \texttt{x[1]==t.data[1]}} {
    	\Return \texttt{t}
    }
    \uElseIf{\texttt{x[cd]<t.data}} {
    	\Return \texttt{pointSearch(t.left, x, (cd+1) \% DIM)}
    }
    \uElseIf{\texttt{x[cd]>t.data}} {
    	\Return \texttt{pointSearch(t.right, x, (cd+1) \% DIM)}
    }
    \caption{Point Query with $K$D-Tree}
    \label{algo: point_query_kdtree}
\end{algorithm}

The point query works in the following steps:

\begin{enumerate}
	\item From Line $2$ to $3$, we first check if current node is \texttt{NULL}. If so, that means that we have already traversed all the possible nodes and found nothing. In this case, the query returns \texttt{NULL}.
	\item From Line $4$ to $5$, we check if the current node contains the same key as the query key. If so, the current node is the node that we are looking for. Hence, we return the current node in this case.
	\item Otherwise, from Line $6$ to $9$, we check if the current dimension of the query key is smaller, larger or equal to the current dimension of the data in the node.
	\begin{enumerate}
	\item If it is smaller, then we search on the left subtree of current node, with the same query key and switched dimension.
	\item If it is larger, then we search on the right subtree of current node, with the same query key and switched dimension.
	\end{enumerate}
\end{enumerate}

\begin{mscexample}    
	In the previous figure \ref{fig:kd_tree_example}, we showed an example $K$D-tree. If we want to search for $(50, 30)$ in this tree, we would follow the following steps:
	\begin{enumerate}
		\item We first check the root node and compares the $x$-coordinate. As $50>30$, we go to the right subtree of the root node.
		\item Then in the subtree, we compare the $y$-coordinate. As $50<70$, we go to the left subtree of this node.
		\item Then in the left subtree, the termination condition is reached, hence we return this node as result.
	\end{enumerate}
\end{mscexample}


\subsection{LISA: Learned Index for Spatial Data}

It is difficult to apply traditional $K$NN query pruning strategies applicable for $K$D-Trees, to LISA model as it doesn't maintain a tree like structure with all nodes and
entries based on MBRs (minimum bounding rectangle) and parent-children relationships. %Shard boundaries are learned per mapped interval and no data structure is maintained to refer to shards in adjacent mapped intervals. 
The key idea in LISA paper $K$NN query implementation is to convert it into a range query by estimating an appropriate query range. LISA paper suggests a learning model to learn an appropriate distance bound from underlying training data for every query point and specific value of K. However, we have used empirically estimates to learn this distance bound for different values of $K$. This distance bound is used to convert the $K$NN query to range query.The query range is augmented if less than K neighbors are found in a range query. 

Consider a query point $q_{knn}=(x_{0},x_{1})$, let $x^{'} \in V$ be the $K$th nearest key to $x$ in database at a distance value $\delta = \| x^{'}-q_{knn}\|_{2} $. Lets define $ \mathcal{Q}(q_{knn},\delta) \triangleq [x_{0}-\delta, x_{0}+\delta) \times[x_{1}-\delta, x_{1}+\delta)$ and $\mathcal{B}(q_{knn}, \delta)  \triangleq \{p \in V \mid \| q_{knn}-p\|_{2} \leq \delta \} $. We can create a query rectangle $qr =  \mathcal{Q}(q_{knn}, \delta + \epsilon)$ where $\epsilon \rightarrow 0$. As shown in Fig. \ref{fig:KNN_Query_Lisa}, K nearest keys to $q_{knn}$ are all in $\mathcal{B}(q_{knn}, \delta)$ and thus in $\mathcal{Q}$. $K$NN query can be solved using the range query if we can estimate an appropriate distance bound $\delta$ for every query point.

\begin{figure*}[t]
    \centering
    \includegraphics[width=0.7\textwidth]{graphs/KNN_Query_Lisa.png}
    \caption{KNN Query Implementation in Lisa(K=3)\\
    1)$q_{knn}$ represents the query point, $ \mathcal{Q}(x,\delta) \triangleq [x_{0}-\delta, x_{0}+\delta) \times[x_{1}-\delta, x_{1}+\delta)$, represents query rectangle and $ \mathcal{B}(x, \delta)$ represents the key space at distance $\delta$ containing K nearest keys.\\
    2)KNN query can be solved by range query if we can estimate an appropriate distance bound $\delta$ for every query point\\
    }
    \label{fig:KNN_Query_Lisa}
\end{figure*}
In our experiments, we find the $\delta$ empirically. We try with different values of $\delta$ and choose the one for which we get the best results. 

\chapter{Evaluation}

\chapter{Insights and Findings}

\chapter{Conclusion}

\bibliographystyle{alpha}
\bibliography{refs}

\end{document}
