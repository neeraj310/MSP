Spatial data and query processing have become ubiquitous due to proliferation of location-based services such as digital mapping, location-based social networking,
and geo-targeted advertising. Motivated by the performance benefits of learned indices
for one-dimensional data, this section explores the application of learned index for spatial data. The main motivation is to use machine learning models through several steps and generate a learned index for spatial data to reduce the storage consumption and IO cost compared to existing indexes such as R-Tree.

\subsubsection{Motivation}
In the last section, we described a recursive model index (RMI) that consists of a
number of machine learning models staged into a hierarchy to enable synthesis of specialised index structures, termed learned indexes. Provided with a search key x, RMI predicts the position of x's data with some error bound, by learning the cumulative distribution function (CDF) over the key search space. However, the idea of RMI is not applicable in the context of spatial data as spatial data invalidates the assumption required by RMI that the data is sorted by key and that any imprecision can be easily corrected by a localised search. Although it is possible to learn multi-dimensional CDFs, such CDFs will result in searching local regions qualified on one dimension but not all dimensions.

For example, consider the joint cumulative function of two random variables X and Y defined as $F_{XY}(x, y)=P(X\leq x, Y\leq y)$.

The joint CDF satisfies the following properties:

\begin{itemize}
  \item  {$F_X(x)=F_{XY}(x,\infty)$, for any x (marginal CDF of X)}
  \item  {$F_Y(y)=F_{XY}(\infty,y)$, for any y (marginal CDF of Y)}
  \item  {if $X$ and $Y$ are independent, then $F_{XY}(x,y)=F_X(x)F_Y(y)$}
\end{itemize}

\textcolor{blue} {Need to find a solid argument to explain why learning multi dimensional CDFs will result in  searching local regions qualified on one dimension. }

LISA solves this problem by partitioning search space into a series of grid cells based on the data distribution and building a partially monotonic function according to the borders of cells to map the data from $\mathbb{R}^d$ into $\mathbb{R}$.

\subsubsection{Baseline Method}  

We can extend the learned index method for range queries on spatial data by using a mapping function. This baseline method works as follows. We first sort all keys according to their mapped values and divide the mapped values into equal number of cells. If a point
(x,y)’s mapped value is larger than those of the keys stored
in the first i cells, i.e. $M(x,y) > \sup \bigcup\limits_{j=0}^{i-1} M(C_{j})$, we store (x,y) in cell i. Subsequently, for a query rectangle
qr = $[l_{0},u_{0}) \times [l_{1},u_{1})$, we only need to predict $i_{1}$ and $i_{2}$, the indices of $(l_{0}, l_{1})$ and $(u_{0},u_{1})$, respectively, scan the keys in $i_{2}$ - $i_{1}$ + 1 cells, and those keys that fall in the query rectangle qr . 

\subsubsection{Definitions}

This section presents the definition

\begin{enumerate}
	\item \textbf{Key}. A key k is a unique identifier for a data record with $k = (x_{0}, x_{1}) \in \mathbb{R}^{2}$. 
    %\textcolor{blue} {Need to check how to write d-1 in subscript }
    
	\item \textbf{Cell}. A grid cell is a rectangle whose lower and upper corners are points $(l_{0},l_{1}) and  (u_{0},u_{1})$, i.e.,  cell = $(l_{0},u_{0}) \times [l_{1},u_{1})$
	
	\item \textbf{Mapping Function}. A mapping function M is a partially monotonic function on the domain $\mathbb{R}^{2}$ to the non-negative range,i.e $M:[0,X_{0}]\times[0,X_{1}]\to [0,+\infty)$ such that $M(x_{0},x_{1}) \leq M(y_{0},y_{1})$ when $x_{0} \leq y_{0}$ and $x_{1} \leq y_{1} $
	\item \textbf{Shard}.shard S is the preimage of an
interval $[a, b) \subseteq [0, +1)$ under the mapping function M,  i.e., $S = M^{-1}([a.b))$.
    
	\item \textbf{Local Model}. local model $L_{i}$ is a model
that processes operations within a shard $S_i$ . It keeps dynamic
structures such as the addresses of pages contained by $S_{i}$ .
\end{enumerate}
% $f:\mathbb{R}\to\mathbb{R}, x\to y$ where $x$ is the input index and $y$ is the corresponding page block


\subsubsection{Training}

In order to construct the baseline model, we need to have several parameters listed below:
\begin{enumerate}
	\item The training dataset, notated as $(X, Y)$ with entries notated as $(x,y)$. $X$ represents the two dimensional key values, and $Y$ represents the corresponding data item value. 
	\item This number represents the number of cells(pages) into which the key space mapped values will be divided. It is an integer variable. Pages or cells will be used interchangeably in the next section to represents the subset of keys in a particular cell. 
\end{enumerate}

During training, sort all keys according to their mapped values, divide the keys into equal sized cells(pages), and store the mapped values of first and last key for each page into an array. For prediction, find the page corresponding to mapped value of query point, and scan this page sequentially. 

\begin{algorithm}[H]
    \SetAlgoLined
    \SetKwInOut{Input}{input}
    \SetKwInOut{Output}{Output}
    \Input{num\_of\_cells; trainset=[$(x,y);x \in \mathbb{R}^{2};y \in \mathbb{R}$]}
    \Output{M:Mapped Function}
     \For{$i\gets0$ \KwTo $len(x)$}{
            	\texttt{x[i].mapped\_value = x[i][0]+x[i][1]} \\
            
     }
     \texttt{Sort x based on x.mapped\_value}\\
     \texttt{Divide x into equal size pages according to num\_of\_cells}\\
     \texttt{Store mapped value of first and last key for each page }\\
      \For{$i\gets0$ \KwTo num\_of\_cells}
     {
         \texttt{denseArray[i].lower = first key in page i  } \\
		 \texttt{denseArray[i].upper = last key in page i  }
		
     }
     \caption{Training Algorithm for Lisa Baseline Method}
     \label{Training_Lisa_Baseline}
\end{algorithm}

\subsubsection{Prediction}

\begin{algorithm}[H]
    \SetAlgoLined
    \SetKwInOut{Input}{input}
    \Input{\texttt{x\_test}: Key; \texttt{d:}the denseArray}
    \texttt{x\_test.mapped\_value = x\_test[0]+x\_test[1] } \\
    \For{$i\gets0$ \KwTo $len(denseArray)$}
    {
        \If{$\mathcal{M}(x\_test) \in [d[i].lower,d[i].upper]$} 
        {
		    \texttt{Key is in Page i } \\
		    \texttt{break }
		}
    }
  
 	 \texttt{Sequentially search for x\_test in page j} \\
     \caption{Prediction Algorithm for Lisa Baseline Model }
\end{algorithm}

\subsubsection{Limitation}

Prediction cost in baseline method consists of following two parts.

\begin{enumerate}
	\item Search cost for the page which contains the key. If page size is small, this cost will be high as keys will be divided into larger number of pages
	\item Scan cost of all the keys in a particular page to match the query point. If page size is large, number of pages will be smaller, number of keys per page will be higher, resulting in higher cost of sequential scan with in the page. 
\end{enumerate}
Consider the example in figure \ref{fig:BaseLine_Method_Limitation}. Dataset is divided into 3 sections based on the mapped values. Any point or range query in the second triangle(page) will result into a sequential scan through all 14 keys in the triangle.   

\begin{figure*}[t]
    \centering
    \includegraphics[width=0.9\textwidth]{graphs/Lisa_Baseline_Model_Limitation.png}
    \caption{Baseline Method Limitation }
    \label{fig:BaseLine_Method_Limitation}
\end{figure*}

\subsubsection{Lisa Overview}

Given a spatial dataset, we generate the mapping function
M, the shard prediction function SP and a series of local
models. Based on them, we build our index structure, LISA, to
process range query,KNN query and data updates. LISA consists of four parts: the representation of grid cells, the mapping function M, the shard prediction function SP, and the local models for all shards. As illustrated in the figure. the procedure of building LISA is composed of four parts.

\begin{figure*}[t]
    \centering
    \includegraphics[width=17cm,height=4cm]{graphs/Lisa_Overview.png}
    \caption{Lisa Framework }
    \label{fig:Lisa_Framework}
\end{figure*}

As seen in the overview of LISA, there are four stages in building the LISA. 
\begin{enumerate}
	\item Grid cell partition.
	\item Mapping spatial coordinates into scalars, i.e. $\mathbb{R}^d\to\mathbb{R}$.
	\item Build shard prediction function $\mathcal{SP}$.
	\item Build local models.
\end{enumerate}

\textbf{Grid Cells Generation} 
First task in Lisa implementation is to partition the 2 dimensional key space into a series of grid cells based on the data distribution along a sequence of axes and numbering the cells also along these axes. The principal idea behind this partition strategy is to divide the key space into cell boundaries and apply a mapping function to create monotonically increasing mapping values at the cell boundaries, i.e. mapped value of all keys in cell i will be less than mapped values of keys in cell j, if i <j. Consider the example shown in the figure \ref{fig:Cell_Parttion} total number of keys are 18, and we decided to partition the key space into 9 cells, resulting in 2 keys per cell. To partition the key space, we first sort the keys values according to $1^{st}$  dimension, divide the keys into 3 cells each containing 6 keys. Then for each cell, we sort the keys again according to $2^{nd}$  dimension, and divide the keys in each cell into 3 new cells. 


\begin{figure*}[t]
    \centering
    \includegraphics[width=0.8\textwidth]{graphs/Cell_partition.png}
    \caption{Cell Partition Strategy }
    \label{fig:Cell_Parttion}
\end{figure*}


\begin{algorithm}[H]
    \SetAlgoLined
    \SetKwInOut{Input}{input}
     \Input{\texttt{num\_of\_cells;x; y}}
     \texttt{trainset=[$(x,y);x \in \mathbb{R}^{2};y \in \mathbb{R}$]} \\
     \texttt{$keysPerPage =  len(x) / num\_of\_cells $}\\
     \texttt{Sort x based on first dimension x[:][0]]}\\
     \texttt{In first for loop, divide the keys into equal size subsets based on first dimension }\\
     \For{$i\gets0$ \KwTo $\sqrt(num\_of\_cells)$}  
     {
        \texttt{Store the 1st dimensional coordinates of first and last key for each cell. Each such cell will contain $keysPerPage*sqrt(num\_of\_cells)$ keys } \\
     }
     
     \texttt{Sort keys in each cell based on 2nd dimension,x[:][1] }\\
     
      \For{$i\gets0$ \KwTo $\sqrt{num\_of\_cells}$}
      {
         \For{$j\gets0$ \KwTo $\sqrt(num\_of\_cells)$}
         {
            \texttt{Store the 2nd dimensional coordinates of first and last key for each cell.} \\
		 }
      }
     \caption{Grid Cell Generation Algorithm for Lisa Method}
     \label{Training_Lisa_Baseline}
\end{algorithm}

\subsubsection{Mapping Function}

\subsubsection{Shard Prediction Function}

After the mapping function, we get a dense array of mapped values. Then we partition them evenly into $U$ parts and let $\boldsymbol{M}_p=[m_1,\cdots, m_U]$. We train linear regression functions $\mathcal{F}_i$ on each interval and suppose $V+1$ is the number of mapped values that each $\mathcal{F}_i$ needs to process and $\Psi$ is the average number of keys falling in a shard. With these definitions, we know that each $\mathcal{F}_i$ generates $D=\ceil{\frac{V+1}{\Psi}}$ shards.

For example, assume we have a dense array of mapped values as $[1,1.2,2.2,3,3.4,4]$, and we want to partition it into 2 parts, then we have $\boldsymbol{M}_p=[3]$ and $V+1=3$. In this case we will train $2$ linear regression functions. Suppose that the average number of keys in a shard is $\Psi=2$, then each $\mathcal{F}_i$ generates $D=\ceil{\frac{V+1}{\Psi}}=\ceil{\frac{3}{2}}=2$ shards.

Then with a given $x$, the predicted shard is given by $\mathcal{SP}(x)=\mathcal{F}_i(x)+i\times D$, where $i=\text{binary-search}(\boldsymbol{M}_p,x)$. More specifically, we first determine $i$ by using binary search. The result tells which interval this $x$ should belong to. Then we find the corresponding linear regression function $\mathcal{F}_i$ and calculate $\mathcal{F}_i(x)$, which is the predicted shard.

In the above example, given a key $x=2.2$, we first perform binary search in $\boldsymbol{M}_p$ and we found $i=1$. Then we find the first linear regression function $\mathcal{F}_1$ and calculate $\mathcal{F}_1(x)$. Since each linear regression function will yield $D=\ceil{\frac{V+1}{\Psi}}=2$ shards, the shards that the first linear regression function generates will be from $0$ to $1$ and the shards that the second linear regression function generates will be from $2$ to $3$. Hence, the predicted shard id is given by 
$$
\mathcal{SP}(x)=\mathcal{F}_i(x)+i\times D
$$

Then the problem left is to train the linear regression functions $\mathcal{F}_i$. Let $\boldsymbol{x}=(x_0,\cdots,x_v)$ be the keys' mapped value that fall in $[m_{i-1}, m_i)$. Suppose that $\boldsymbol{x}$ is sorted, i.e. $x_i\leq x_j, \forall 0\leq i<j\leq v$. Let $\boldsymbol{y}=(0,\cdots, V)$. Then we build a piecewise linear regression function $f_i$ with inputs $\boldsymbol{x}$ and ground truth $\boldsymbol{y}$. For a given point with mapped value $m\in[m_{i-1}, m_i)$, its shard id is given by $\ceil{\frac{f_i(m)}{\Psi}}+i\times D$, i.e. $\mathcal{F}_i(x)=\frac{f_i(m)}{\Psi}$.

In our previous example, in the interval $[0,3)$, we have $\boldsymbol{x}=(1,1.2,2.2)$ and $\boldsymbol{y}=(0,1,2)$. Then for a point with the mapped value $m=1.2$, the expected output will be $f_i(m)=1$ and the shard id is given by $\ceil{\frac{1}{2}}+0\times 2=1$. Hence, the point with mapped value $m=1.2$ will be allocated to the first shard. Then the problem is to train a continuous piecewise linear regression function in each interval. We constrain the piecewise linear regression function to be continuous so that it is guaranteed be monotonic.

Formally, a piecewise linear function can be described as 

\begin{equation}
\label{piecewise_linear_function}
	f(x)= \begin{cases} 
      b_0+\alpha_0(x-\beta_0) & \beta_0\leq x < \beta_1 \\
      b_1+\alpha_1(x-\beta_1) &  \beta_1\leq x < \beta_2 \\
      \vdots \\
      b_\sigma+\alpha_\sigma(x-\beta_\sigma) &  \beta_\sigma\leq x \\
   \end{cases}
\end{equation}

In order to make this piecewise linear function continuous, the slopes and intercepts of each linear region depend on previous values. Formally, let $\bar{a}=b_0$, then Eq. (\ref{piecewise_linear_function}) reduces to

\begin{equation}
	\label{continuous_piecewise_linear_function}
	f(x)= \begin{cases} 
      \bar{\alpha}+\alpha_0(x-\beta_0) & \beta_0\leq x < \beta_1 \\
      \bar{\alpha}+\alpha_0(x-\beta_0) + \alpha_1(x-\beta_1) &  \beta_1\leq x < \beta_2 \\
      \cdots \\
      \bar{\alpha}+\alpha_0(x-\beta_0) + \alpha_1(x-\beta_1)+\cdots+\alpha_\sigma(x-\beta_\sigma) &  \beta_\sigma\leq x \\
   \end{cases}
\end{equation}


Then to make Eq. (\ref{continuous_piecewise_linear_function}) monotonically increasing, we only need to ensure that $$\sum_{i=0}^\eta \alpha_i\geq 0, \forall 0\leq \eta\leq \sigma$$

Let $\boldsymbol{\alpha}=(\bar{\alpha},\alpha_0,\cdots,\alpha_\sigma)$, the square loss function $L(\boldsymbol{\alpha},\boldsymbol{\beta})=\sum_{i=1}^{V}(f(x_i)-y_i)^2$. We then optimise $\boldsymbol{\alpha}$ and $\boldsymbol{\beta}$ iteratively.

Assume that $\boldsymbol{\beta}=\hat{\boldsymbol{\beta}}=(\hat{\beta_0},\hat{\beta_1},\cdots,\hat{\beta_\sigma})$ is fixed, then $\boldsymbol{\alpha}$ can be regarded as the least square solution of the linear equation $\boldsymbol{A\alpha}=\boldsymbol{y}$, where

$$
\boldsymbol{A}=\left[\begin{array}{ccccc}
1 & x_{0}-\hat{\beta}_{0} & \left(x_{0}-\hat{\beta}_{1}\right) 1_{x_{0} \geq \hat{\beta}_{1}} & \ldots & \left(x_{0}-\hat{\beta}_{\sigma}\right) 1_{x_{0} \geq \hat{\beta}_{\sigma}} \\
1 & x_{1}-\hat{\beta}_{0} & \left(x_{1}-\hat{\beta}_{1}\right) 1_{x_{1} \geq \hat{\beta}_{1}} & \ldots & \left(x_{1}-\hat{\beta}_{\sigma}\right) 1_{x_{1}} \geq \hat{\beta}_{\sigma} \\
\vdots & \vdots & \vdots & \ddots & \vdots \\
1 & x_{N}-\hat{\beta}_{0} & \left(x_{V}-\hat{\beta}_{1}\right) 1_{x_{V} \geq \hat{\beta}_{2}} & \cdots & \left(x_{V}-\hat{\beta}_{\sigma}\right) 1_{x_{V} \geq \hat{\beta}_{\sigma}}
\end{array}\right]$$

where $1_{x_{0} \geq \hat{\beta}_{1}}$ equals to $1$ if ${x_{0} \geq \hat{\beta}_{1}}$, otherwise it equals to $0$.

We have
\begin{equation}
 \begin{split}
	L(\boldsymbol{\alpha},\boldsymbol{\beta}) 
	 =(\boldsymbol{y-A\alpha})^T(\boldsymbol{y-A\alpha}) 
	&=\boldsymbol{y}^T\boldsymbol{y}-\boldsymbol{\alpha}{^T}\boldsymbol{A}^T\boldsymbol{y}-\boldsymbol{y}^T\boldsymbol{A\alpha}+\boldsymbol{\alpha}^T\boldsymbol{A}^T\boldsymbol{A\alpha} \\
	& = \boldsymbol{y}^T\boldsymbol{y}-2\boldsymbol{\alpha}^T\boldsymbol{A}^T\boldsymbol{y}+\boldsymbol{\alpha}^T\boldsymbol{A}^T\boldsymbol{A}\boldsymbol{\alpha}
\end{split}
\end{equation}

and if we let

\begin{equation}
\label{alpha_form}
	\begin{split}
		\frac{\partial L(\boldsymbol{\alpha}, \boldsymbol{\beta})}{\boldsymbol{\alpha}}=2\boldsymbol{A}^T\boldsymbol{A}\boldsymbol{\alpha}-2\boldsymbol{A^T}\boldsymbol{y}=0 \\ \implies 
		\boldsymbol{\alpha}=(\boldsymbol{A}^T\boldsymbol{A})^{-1}\boldsymbol{A}\boldsymbol{y}
	\end{split}
\end{equation}


we get the $\boldsymbol{\alpha}$ with the given fixed $\boldsymbol{\beta}$. Clearly, different $\boldsymbol{\beta}$ give rise to different optimal parameters. Let $\boldsymbol{\alpha^\star}(\boldsymbol{\beta})$ be the optimal $\boldsymbol{\alpha}$ for a particular $\boldsymbol{\beta}$, then we want to find $\boldsymbol{\beta}$ such that


\begin{equation}
	L(\boldsymbol{\alpha^\star}(\boldsymbol{\beta^\star)}, \boldsymbol{\beta^\star})=\text{min}\{L(\boldsymbol{\alpha^\star}(\boldsymbol{\beta)}, \boldsymbol{\beta}) | \boldsymbol{\beta\in\mathbb{R}^{\sigma+1}}\}
\end{equation}

For $\boldsymbol{\beta}$, we define $\boldsymbol{r}=\boldsymbol{A\alpha-y}$ and 

$$
\boldsymbol{K}=\text{diag}(\bar{\alpha},\alpha_0, \cdots, \alpha_\sigma), \boldsymbol{G}=\begin{bmatrix}
 -1 & -1 & \cdots & -1 \\
  p_0^{(0)} & p_0^{(1)} & \cdots & p_0^{(V)} \\
  p_1^{(0)} & p_1^{(1)} & \cdots & p_1^{(V)} \\
  \vdots & \vdots & \ddots & \vdots \\
  p_\sigma^{(0)} & p_\sigma^{(1)}& \cdots & p_\sigma^{(V)} \\
\end{bmatrix}
$$

where $p_i^{(l)}=-1_{x_l\geq \beta_i}$. Then

$$
\boldsymbol{KG}=\begin{bmatrix}
 -\bar{\alpha} & 0 & \cdots & 0 \\
 0 & \alpha_0p_0^{(1)} & \cdots  & 0 \\
 \vdots & \vdots & \ddots & \vdots \\
 0 & 0 & \cdots & \alpha_\sigma p_\sigma^{(V)}
\end{bmatrix}
$$

then we have 

\begin{equation}
	g=\frac{\partial L(\boldsymbol{\alpha},\boldsymbol{\beta})}{\partial \boldsymbol{\beta}}=2\boldsymbol{KGr},
	Y=\frac{\partial g}{\partial \boldsymbol{\beta}}=2\boldsymbol{KGG}^T \boldsymbol{K}^T
\end{equation}
\textcolor{red}{Show how these are calculated}

As $g=\nabla_{\boldsymbol{\beta}} L$, $-g$ specifies the steepest descent direction of $\boldsymbol{\beta}$ for $L$. However, the convergence rate of $-g$ is low as it does not consider the second order derivative of $L$. Hence, we perform the update along the direction of $s=-\boldsymbol{Y}^{-1}g$.

\textcolor{red}{Show that Y is positive definite and explain why it matters}  
%TODO: Show that Y is positive definite and explain why it matters

In the beginning, we set $\beta^{(0)}=x_0$ and $\beta_i^{(0)}=x_{\floor{i\times \frac{V}{\Psi}}}, \forall i\in[1,\sigma]$. Then we can obtain $\boldsymbol{\alpha}$ by solving Eq. (\ref{alpha_form}). Then at each step, we perform a grid search to find the step $lr^{(k)}$ such that the loss $L$ is minimal. Then at the next iteration, we increase $k$ by one and set 

$$
\boldsymbol{\beta}^{(k+1)}=\boldsymbol{\beta}^{(k)} + lr^{(k)}s^{(k)}
$$

The iteration continues until $L$ converges, i.e. the $|L^{(k)}-L^{(k-1)}|<\delta$ where $\delta$ is a pre-set hyperparameter.


























