For a range query $\mathcal{Q}(\boldsymbol{l},\boldsymbol{u})$, we first find the cells that overlap with $\mathcal{Q}$. Then we decompose $\mathcal{Q}$ into the union of smaller query rectangles $\bigcup \mathcal{Q}_i$ such that each smaller query rectangles intersects only one cell, as shown in the Fig. \ref{fig:Range_Query_LISA}.

 Suppose that $\mathcal{Q}=\bigcup \mathcal{Q}_i$ where $\mathcal{Q}_i=[l_{i_0}, u_{i_o})\times [l_{i_1}, u_{i_1})$, i.e. we have $\mathcal{Q}_i$ representing the $i$th smaller query rectangles of one cell $C_j$.
 
 Then we can calculate the mapped values of $\mathcal{Q}_i$, i.e. $\mathcal{M}(l_{i_0}, l_{i_1})$ and $\mathcal{M}(u_{i_0}, u_{i_1})$. For simplicity, we use $m_l^{(i)}$ and $m_u^{(i)}$ to denote $\mathcal{M}(l_{i_0}, l_{i_1})$ and $\mathcal{M}(u_{i_0}, u_{i_1})$ respectively.
 
After creating corresponding mapped values, we then apply the shard prediction function $\mathcal{SP}(m_{l}^{i})$ and $\mathcal{SP}(m_{u}^{i})$ to predict the shard that could possibly contain keys that lie in the query rectangle $\mathcal{Q}_i$. Then in each shard, we perform a sequential search to find the desired keys. 

\begin{figure}[!htb]
    \centering
    \includegraphics[width=\textwidth]{graphs/range_query_lisa.png}
    \caption{Range Query Search in LISA} 
    %1) Find the cells that overlap with query rectangle qr. \\
    %2) Decompose qr into the unions of smaller query rectangles, each of which intersect one only one cell. \\
    %3) Find shards corresponding to lower and upper coordinates for each query rectangle, and perform a sequential search.}
    \label{fig:Range_Query_LISA}
\end{figure}

\begin{mscexample}
	In Fig. \ref{fig:Range_Query_LISA}, we present a range query example. Following steps show how we perform range query represented by the red rectangle.
	\begin{enumerate}
		\item Find the cells that overlap with query rectangle qr. In our example, 3 cells are overlapping with the query rectangle. 
		\item Decompose qr into the unions of smaller query rectangles, each of which intersect one only one cell as represented by $qr_{0}^{'}$, $qr_{1}^{'}$ and $qr_{2}^{'}$ in Fig. \ref{fig:Range_Query_LISA}.
		
		\item Calculate mapped values of $qr_{0}^{'}$, $qr_{1}^{'}$ and $qr_{2}^{'}$ vertices represented by \\
		$m_{l}^{0}, m_{u}^{0}, m_{l}^{1}, m_{u}^{1}, m_{l}^{2}, m_{u}^{2} $.
		\item Find shards corresponding to lower and upper coordinates for each smaller query rectangle represented by 
		$\mathcal{SP}(m_{l}^{0}),SP{(m_{u}^{0})},{SP}(m_{l}^{1}),SP{(m_{u}^{1})}, {SP}(m_{l}^{2}),SP{(m_{u}^{2})} $
		\item Do a sequential search for  each shard and collect the keys that fall in the query rectangle. 
	\end{enumerate}
\end{mscexample}
%TODO: Below are texts in the image caption, put them in main text.

%1) Find the cells that overlap with $\mathcal{Q}(\boldsymbol{l}, \boldsymbol{u})$. \\
%    2) Decompose $\mathcal{Q}(\boldsymbol{l}, \boldsymbol{u})$ into the unions of smaller query rectangles, each of which intersect only one cell. \\
%    3) Find shards corresponding to lower and upper coordinates for each $\mathcal{Q}(\boldsymbol{l}, \boldsymbol{u})$, and perform a sequential search. 