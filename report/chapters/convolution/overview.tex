From the previous discussion, we summarise that both one-dimensional and two-dimensional indexes requires to learn a function from a one-dimensional array. 
\begin{enumerate}
	\item In one dimensional data, the learned function is used directly as an approximation to the CDF. We use a fully connected neural network or a recursive model to learn such function.
	\item In two dimensional data, the learned function is used to predict the corresponding shard. We use a piecewise linear function to achieve this task.
\end{enumerate}

These models in our previous chapters have their shortcomings:

\begin{enumerate}
	\item The fully connected neural network (with ReLU as activation functions) is essentially a continuous piecewise linear function. The training of such neural network is unstable and highly dependent on the initial values, well-tuned hyper-parameters, etc. Meanwhile, it requires more work if we want to ensure the monotonicity.
	\item The recursive model takes more memory than a single neural network, and there are many more parameters to tune than neural network. 
	\item The training method for a piecewise linear function is an iteration-based method. If we choose a large number of break points, then the training time will be long.
\end{enumerate}

In fact, in order to train the piecewise linear function, we only need to know either the slope of each segments or the position of the breakpoints. If we know any one of them, we can learn the other in a closed form, as we shown in the shard prediction section. In this chapter, we present a method to learn the position of breakpoints.

Learning the position of breakpoints can be regarded as a binary point-wise classification problem. That means, for each point in our $X$, we want to learn to classify it to be $1$ if it is a breakpoint, otherwise we want it to be $0$. Then we classify all the points in our $X$ and each point is classified to be $1$ with a confidence. Afterwards, we only need to filter the top-$K$ points to be the breakpoints.

This task is similar to the image segmentation in computer vision, which essentially tries to classify every pixel in the image. One successful technique in image segmentation is by using the convolution and convolution transpose network \cite{long2015fully}. Inspired by this, we propose a similar network to perform binary point-wise classification. Due to the page limit, we will only describe the most important steps in this chapter.

\section{Training}

